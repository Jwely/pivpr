
\section{Summary}

Axial wake vortices were created with a bi-wing vortex generator in a low speed 
wind tunnel, at free stream velocities between 15 and 33 $m/s$. Particle image 
velocimetry was used to resolve the three dimensional vector field in seven 
slices between 5.4 and 9.5 chord lengths down stream. The core of these 
vortices was observed to periodically ingest turbulent energy and squeeze it to 
within one half of a core radii. The cyclical ingestion of turbulence was shown 
to have the effect of significantly reducing the core radius. If this phenomena 
persists for the life of the vortex, it could provide an explanation for the 
longevity of the azimuthal velocity component observed in natural wake 
vortices. Repeated patterns in radial and azimuthal velocity fluctuations 
suggested turbulent structures with periodicity near 60Hz, though insufficient 
data was gathered to discern the rate at which these structures decay.
Velocity profiles given by an exact solution to the Navier Stokes equations 
from non-equilibrium pressure theory were found to accurately predict 
the velocity profiles of the experimental data set \cite{ash2011}.

\section{recommendations for future work}
Evidence was found that periodic motion with frequencies between 50 and 130 Hz 
was present in the axial wake vortices. The present study was not able to 
resolve velocity fluctuations over a sufficient range of time and length scales 
to meaningfully explore the power spectral density of turbulent energies. If 
future work is to be performed with PIV, a sampling rate of at least 300Hz 
conducted over a period of 10 seconds should be sufficient. The reduced run 
time will help to mitigate the effects of environmental changes over the 
duration of tests. Additionally, PIV 
data need only be taken at one or two stations, but further downstream than in 
the present study. A test at 10 and 20 chord lengths down stream over a free 
stream velocity range of 15 to 40 $m/s$ in 0.5 $m/s$ increments could allow a 
more detailed view of the evolution of vortex turbulence structures by 
comparing velocity fluctuation profiles between vortices of different ages. 
Furthermore, fewer interrogation stations and corresponding PIV geometries will 
simplify the Monte Carlo uncertainty analysis significantly. The use of a 
bi-wing vortex generator without a center body may result in a faster merging 
of the co-rotating vortex pair, to accommodate testing geometries in wind 
tunnels short test sections such as the ODU low speed wind tunnel.
