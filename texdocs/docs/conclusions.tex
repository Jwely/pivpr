
\section{Summary}

Axial wake vortices were created with a bi-wing vortex generator in a low speed 
wind tunnel, at free stream velocities between 15 and 33 $m/s$. Particle image 
velocimetry was used to resolve the three dimensional vector field in seven 
slices between 5.4 and 9.5 chord lengths down stream. The core of these 
vortices was observed to periodically ingest turbulent energy and squeeze it to 
within one half of a core radii. The cyclical ingestion of turbulence was shown 
to have the effect of significantly reducing the core radius. If this phenomena 
persists for the life of the vortex, it could provide an explanation for the 
longevity of the azimuthal velocity component observed in natural wake vortices.
Velocity profiles given by an exact solution to the Navier Stokes equations 
from non-equilibrium pressure theory were found to accurately predict 
the velocity profiles of the experimental data set \cite{ash2011}.

\section{recommendations for future work}
The present study was not able to resolve velocity fluctuations over a 
sufficient range of time and length scales to meaningfully explore the power 
spectral density of turbulent energies.
