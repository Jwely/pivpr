\thispagestyle{plain}
\begin{center}
    \Large
    \textbf{Experimental Investigation of Turbulent Structures and 
    Non-Equilibrium Effects in Axial Wake Vortices via Particle Image 
    Velocimetry}
    
    \vspace{0.4cm}
    \large
%    Thesis Subtitle
    
    \vspace{0.4cm}
    Jeffry William Ely
    
    \vspace{0.9cm}
    \textbf{Abstract}
\end{center}


Vortices are a common phenomenon in fluid flows that arise to dissipate kinetic 
energy into heat via viscous interaction. They arise naturally at large scales 
in the form of dust devils, tornadoes, and in the wake of aircraft. It is 
important to understand the conditions leading to their formation, their 
duration, and their dissipation to prevent undesirable effects. Among these 
effects is an decrease in safety of aircraft operations in the wake of other 
aircraft, an extremely common situation at airports around the world. A large 
number of mathematical models and experimental data exists to help explain 
various aspects of axial wake vortex behavior, but current models fail to 
understand why many vortices remain tightly wound about their cores for as long 
as they have been observed to do. The current study builds upon the theoretical 
work of Ash, Zardadkhan and Zuckerwar \cite{ash2011}, and tests specific 
attributes of a vortex for agreement with non-equilibrium pressure relaxation 
theory. A bi-wing vortex generator was constructed, and placed in a low speed 
wind tunnel. Stereo particle image velocimetry was employed to map three 
dimensional velocity vectors in a plan perpendicular to the free stream flow 
direction at a rate of 1Hz for 200 seconds. A Reynolds time averaging approach 
was used to synthesize this velocity data into stable and fluctuating 
components. Free stream velocity and downstream distance were varied to create 
a variety of vortices, and to ensure the availability of high quality data.

