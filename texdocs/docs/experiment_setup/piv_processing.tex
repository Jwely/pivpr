\section{Stereo PIV Data processing}

PIV data were processed into text files containing lists of vectors and 
positions from raw image data using 
INSIGHTINSIGHT\textsuperscript{\textcopyright} software. 
This format was considered to be the 
starting point for the present analysis. Each vector is recorded as a row in a 
"v3d" file, with a set of position coordinates,velocity 
vector components, and two quality control flags that were used to reduce 
spurious vector counts. An example of this data is shown in Table 
\ref{table:v3d_row_example}.

\begin{table}[H]
\begin{center}
\begin{tabular}{|cccccccc|}
	\hline
	X mm & Y mm  & Z mm & U m/s & V m/s & W m/s & CHC & Residual Pixels\\
	\hline
	8.02046 & 0.174553 & 0 & 0.415872 & -2.25951 & 13.5873 & 1 & 0.127058\\
	9.74656 & 0.174553 & 0 & 0.386507 & -2.4523 & 13.9244 & 1 & 0.166965\\
	11.4727 & 0.174553 & 0 & 1.01919 & -2.8773 & 14.9454 & 1 & 0.0480147\\
	13.1988 & 0.174553 & 0 & 1.30872 & -3.02836 & 15.2081 & 1 & 0.0560525\\
	\hline
\end{tabular}
\caption{Example rows from v3d files with raw 3d vector data}
\label{table:v3d_row_example}
\end{center}
\end{table}


The position coordinates are expressed in units of millimeters away from 
the fiducial mark on the target used for calibration. These coordinates are 
expressed from the viewpoint of the cameras, which were pointed upstream; 
positive $X$ coordinates are to the right, positive $Y$ is upwards, and 
positive $Z$ is towards the cameras. Vector components are recorded as $U$, 
$V$ and $W$. 

\subsection{Reynolds Averaging}
Since each run contained 200 separate vector sets, the data needed to be 
synthesized into unsteady velocity can enable examination of the Reynolds 
Averaged Navier Stokes equations in cylindrical coordinates. This processing 
was performed employing \textit{Python 2.7} and 
\textit{Matlab}, with much of the code entirely replicated in each language. 
First, text files with tables of vector data were loaded into a parser to 
construct a mesh grid of the $X$, $Y$ coordinate space in a given plane. These 
mesh grids established the relationship between matrix indices and real 
coordinate space with units of $mm$.Then, for each test as referenced in Table 
\ref{table:test_matrix_table}, data from 
each of the 200 snapshots is Reynolds averaged to produce a mean velocity 
component, and an associated fluctuating velocity components, that is

\begin{equation}
u(x,t) = \overline{u}(x) + u^\prime(x,t)
\label{eq:rans_components}
\end{equation}


Each component is given an individual variable name, with mean nonfluctuating 
components determined employing simple averages of all sets expressed as the 
component letter with an over bar ($\overline{v_x}$, $\overline{v_y}$, 
$\overline{v_z}$), and 
time averaged fluctuating component expressed with a prime 
($\overline{v_x^\prime}$, $\overline{v_y^\prime}$, $\overline{v_z^\prime}$). 
Velocities referring to just one of the 200 sets will use a 
subscript $i$ as ($v_{x_i}$, $v_{y_i}$, $v_{z_i}$). The stable components are 
computed with 
a simple average to preserve the sign, while the fluctuating components are 
computed with a root mean squared method as shown in equations \ref{eq:ubar} 
through \ref{eq:wprime}.

\begin{equation}
\overline{v_x}  = \frac{1}{N} \sum_{i=1}^{200} v_{x_i}
\label{eq:ubar}
\end{equation}

\begin{equation}
\overline{v_y}  = \frac{1}{N} \sum_{i=1}^{200} v_{y_i}
\end{equation}

\begin{equation}
\overline{v_z}  = \frac{1}{N} \sum_{i=1}^{200} v_{z_i}
\end{equation}

\begin{equation}
\overline{v_x^\prime} = \sqrt{\frac{1}{N-1} \sum_{i=1}^{200} (v_{x_i} - 
\overline{v_x})^2}
\end{equation}

\begin{equation}
\overline{v_y^\prime} = \sqrt{\frac{1}{N-1} \sum_{i=1}^{200} (v_{y_i} - 
	\overline{v_y})^2}
\end{equation}

\begin{equation}
\overline{v_z^\prime} = \sqrt{\frac{1}{N-1} \sum_{i=1}^{200} (v_{z_i} - 
	\overline{v_z})^2}
\label{eq:wprime}
\end{equation}

where $\overline{v_x}$, $\overline{v_y}$ and $\overline{v_z}$ are the time 
averaged velocity 
components in the $X$, $Y$, $Z$ directions respectively, and  
$\overline{v_x^\prime}$, $\overline{v_y^\prime}$ and $\overline{v_z^\prime}$ 
are the time averaged fluctuating velocity components.

Similarly, time averaged Reynolds stresses can be calculated by multiplying the 
fluctuating components together before taking the root mean square average by 
equations \ref{eq:rs_uu} through \ref{eq:rs_vw}.

\begin{equation}
\overline{v_x^\prime v_x^\prime} = \sqrt{\frac{1}{N-1} \sum_{i=1}^{200} 
	((v_{x_i} - \overline{v_x})(v_{x_i} - \overline{v_x}))^2}
\label{eq:rs_uu}
\end{equation}

\begin{equation}
\overline{v_y^\prime v_y^\prime} = \sqrt{\frac{1}{N-1} \sum_{i=1}^{200} 
	((v_{y_i} - \overline{v_y})(v_{y_i} - \overline{v_y}))^2}
\end{equation}

\begin{equation}
\overline{v_z^\prime v_z^\prime} = \sqrt{\frac{1}{N-1} \sum_{i=1}^{200} 
	((v_{z_i} - \overline{v_z})(v_{z_i} - \overline{v_z}))^2}
\end{equation}

\begin{equation}
\overline{v_x^\prime v_y^\prime} = \sqrt{\frac{1}{N-1} \sum_{i=1}^{200} 
	((v_{x_i} - \overline{v_x})(v_{y_i} - \overline{v_y}))^2}
\end{equation}

\begin{equation}
\overline{v_x^\prime v_z^\prime} = \sqrt{\frac{1}{N-1} \sum_{i=1}^{200} 
	((v_{x_i} - \overline{v_x})(v_{z_i} - \overline{z_y}))^2}
\end{equation}

\begin{equation}
\overline{v_y^\prime v_z^\prime} = \sqrt{\frac{1}{N-1} \sum_{i=1}^{200} 
	((v_{y_i} - \overline{v_y})(v_{z_i} - \overline{z_y}))^2}
\label{eq:rs_vw}
\end{equation}

Turbulent kinetic energy $k$ can be calculated as one half of the sum of the 
squares of the three instantaneous fluctuation components, taken over the set, 
i.e

\begin{equation}
k = \frac{1}{2} \left(\overline{v_x^\prime v_x^\prime}^2 + 
	\overline{v_y^\prime v_y^\prime}^2 + 
	\overline{v_z^\prime v_z^\prime}^2\right)
\label{eq:tke}
\end{equation}


Once all the turbulence statistics have been 
generated in Cartesian coordinates, the 
vortex core must be located in that survey plane in order to represent the 
turbulence data in cylindrical coordinates. The core was established in terms 
of the minimum in-plane velocity magnitudes near the center of the 
interrogation plane, and excluding the stream-wise $w$ component. In the 
present case, it was necessary to avoid confusion produced by spurious edge 
values, and a threshold value was defined that limited the search for in-plane 
velocity minima to the area surrounding the center of the field of view. Once 
the lowest value was found, sub-grid accuracy was 
achieved by taking utilizing cubic interpolation of the grid points surrounding 
the minimum. This technique was sufficiently robust to automatically establish 
the core axis correctly for every experiment.

Employing the location for the vortex core, new mesh grids in radial and 
tangential coordinates, $R$ and $\theta$, were created from the $X$ and 
$Y$ mesh grids. Velocity components, $v_r$ and $v_\theta$, were then 
calculated by equations \ref{eq:uv_r} and \ref{eq:uv_t}.

\begin{equation}
v_{r} = v_{x} \cos{(\theta)} + v_{y} \sin{(\theta)}
\label{eq:uv_r}
\end{equation}

\begin{equation}
v_{\theta} = v_{x} \sin{(\theta)} - v_{y} \cos{(\theta)}
\label{eq:uv_t}
\end{equation}

where $v_{r}$ and $v_{\theta}$ represent a radial and tangential velocity 
matrix, and $\theta$ is the mesh grid of angles 
about the vortex core. These equations were applied on an element wise basis to 
every grid point in the vector field. Once this was done, the same Reynolds 
averaging techniques were applied to extract the radial and tangential velocity 
components $v_{r}$ and $v_{\theta}$ into mean and fluctuating components in 
equations 
\ref{eq:rbar} through \ref{eq:tprime}.

\begin{equation}
\overline{v_r}  = \frac{1}{N} \sum_{i=1}^{200} v_{r_i}
\label{eq:rbar}
\end{equation}

\begin{equation}
\overline{v_\theta}  = \frac{1}{N} \sum_{i=1}^{200} v_{\theta_i}
\label{eq:tbar}
\end{equation}

\begin{equation}
\overline{v_r^\prime} = \sqrt{\frac{1}{N-1} \sum_{i=1}^{200} (v_{r_i} - 
\overline{v_r})^2}
\label{eq:rprime}
\end{equation}

\begin{equation}
\overline{v_\theta^\prime} = \sqrt{\frac{1}{N-1} \sum_{i=1}^{200} (v_{\theta_i} 
- 
\overline{v_\theta})^2}
\label{eq:tprime}
\end{equation}

The Reynolds stress correlations in cylindrical coordinates were calculated 
directly from the transformed cylindrical velocities with equations 
\ref{eq:rs_rr) through \ref{eq:rs_tw}.
	
\begin{equation}
\overline{v_{r}^\prime v_{r}^\prime} = \sqrt{\frac{1}{N-1} \sum_{i=1}^{200} 
	((v_{r_i} - \overline{v_{r}})(v_{r_i} - \overline{v_{r}}))^2}
\label{eq:rs_rr}
\end{equation}

\begin{equation}
\overline{v_{\theta}^\prime v_{\theta}^\prime} = \sqrt{\frac{1}{N-1} 
\sum_{i=1}^{200} 
	((v_{\theta_i} - \overline{v_{\theta}})(v_{\theta_i} - 
	\overline{v_{\theta}}))^2}
\end{equation}

\begin{equation}
\overline{v_r^\prime v_\theta^\prime} = \sqrt{\frac{1}{N-1} \sum_{i=1}^{200} 
	((v_{r_i} - \overline{v_r})(v_{\theta_i} - \overline{v_\theta}))^2}
\end{equation}

\begin{equation}
\overline{v_r^\prime v_z^\prime} = \sqrt{\frac{1}{N-1} \sum_{i=1}^{200} 
	((v_{r_i} - \overline{v_r})(v_{z_i} - \overline{v_z}))^2}
\end{equation}

\begin{equation}
\overline{v_\theta^\prime v_z^\prime} = \sqrt{\frac{1}{N-1} \sum_{i=1}^{200} 
	((v_{\theta_i} - \overline{v_\theta})(v_{z_i} - \overline{v_z}))^2}
\label{eq:rs_tw}
\end{equation}

	
Once this processing was complete, a large set of Reynolds averaged 
velocity data were available for each of the 70 test cases. Plots of this data 
from experiment 55 will be employed in this thesis as a representative example. 
Stream line plots can be produced as shown in 
Figure \ref{fig:examp_stream}. Cartesian average velocity components for test 
case are shown in Figures \ref{fig:examp_U} through \ref{fig:examp_W}, although 
they are not particularly useful when compared to representations in 
cylindrical coordinates. 
Cylindrical average velocity components are shown in Figures \ref{fig:examp_R} 
and \ref{fig:examp_T}. Reynolds stress distributions are shown in Figures 
\ref{fig:examp_rt} through \ref{fig:examp_tw}. Variation of the fluctuating 
magnitudes of the individual cylindrical velocity components are shown in 
Figures \ref{fig:examp_rr} through \ref{fig:examp_ww}. Variation in the total 
turbulent kinetic 
energy $k$ is shown in Figure \ref{fig:examp_tke}. Scatter plots that 
flatten the dataset into one dimension can be generated as in the tangential 
velocity vs radial distance from the rotational axis is shown in Figure 
\ref{fig:examp_Tscatter}. 

\begin{figure}[H]
	\centering
	\includegraphics[width=4.25in]{figs/run_55/run_55_stream}
	\caption{Example stream plot.}
	\label{fig:examp_stream}
\end{figure}

\begin{figure}[H]
	\centering
	\includegraphics[width=4.25in]{figs/run_55/run_55_U_contour}
\caption{Example contour plot of $\overline{v_x}$.}
\label{fig:examp_U}
\end{figure}

\begin{figure}[H]
	\centering
	\includegraphics[width=4.25in]{figs/run_55/run_55_V_contour}
\caption{Example contour plot of $\overline{v_y}$.}
\label{fig:examp_V}
\end{figure}

\begin{figure}[H]
	\centering
	\includegraphics[width=4.25in]{figs/run_55/run_55_W_contour}
\caption{Example contour plot of $\overline{v_z}$.}
\label{fig:examp_W}
\end{figure}

\begin{figure}[H]
	\centering
	\includegraphics[width=4.25in]{figs/run_55/run_55_R_contour}
\caption{Example contour plot of $\overline{v_r}$.}
\label{fig:examp_R}
\end{figure}

\begin{figure}[H]
	\centering
	\includegraphics[width=4.25in]{figs/run_55/run_55_T_contour}
\caption{Example contour plot of $\overline{v_\theta}$.}
\label{fig:examp_T}
\end{figure}

\begin{figure}[H]
	\centering
	\includegraphics[width=4.25in]{figs/run_55/run_55_rt_contour}
\caption{Example contour plot of $\overline{v_r^\prime v_\theta^\prime}$.}
\label{fig:examp_rt}
\end{figure}

\begin{figure}[H]
	\centering
	\includegraphics[width=4.25in]{figs/run_55/run_55_rw_contour}
\caption{Example contour plot of $\overline{v_r^\prime v_z^\prime}$.}
\label{fig:examp_rw}
\end{figure}

\begin{figure}[H]
	\centering
	\includegraphics[width=4.25in]{figs/run_55/run_55_tw_contour}
\caption{Example contour plot of $\overline{v_\theta^\prime v_z^\prime}$.}
\label{fig:examp_tw}
\end{figure}

\begin{figure}[H]
	\centering
	\includegraphics[width=4.25in]{figs/run_55/run_55_rr_contour}
\caption{Example contour plot of $\overline{v_r^\prime v_r^\prime}$.}
\label{fig:examp_rr}
\end{figure}

\begin{figure}[H]
	\centering
	\includegraphics[width=4.25in]{figs/run_55/run_55_tt_contour}
\caption{Example contour plot of $\overline{v_\theta^\prime v_\theta^\prime}$.}
\label{fig:examp_tt}
\end{figure}

\begin{figure}[H]
	\centering
	\includegraphics[width=4.25in]{figs/run_55/run_55_ww_contour}
\caption{Example contour plot of $\overline{v_z^\prime v_z^\prime}$.}
\label{fig:examp_ww}
\end{figure}

\begin{figure}[H]
	\centering
	\includegraphics[width=4.25in]{figs/run_55/run_55_ctke_contour}
\caption{Example contour plot of turbulent kinetic energy $k$.}
\label{fig:examp_tke}
\end{figure}

\begin{figure}[H]
	\centering
	\includegraphics[width=6.5in]{figs/run_55/run_55_T_vs_r_mesh_ctkescatter}
\caption{Example scatter plot of $\bar{v_\theta}$, colored by $k$.}
\label{fig:examp_Tscatter}
\end{figure}

Insight to the quality and uncertainty in each measurement was gained by 
examining the number of vectors successfully extracted from the set of 200 
snapshots as shown in Figure \ref{fig:example_num_contour}. Locations with 
fewer than twenty valid vectors in each set were omitted. A clear 
relationship between the total turbulent kinetic energy, $k$, and the number of 
vectors making up the average was observed, and will be discussed in Chapter 
\ref{chapter:results}.

\begin{figure}[H]
	\centering
	\includegraphics[width=5in]{figs/example_vortex_figs/example_num_contour}
	\caption{Example contour plot showing the variation in the number of 
	measurements within the interrogation plane}
	\label{fig:example_num_contour}
\end{figure}
