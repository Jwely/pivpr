
\section{PIV Overview}

The present study uses a stereo PIV system which resolves three dimensional 
near-instantaneous velocity vectors gridded on a two dimensional cross section 
of flow. The cameras are capable of taking two images just a few microseconds 
apart. Determining velocity vectors requires two images just a few 
microseconds apart, but pairs of images can be taken at greater time intervals. 
The PIV method used in this study utilizes a "frame-straddling" technique that 
spaces the time interval between two laser pulses so that the camera sensor 
arrays receive the first laser illuminated image data as near as possible to 
the closing of the shutter, followed by a second laser illuminated image as 
close as possible to the beginning of the second shutter opening. This 
technique allows both cameras to capture image sequence pairs simultaneously, 
spaced 25 to 50 microseconds apart. This can be repeated once every second, 
resulting in a true velocity sampling frequency of 1 Hz.

\subsection{Seeding the Flow}

Appropriate particle seeding density and time between straddled frames is the
subject of continued study, and is difficult to predict \textit{a priori}. 
Complete coverage of a two dimensional vector field is highly dependent upon 
uniform optimal particle density conditions which are difficult to obtain, and 
maintain over an extended test interval, due to seed-particle accumulation. For 
stereo PIV, incomplete data in either of the two dimensional vector 
sets from either camera at a given spatial location will result in an 
indeterminate vector displacement in the three dimensional vector data. To 
elevate the likelihood that a displacement vector at a given location can be 
properly determined, an additional data refining technique outlined by 
\cite{hart1998} was employed. The Hart method compares correlation maps 
between adjacent vector spots to produce fewer errors than would otherwise be 
produced by completely independent evaluation of each small image sector. in 
instances where two adjacent regions lack a well-defined peak, the Hart method 
emphasizes shared 
peaks in order to reveal a significant correlation that might otherwise have 
been missed. In instances 
where sub optimal seeding conditions exist and a correlation map 
produces a false peak, the Hart method can isolate and eliminate those 
anomalous peaks based on comparison with adjacent sectors \cite{hart1998}. The 
actual process by which the Hart method reduces erroneous vectors is difficult 
to quantify on a case by case basis, but is expected to have a positive effect 
and reduce overall uncertainty in the PIV measurement.