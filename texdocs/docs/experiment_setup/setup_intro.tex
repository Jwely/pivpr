A commercial particle image velocimetry system was installed in the Old 
Dominion University low 
speed 
wind tunnel and employed to measure three-dimensional velocity fields produced 
by an axial vortex at multiple nominal velocities $V_{nom}$ and multiple 
interrogation planes. The interrogation planes were defined by their 
distance downstream of the bi-wing vortex generator center body
$I_Z$. Nominal wind tunnel velocity was varied from 15$m/s$ to 33$m/s$, 
sampling 10 distinct velocities in increments of 2$m/s$. The interrogation 
plane was moved at irregular intervals from 546$mm$ to 1016$mm$, (5.4 to 10 
chord lengths), as summarized in Table \ref{table:test_matrix_table}. This 
chapter 
discusses details of the experimental setup used to produce the datasets, 
including wind tunnel control, vortex generator setup, PIV system calibration, 
data acquisition, data processing, and data quality control.