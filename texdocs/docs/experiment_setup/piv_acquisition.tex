

\section{Stereo PIV Data Acquisition}

The low speed wind tunnel was outfitted with a stereo particle image 
velocimetry system which included two cameras that were mounted with a simple 
frame built from 80$mm$x120$mm$ T-slot extruded aluminum with six sliding 
fastener points on the exterior of the wind tunnel, just outside the test 
section on the left and right sides as shown in figure \ref{fig:pivsetup}. 

\begin{figure}[H]
	\centering
	\includegraphics[width=5in]{figs/piv_method/piv_camera_diagram}
	\caption{Top view schematic of PIV camera positions. A-vortex generator, 
	B-PIV 
		calibration target and interrogation plane, C-left camera, D-right 
		camera, E-station distance dimension.}
	\label{fig:pivsetup}
\end{figure}

The roof of the wind tunnel has a long glass window along the center of the 
test section such that an Nd:YAG laser could be mounted to create a vertically 
oriented laser sheet. This laser sheet functioned to illuminate the fluid flow 
perpendicular to the free stream velocity vector as 
shown in Figure \ref{fig:laser_sheet_picture}. In this image, the fluid flow 
has been freshly seeded with a wand to deposit concentrated fog directly 
downstream of the vortex generator to visualize a vortex with a clearly 
defined, smoke free, vortex core. Bright outer lines mark the edges of the 
light curtain. This was initially performed to aid in the alignment of a 
pressure probe (visible in the image) with the vortex core. PIV data was not 
taken with the fog wand or the pressure probe present in the wind tunnel.


The equipment used for this study was a "TSI Stereo Image Velocimeter System", 
which consists of a pair of TSI PIV 13-8 cameras (Figure 
\ref{fig:camera_picture}), a TSI synchronizer and frame 
grabber specific to the cameras (Figures \ref{fig:synchronizer} and 
\ref{fig:synchronizer2}), a New Wave Dual Mini-YAG Laser (Figure 
\ref{fig:laser}) and Light Guide, a precision 3-D calibration target for stereo 
PIV camera alignment, and a precision laser traverse system. The particle seed 
was generated by an MDG fog generator (Figure \ref{fig:fog_machine}). The TSI 
INSIGHT\textsuperscript{\textcopyright} software was used 
for data acquisition and image to vector processing.

\begin{figure}[H]
	\centering
	\includegraphics[width=5in]{figs/piv_method/laser_sheet_picture}
	\caption{Picture of a light curtain illuminating a cross section of an 
	axial wake vortex.}
	\label{fig:laser_sheet_picture}
\end{figure}

\begin{figure}[H]
	\centering
	\includegraphics[width=5in]{figs/piv_method/piv_cams}
	\caption{Photograph of two TSI PIV 13-8 cameras. Model 630047}
	\label{fig:camera_picture}
\end{figure}

\begin{figure}[H]
	\centering
	\includegraphics[width=5in]{figs/piv_method/synchronizer}
	\caption{Photograph of LaserPulse synchronizer by TSI. Model 610034}
	\label{fig:synchronizer}
\end{figure}

\begin{figure}[H]
	\centering
	\includegraphics[width=5in]{figs/piv_method/synchronizer_rear}
	\caption{Photograph of synchronizer connectors. Model 610034}
	\label{fig:synchronizer2}
\end{figure}

\begin{figure}[H]
	\centering
	\includegraphics[width=5in]{figs/piv_method/laser}
	\caption{Photograph of a PIV laser by SoloPIV. Model 610034}
	\label{fig:laser}
\end{figure}

\begin{figure}[H]
	\centering
	\includegraphics[width=5in]{figs/piv_method/fog_generator}
	\caption{Photograph MDG MAX 3000APS Fog Generator.}
	\label{fig:fog_machine}
\end{figure}

%\begin{figure}[H]
%	\centering
%	\includegraphics[width=5in]{figs/piv_method/INSIGHT_processing}
%	\caption{Photograph of INSIGHT software computing stereo velocity fields}
%	\label{fig:processing_screenshot}
%\end{figure}

\subsection{Synchronizing the Cameras and Laser}
In order to resolve particle displacements on such a small scale in a 
relatively fast moving fluid, the time interval between a laser pulse sequence 
must be tuned to allow sufficient particle displacement occur to obtain a 
meaningful velocity measurement. If the time interval ($dt$) between successive 
exposures was too long, the particles escaped the interrogation plane, and 
could not be tracked. This study varied the laser pulse timing 
between 25 microseconds for higher wind tunnel speeds, and 50 microseconds for 
slower vortex velocities associated with slower wind tunnel speed. As 
introduced in Chapter 1, this PIV system employed a frame straddling technique, 
which initiates the first shutter opening well before the first laser 
pulse begins such that the camera shutter first closes at as the first laser 
pulse is ending, but before the second laser pulse starts. The second exposure 
begins just as the second laser pulse is initiating, though it extends well 
beyond the termination of the first laser pulse as shown schematically in 
Figure \ref{fig:frame_straddling}. 

\begin{figure}
	\centering
	\includegraphics[width=5.5in]{figs/piv_method/frame_straddling}
	\caption{Frame straddling technique. Laser pulses are shown in green, and 
		camera exposures are shown in gray.}
	\label{fig:frame_straddling}
\end{figure}

Timing is achieved with a TSI synchronizer (Figure \ref{fig:synchronizer}) and 
control PC shown in Figure 
\ref{fig:pivblockdiagram}. The control PC sends a command to the synchronizer 
to take an image sample, then the synchronizer sends precisely timed signals to 
operate both cameras and the laser control unit. The cameras each store the 
image pair in an internal memory buffer before returning the data to the 
control PC.

\begin{figure}[H]
	\centering
	\includegraphics[width=4in]{figs/piv_method/experiment_block_diagram}
	\caption{Blockdiagram of PIV hardware components. C-left camera, D-right 
		camera, F-Nd:YAG laser, G-laser control unit and power supply, 
		H-synchronizer, 
		J-control PC running INSIGHT software suite.}
	\label{fig:pivblockdiagram}
\end{figure}

\subsection{Calibration employing a PIV Target}

Dimensional calibration was accomplished with the use of a calibration target, 
which was a 10$cm$ by 10$cm$ black calibration plate with precisely positioned 
white divots and a center fiducial mark ,as shown in figure 
\ref{fig:calibration_target}. The calibration target was positioned inside the 
tunnel test section and aligned with the desired interrogation plane. 
The laser illumination sheet was operated in a continuous firing mode and aimed 
into the test section so that the location of the interrogation sheet is 
clearly visible with the use of polarized eye protecting goggles. The camera 
facing surface of the 
interrogation target is then manually aligned with the interrogation plane, 
which was visible as an intense line of light with a nominal thickness of 
1.5$mm$ observable across the top of the target. Once the calibration target 
was properly aligned, the laser and all other light sources in the laboratory 
were extinguished and a lamp was placed downstream of the test section (to 
create ideal lighting conditions) in order to illuminate the target for camera 
focusing. Cameras outfitted with 50mm lenses were manually aimed at the 
calibration target, taking care to place the fiducial mark near the center of 
the image frame. Focusing was accomplished manually by adjusting each lens, and 
a limited PIV data set was created after focusing to ensure that the focus ws 
sharp enough to resolve a partially complete velocity vector field within the 
interrogation plane.

\begin{figure}[H]
	\centering
	\includegraphics[width=4in]{figs/piv_method/calibration_target}
	\caption{Inverted color and elevated contrast photograph of the calibration 
		target, highlighting the central fiducial mark.}
	\label{fig:calibration_target}
\end{figure}


INSIGHT\textsuperscript{\textcopyright} software was used to take snapshots and 
view them to ensure proper focus 
and alignment. Once satisfactory adjustments were obtained, calibration images 
were taken and the imaging software was used to recognize the fiducial mark, as 
well as each divot, in order to create the necessary two dimensional coordinate 
transformation maps that relate
pixel distance to physical distance. This calibration was employed both to 
determine the magnitudes of the two dimensional velocity vectors unique to each 
camera, and aligning the two separate vector fields, in order to create a three 
dimensional velocity field. Once calibration images were acquired, the 
target was removed and data were taken over a range of velocities in 
that interrogation plane. Each interrogation plane was defined by its position 
downstream relative to the trailing edge of the vortex generator center body. 
Seven interrogation planes were utilized in this study, and every time the 
survey plane was changed, it was necessary to repeat the calibration process. 
Ideally, as dictated by good design of experiments 
practice, the order of experiments should be randomized. This, however, would 
have required exclusive tunnel use for a longer period of time than could be 
scheduled due to the level of demand at the time.

\subsection{Seeding the Flow}

Gathering data with a PIV system requires the flow be properly seeded with 
observable particles. Particles should be sufficiently small to be entrained in 
the flow such that particle motion and fluid motion are the same. In the 
present study, $CO_2$ filled mineral oil bubbles generated with an MDG MAX 3000 
APS fog generator with a Laskin nozzle were used as particle seed. These 
droplets have a nominal diameter of 0.6 micrometers and densities of 1.55 times 
that of air were used \cite{mdgfog}. This was within the optimal range of 
particle densities for good particle entrainment and representation of the 
fluid flow as defined by \cite{mei1996}, as discussed in Chapter 1.

Each velocity measurement 
required a sector of pixels through which displacement could be tracked. If 
a majority of the illuminated particles were allowed to travel outside of a 
given sector between consecutive frame straddled image captures, a meaningful 
correlation map could not be generated. Care was taken to ensure that the 
observable particles 
traversed no more than 50\% of a sector width during the time interval between 
images by adjusting the time interval between image captures ($dt$). 
Utilization of a 50\% border when processing each sector, as described in the 
previous section, helped to ensure that particles did not escape the 
interrogation area. A sufficient quantity of particle groups must be visible in 
each sector in order to yield a meaningful correlation map. If the particle 
seed density was too low, or too high, it would result in a somewhat 
homogeneous image, with ambiguous peaks in the correlation map. A situation 
which results in an increased frequency of spurious measurements with a 
poor signal to noise ratio. Manipulation of sector size 
can increase the number of particles and the reliability of the correlation 
mapping, but only at the expense of reduced velocity field resolution, since 
more pixels are required to resolve each vector.


\subsection{Pressure Measurements}

Pressure measurements within the vortex core could assist in verifying the 
predicted non-eqiulirbium pressure influences on the behavior of axial wake 
vortices, especially when combined with simultaneous three dimensional 
turbulence measurements with PIV \cite{ash2011}.


Initially, it was believed that a pressure probe could be positioned 
along the axis of the vortex core. Without positioning the probe in many 
locations across the flow and post-processing the data set, it was necessary to 
have some other way to verify that the pressure probe was in fact positioned 
within the vortex core. Simultaneous PIV and pressure measurements were 
attempted utilizing a seven hole probe, but the probe and probe mounting system 
was found to severely obstruct the PIV laser sheet and camera views, and 
created bright reflections that saturated the image values in the vicinity of 
the vortex. No usable PIV data could be obtained when the a pressure probe was 
mounted in the tunnel test section. Direct visual inspection of the actual 
vortex core employing a continuous laser sheet, revealed that the core appeared 
to dilate significantly in the presence of the pressure probe. Without a method 
to compensate for this potential core dilation, pressure measurements of the 
vortex core region have not been included in this study. It is possible that, 
with a much larger vortex core which is expected to be less 
sensitive to the presence of an inserted pressure probe, and when coated with 
a non-reflective coating utilizing forward mounted PIV cameras looking 
downrange, simultaneous vortex core pressure measurements and PIV measurements 
could be obtained.

\subsection{Capturing PIV Imagery}

The experiments were conducted with the interrogation plane at seven 
downstream  between and 546$mm$ and 1016$mm$. Each of these downstream 
locations will be referred to as 
a "station". Tests were conducted starting nearest to the vortex 
generator and progressing downstream. Since changing the position of the 
interrogation plane required a complete re-calibration of the PIV system, the 
testing sequence was not randomized. For each station, data were taken for each 
of 
the ten test velocities between 15$m/s$ and 33$m/s$ in ascending order, gaining 
speed over time. Image pairs were acquired once per second, at a 
rate of $1Hz$ for a period of 200 seconds, generating a total of 200 image 
sets. Each test set of images was then processed using INSIGHT software in 
order to produce text files containing three-dimensional vector fields for each 
image set.

Tunnel conditions were monitored closely and recorded for the duration of each 
test. Ancillary wet bulb and dry bulb temperatures were taken with a sling 
psychrometer. The amount of mineral oil fog (seed) was not controllable in a 
quantitative manner over an extended period of testing, because the dissipation 
rate was not constant. Additional fog was added to the tunnel through a hose in 
the tunnel wall, far upstream of the high speed test section, on a qualitative 
basis. Prior to actual PIV data acquisition, a few test PIV images 
were captured, and a low quality two dimensional computation was performed real 
time for a heads-up evaluation of data quality. 
On occasion, low data quality indicated a low particle density in the tunnel, 
so additional particle fog was added. 
Specific details for each test including the relative humidity ($\phi$) derived 
from wet bulb and dry bulb measurements and associated pressure relaxation 
coefficient 
($\eta_P$), are summarized in Tables \ref{table:station_1_measurements} through 
\ref{table:station_7_measurements}. 

Stations are defined by the position of the interrogation plane, measured 
downstream from the wing edge of the bi-wing vortex generator, expressed as
$I_Z$. Nominal velocity is denoted by $V_{nom}$, while actual mean free stream 
velocity is denoted by $V_{fs}$, and variance in the measurement for the 
duration of the test is labeled $\sigma_{V_{fs}}$. Dynamic pressure is $Q$, 
atmospheric 
pressure is $P_{atm}$, and the tunnel temperature is $T_{tunnel}$. Relative 
humidity $\phi$ was calculated from wet bulb and dry bulb temperatures ($T_w$ 
and $T$, respectively) by 
equations \ref{eq:rh_es} through \ref{eq:rh_rh} according to \cite{owen1977}. 
Finally, pressure relaxation coefficient is calculated according to the table 
in \cite{ash2011}.

\begin{equation}
e_S = 6.112 exp \left( \frac{17.67 T}{T + 243.5} \right)
\label{eq:rh_es}
\end{equation}

\begin{equation}
e_W = 6.112 exp \left( \frac{17.67 T_W}{T_W + 243.5} \right)
\label{eq:rh_ew}
\end{equation}

\begin{equation}
e = e_W - P_{atm} (T - T_W) 0.00066(1 +( 0.00115T_W))
\label{eq:rh_e}
\end{equation}

\begin{equation}
\phi = 100 \frac{e}{e_S}
\label{eq:rh_rh}
\end{equation}




\renewcommand\baselinestretch{1.3}\selectfont
\begin{table}[H]
\begin{center}
\begin{tabular}{|ccccccccccc|}
	\hline
	Run & $I_Z$ & $V_{nom}$ & $dt$ & $V_{fs}$ & $\sigma_{V_{fs}}$ & $Q$ & $P_{atm}$ & $T_{tunnel}$ & $\phi$ & $\eta_P$\\
	$ID$ & $(mm)$ & $(m/s)$ & $(\mu s)$ & $(m/s)$ & $(m/s)$ & $(Pa)$ & $(Pa)$ & $(\degree K)$ & $(\%)$ & $(\mu s)$\\
	\hline
	1 & 546 & 15 & 50 & 15.22 & 0.02 & 135 & 102036 & 299.85 & 60.4 & 0.35\\
	2 & 546 & 17 & 50 & 16.88 & 0.02 & 170 & 102115 & 297.55 & 66.3 & 0.329\\
	3 & 546 & 19 & 50 & 19.43 & 0.01 & 225 & 102105 & 297.55 & 66.3 & 0.329\\
	4 & 546 & 21 & 50 & 21.06 & 0.02 & 264 & 102100 & 297.75 & 66.3 & 0.324\\
	5 & 546 & 23 & 35 & 23.21 & 0.04 & 321 & 102097 & 297.95 & 66.3 & 0.324\\
	6 & 546 & 25 & 35 & 24.86 & 0.05 & 371 & 102093 & 298.15 & 66.3 & 0.324\\
	7 & 546 & 27 & 35 & 27.02 & 0.03 & 434 & 102092 & 298.3 & 66.3 & 0.324\\
	8 & 546 & 29 & 25 & 29.12 & 0.04 & 505 & 102080 & 298.35 & 66.3 & 0.324\\
	9 & 546 & 31 & 25 & 30.86 & 0.06 & 564 & 102050 & 299.15 & 66.3 & 0.324\\
	10 & 546 & 33 & 25 & 32.98 & 0.05 & 641 & 102054 & 299.9 & 60.4 & 0.35\\
	\hline
\end{tabular}
\caption{Experimental measurements for station 1}
\label{table:station_1_measurements}
\end{center}
\end{table}
\renewcommand\baselinestretch{2}\selectfont

\renewcommand\baselinestretch{1.3}\selectfont
\begin{table}[H]
\begin{center}
\begin{tabular}{|ccccccccccc|}
	\hline
	Run & $I_Z$ & $V_{nom}$ & $dt$ & $V_{fs}$ & $\sigma_{V_{fs}}$ & $Q$ & $P_{atm}$ & $T_{tunnel}$ & $\phi$ & $\eta_P$\\
	$ID$ & $(in)$ & $(m/s)$ & $(\mu s)$ & $(m/s)$ & $(m/s)$ & $(Pa)$ & $(Pa)$ & $(\degree K)$ & $(\%)$ & $(\mu s)$\\
	\hline
	11 & 708 & 15 & 40 & 15.27 & 0.02 & 138 & 101185 & 296.05 & 69.8 & 0.312\\
	12 & 708 & 17 & 40 & 16.89 & 0.02 & 169 & 101218 & 296.55 & 69.8 & 0.312\\
	13 & 708 & 19 & 40 & 19.03 & 0.02 & 215 & 101219 & 296.55 & 69.8 & 0.312\\
	14 & 708 & 21 & 40 & 21.13 & 0.02 & 264 & 101186 & 296.85 & 66.3 & 0.329\\
	15 & 708 & 23 & 40 & 23.21 & 0.04 & 321 & 101150 & 297.85 & 66.8 & 0.329\\
	16 & 708 & 25 & 25 & 25.36 & 0.04 & 380 & 101120 & 297.45 & 71.7 & 0.301\\
	17 & 708 & 27 & 25 & 27.03 & 0.03 & 432 & 101120 & 297.75 & 70.1 & 0.306\\
	18 & 708 & 29 & 25 & 29.12 & 0.06 & 498 & 101106 & 298.55 & 73.3 & 0.297\\
	19 & 708 & 31 & 25 & 30.87 & 0.04 & 562 & 101109 & 298.95 & 73.3 & 0.297\\
	20 & 708 & 33 & 25 & 33.39 & 0.04 & 653 & 101101 & 299.65 & 73.3 & 0.297\\
	\hline
\end{tabular}
\caption{Experimental measurements for station 2}
\label{table:station_2_measurements}
\end{center}
\end{table}
\renewcommand\baselinestretch{2}\selectfont

\input{tables/station_3_measurements}
\renewcommand\baselinestretch{1.3}\selectfont
\begin{table}[H]
\begin{center}
\begin{tabular}{|ccccccccccc|}
	\hline
	Run & $I_Z$ & $V_{nom}$ & $dt$ & $V_{fs}$ & $\sigma_{V_{fs}}$ & $Q$ & $P_{atm}$ & $T_{tunnel}$ & $\phi$ & $\eta_P$\\
	$ID$ & $(in)$ & $(m/s)$ & $(\mu s)$ & $(m/s)$ & $(m/s)$ & $(Pa)$ & $(Pa)$ & $(\degree K)$ & $(\%)$ & $(\mu s)$\\
	\hline
	31 & 863 & 15 & 40 & 14.86 & 0.02 & 133 & 101865 & 295.75 & 63.8 & 0.354\\
	32 & 863 & 17 & 40 & 17.39 & 0.02 & 180 & 101855 & 295.95 & 63.8 & 0.354\\
	33 & 863 & 19 & 40 & 19.08 & 0.02 & 219 & 101847 & 296.1 & 63.8 & 0.354\\
	34 & 863 & 21 & 40 & 21.13 & 0.05 & 267 & 101845 & 296.15 & 63.8 & 0.354\\
	35 & 863 & 23 & 40 & 23.29 & 0.02 & 323 & 101844 & 296.45 & 63.8 & 0.354\\
	36 & 863 & 25 & 25 & 24.98 & 0.05 & 373 & 101840 & 296.65 & 65.6 & 0.344\\
	37 & 863 & 27 & 25 & 27.09 & 0.05 & 438 & 101843 & 297 & 65.6 & 0.344\\
	38 & 863 & 29 & 25 & 28.81 & 0.03 & 493 & 101848 & 297.55 & 65.6 & 0.344\\
	39 & 863 & 31 & 25 & 30.87 & 0.04 & 570 & 101842 & 298.15 & - & -\\
	40 & 863 & 33 & 25 & 33.44 & 0.06 & 661 & 101844 & 298.35 & - & -\\
	\hline
\end{tabular}
\caption{Experimental measurements for station 4}
\label{table:station_4_measurements}
\end{center}
\end{table}
\renewcommand\baselinestretch{2}\selectfont

\begin{table}[H]
\begin{center}
\begin{tabular}{|ccccccccccc|}
	\hline
	Run & $I_Z$ & $V_{nom}$ & $dt$ & $V_{fs}$ & $\sigma_{V_{fs}}$ & $Q$ & $P_{atm}$ & $T_{tunnel}$ & $\phi$ & $\eta_P$\\
	$ID$ & $(in)$ & $(m/s)$ & $(\mu s)$ & $(m/s)$ & $(m/s)$ & $(Pa)$ & $(Pa)$ & $(\degree K)$ & $(\%)$ & $(\mu s)$\\
	\hline
	41 & 914 & 15 & 40 & 14.88 & 0.02 & 132 & 101815 & 296.25 & 57.5 & 0.386\\
	42 & 914 & 17 & 40 & 17.24 & 0.03 & 180 & 101812 & 296.35 & 55.8 & 0.398\\
	43 & 914 & 19 & 40 & 19.08 & 0.03 & 217 & 101812 & 294.45 & 55.8 & 0.398\\
	44 & 914 & 21 & 40 & 21.18 & 0.03 & 267 & 101816 & 296.65 & 55.8 & 0.398\\
	45 & 914 & 23 & 40 & 23.24 & 0.03 & 323 & 101809 & 296.95 & 55.8 & 0.398\\
	46 & 914 & 25 & 25 & 24.9 & 0.03 & 371 & 101802 & 297.15 & 55.8 & 0.398\\
	47 & 914 & 27 & 25 & 27.08 & 0.04 & 435 & 101788 & 297.45 & 55.8 & 0.398\\
	48 & 914 & 29 & 25 & 29.19 & 0.03 & 506 & 101784 & 297.85 & 55.8 & 0.398\\
	49 & 914 & 31 & 25 & 31.28 & 0.05 & 584 & 101786 & 298.15 & 56.1 & 0.393\\
	50 & 914 & 33 & 25 & 33.05 & 0.05 & 645 & 101789 & 298.75 & 56.1 & 0.393\\
	\hline
\end{tabular}
\caption{Experimental conditions for tests at station 5}
\label{table:station_5_measurements}
\end{center}
\end{table}

\begin{table}[H]
\begin{center}
\begin{tabular}{|ccccccccccc|}
	\hline
	Run & $I_Z$ & $V_{nom}$ & $dt$ & $V_{fs}$ & $\sigma_{V_{fs}}$ & $Q$ & $P_{atm}$ & $T_{tunnel}$ & $\phi$ & $\eta_P$\\
	$ID$ & $(in)$ & $(m/s)$ & $(\mu s)$ & $(m/s)$ & $(m/s)$ & $(Pa)$ & $(Pa)$ & $(\degree K)$ & $(\%)$ & $(\mu s)$\\
	\hline
	51 & 965 & 15 & 40 & 15.31 & 0.02 & 140 & 102039 & 294.85 & 61.2 & 0.381\\
	52 & 965 & 17 & 40 & 17.36 & 0.03 & 182 & 102034 & 294.95 & 61.2 & 0.381\\
	53 & 965 & 19 & 40 & 19.08 & 0.03 & 219 & 102035 & 295.15 & 59.5 & 0.392\\
	54 & 965 & 21 & 40 & 21.23 & 0.03 & 270 & 102037 & 295.35 & 59.5 & 0.392\\
	55 & 965 & 23 & 40 & 23.33 & 0.03 & 327 & 102018 & 295.65 & 59.5 & 0.392\\
	56 & 965 & 25 & 25 & 24.97 & 0.03 & 375 & 102021 & 295.95 & 59.5 & 0.392\\
	57 & 965 & 27 & 25 & 27.05 & 0.06 & 437 & 102008 & 297.85 & 52.6 & 0.422\\
	58 & 965 & 29 & 25 & 29.17 & 0.04 & 505 & 102005 & 298.15 & 47.4 & 0.454\\
	59 & 965 & 31 & 25 & 30.9 & 0.06 & 571 & 102015 & 297.85 & 53.7 & 0.421\\
	60 & 965 & 33 & 25 & 33.04 & 0.06 & 653 & 102009 & 297.35 & 53.7 & 0.421\\
	\hline
\end{tabular}
\caption{Experimental measurements for station 6}
\label{table:station_6_measurements}
\end{center}
\end{table}

\renewcommand\baselinestretch{1.3}\selectfont
\begin{table}[H]
\begin{center}
\begin{tabular}{|ccccccccccc|}
	\hline
	Run & $I_Z$ & $V_{nom}$ & $dt$ & $V_{fs}$ & $\sigma_{V_{fs}}$ & $Q$ & $P_{atm}$ & $T_{tunnel}$ & $\phi$ & $\eta_P$\\
	$ID$ & $(in)$ & $(m/s)$ & $(\mu s)$ & $(m/s)$ & $(m/s)$ & $(Pa)$ & $(Pa)$ & $(\degree K)$ & $(\%)$ & $(\mu s)$\\
	\hline
	61 & 1016 & 15 & 40 & 15.31 & 0.03 & 140 & 101977 & 296.15 & 52 & 0.434\\
	62 & 1016 & 17 & 40 & 16.98 & 0.04 & 171 & 101969 & 296.25 & 52 & 0.434\\
	63 & 1016 & 19 & 40 & 19.1 & 0.03 & 218 & 101961 & 296.3 & 52 & 0.434\\
	64 & 1016 & 21 & 40 & 21.15 & 0.05 & 269 & 101950 & 296.45 & 49.4 & 0.45\\
	65 & 1016 & 23 & 25 & 23.23 & 0.05 & 321 & 101953 & 296.77 & 52.6 & 0.422\\
	66 & 1016 & 25 & 25 & 24.96 & 0.05 & 371 & 101951 & 297.07 & 52.6 & 0.422\\
	67 & 1016 & 27 & 25 & 27.05 & 0.5 & 436 & 101936 & 297.25 & 52.6 & 0.422\\
	68 & 1016 & 29 & 25 & 29.17 & 0.03 & 505 & 101928 & 297.76 & 52.6 & 0.422\\
	69 & 1016 & 31 & 25 & 31.31 & 0.05 & 581 & 101921 & 297.85 & 52.6 & 0.422\\
	70 & 1016 & 33 & 25 & 33.01 & 0.05 & 647 & 101922 & 298.3 & 52.6 & 0.422\\
	\hline
\end{tabular}
\caption{Experimental measurements for station 7}
\label{table:station_7_measurements}
\end{center}
\end{table}
\renewcommand\baselinestretch{2}\selectfont

