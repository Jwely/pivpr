
\section{Discussion}

Axial vortices were created over a range of velocities, and interrogated at 
seven different stations ranging between 546$mm$ and 1016$mm$ downstream, or 
5.4 and 10.0 chord lengths respectively. For the 
purposes of results and discussion, it is desirable to reduce the set to a few 
exemplary data sets that illustrate observed phenomena. The vortex generator 
used in these experiments had a center body with a 1 inch diameter, and thus 
produced two co-rotating vortices that merge together as they moved downstream. 
The first three stations display characteristics that resemble co-rotating 
vortices that had not yet finished merging. Furthermore, analysis shows that 
the uncertainty in measurements decreases significantly as the interrogation 
plane moves down stream. Therefore, results from only a few vortices at 
downstream locations of are discussed here. Full results for all vortices 
generated can be obtained from an online public repository, the URL for which 
can be found in the notes. Turbulent quantities are compared to hot-wire 
anemometry experiments conducted by Thompson \cite{thompson2016}.


\subsection{Vortex Stabilization}
A bi wing vortex generator actually creates a set of co-rotating vortices from 
the intersections of the wing and center body or the wing edges. A single axial 
vortex stabilizes as the co rotating vortices merge together and as the 
wake of the center body diminishes. The rotational motion of the vortex 
stretches the axial velocity deficit around the core region. This was observed 
in the PIV data in Figure \ref{fig:run_1_W_contour} showing a far upstream 
average axial velocity distribution with very clear wing wakes coming off 
center, and in Figure \ref{fig:run_55_W_contour} showing a downstream station 
at a higher velocity with wake remnants wound about the core. In these figures, 
positive values indicate fluid flowing away from the core.

\begin{figure}[H]
\centering
\includegraphics[width=4.25in]{figs/run_1/run_1_W_contour}
\caption{Contour plot of $\overline{w}$ at $z/c$=5.37, $V_{free}$=15.22, station 1.}
\label{fig:run_1_W_contour}
\end{figure}



\begin{figure}[H]
\centering
\includegraphics[width=4.25in]{figs/run_55/run_55_W_contour}
\caption{Contour plot of $\overline{w}$ at $z/c$=9.50, $V_{free}$=23.33, station 6.}
\label{fig:run_55_W_contour}
\end{figure}




Each wing generated a lifting force which converted axial momentum into radial 
and tangential momentum about the vortex core. As the vortex developed, radial 
movements in the fluid persisted with similar magnitude, though were broken up 
into smaller localized jets as shown in Figures \ref{fig:run_1_R_contour} and 
\ref{fig:run_51_R_contour}, which were taken at similar free stream velocities 
but several chord lengths apart. For the majority of tests, the radial velocity 
patterns from the wings remain intact, with radial outflow in the top right and 
bottom left quadrants. 

\begin{figure}[H]
\centering
\includegraphics[width=4.25in]{figs/run_1/run_1_R_contour}
\caption{Contour plot of $\overline{v_{r}}$ at $z/c$=5.37, $V_{free}$=15.22, station 1.}
\label{fig:run_1_R_contour}
\end{figure}



\begin{figure}[H]
\centering
\includegraphics[width=4.25in]{figs/run_51/run_51_R_contour}
\caption{Contour plot of $\overline{v_{r}}$ at $z/c$=9.50, $V_{free}$=15.31, station 6.}
\label{fig:run_51_R_contour}
\end{figure}




Under the conditions of early vortex formation, stronger asymmetry was seen in 
the azimuthal velocity component, $\bar{t}$ than in more developed vortices as 
shown in Figures \ref{fig:run_1_T_contour} and \ref{fig:run_55_T_contour}. When 
this data is flattened into a scatter plot for every grid point making up the 
two dimensional dataset, the asymmetries become particularly apparent as shown 
in Figure \ref{fig:run_1_T_vs_r_mesh_scatter}.

\begin{figure}[H]
\centering
\includegraphics[width=4.25in]{figs/run_1/run_1_T_contour}
\caption{Contour plot of $\overline{v_{\theta}}$ at $z/c$=5.37, $V_{free}$=15.22, station 1.}
\label{fig:run_1_T_contour}
\end{figure}



\begin{figure}[H]
\centering
\includegraphics[width=4.25in]{figs/run_55/run_55_T_contour}
\caption{Contour plot of $\overline{v_{\theta}}$ at $z/c$=9.50, $V_{free}$=23.33, station 6.}
\label{fig:run_55_T_contour}
\end{figure}



\begin{figure}[H]
\centering
\includegraphics[width=6in]{figs/run_1/run_1_T_vs_r_mesh_scatter}
\caption{Scatter plot of azimuthal velocity vs radius at $z/c$=5.37, $V_{free}$=15.22, station1}
\label{fig:run_1_T_vs_r_mesh_scatter}
\end{figure}




Vortices are characterized principally by the maximum azimuthal velocity 
and the radius of the core. The absolute maximum velocity from the scatter plot 
is likely to be impacted by random noise in the data, and may 
not occur near the true core boundary. Therefore, the vortices were 
characterized by taking a moving average of the azimuthal velocity plot, and 
taking the maximum value and radial point at which it occurs. Thus, more 
developed vortices with greater radial symmetry were better suited for further 
study. With a smoothed azimuthal velocity profile, the experimental data were 
compared to the theoretical profiles of common vortices introduced previously, 
as in figure \ref{fig:run_55_comparison}. In this comparison, the Lamb-Oseen 
and Ash vortices are scaled to match the experimental maximum azimuthal 
velocity, while the Rankine vortex is scaled to match the core radius.

\begin{figure}[H]
\centering
\includegraphics[width=6in]{figs/run_55/run_55_comparison}
\caption{Theoretical vortex profile fits to experimental data at $z/c$=9.50, $V_{free}$=23.33, station 6.}
\label{fig:run_55_comparison}
\end{figure}




A summary of these characteristics can be found in Tables 
\ref{table:experiment_results_1} through \ref{table:experiment_results_7}. 

\begin{table}[H]
\begin{center}
\begin{tabular}{|cccccccccccc|}
	\hline
	Run & $I_Z$ & $V_{nom}$ & $dt$ & $V_{free}$ & $Q$ & $P_{atm}$ & $T_{tunnel}$ & $\phi$ & $R_{core}$ & $\overline{v_{\theta}}_{max}$ & $\overline{v_{z}}_{mean}$\\
	  & $mm$ & $m/s$ & $\mu s$ & $m/s$ & $Pa$ & $Pa$ & K & $\%$ & $mm$ & $m/s$ & $m/s$\\
	\hline
	1 & 546 & 15.0 & 50 & 15.2 & 135 & 102036 & 299.9 & 60.4 & 17.4 & 3.1 & 15.3\\
	2 & 546 & 17.0 & 50 & 16.9 & 170 & 102115 & 297.6 & 66.3 & 16.4 & 3.2 & 17.0\\
	3 & 546 & 19.0 & 50 & 19.4 & 225 & 102105 & 297.6 & 66.3 & 18.9 & 3.7 & 19.6\\
	4 & 546 & 21.0 & 50 & 21.1 & 264 & 102100 & 297.8 & 63.3 & 18.2 & 4.0 & 21.3\\
	5 & 546 & 23.0 & 35 & 23.2 & 321 & 102097 & 297.9 & 63.3 & 17.5 & 4.3 & 23.2\\
	6 & 546 & 25.0 & 35 & 24.9 & 371 & 102093 & 298.1 & 63.3 & 17.3 & 4.7 & 25.2\\
	7 & 546 & 27.0 & 35 & 27.0 & 434 & 102092 & 298.3 & 63.3 & 19.1 & 5.0 & 27.3\\
	8 & 546 & 29.0 & 25 & 29.1 & 505 & 102080 & 298.4 & 63.3 & 18.6 & 5.4 & 29.3\\
	9 & 546 & 31.0 & 25 & 30.9 & 564 & 102050 & 299.1 & 63.3 & 18.5 & 5.8 & 31.2\\
	10 & 546 & 33.0 & 25 & 33.0 & 641 & 102054 & 299.9 & 60.4 & 16.5 & 6.2 & 33.2\\
	\hline
\end{tabular}
\caption{Summary of experimental measurements at station 1}
\label{table:experiment_results_1}
\end{center}
\end{table}

\begin{table}[H]
\begin{center}
\begin{tabular}{|cccccccccccc|}
	\hline
	Run & $I_Z$ & $V_{nom}$ & $dt$ & $V_{free}$ & $P_{atm}$ & $T_{tunnel}$ & $\phi$ & $\eta_P$ & $R_{core}$ & $\overline{v_{\theta}}_{max}$ & $\overline{v_{\bar{z}}}$\\
	  & $mm$ & $m/s$ & $\mu s$ & $m/s$ & $KPa$ & K & $\%$ & $\mu s$ & $mm$ & $m/s$ & $m/s$\\
	\hline
	11 & 708 & 15 & 40 & 15.3 & 101.2 & 296.1 & 69.8 & 0.31 & 20.0 & 2.9 & 15.4\\
	12 & 708 & 17 & 40 & 16.9 & 101.2 & 296.6 & 69.8 & 0.31 & 19.4 & 3.6 & 17.1\\
	13 & 708 & 19 & 40 & 19.0 & 101.2 & 296.6 & 69.8 & 0.31 & 18.7 & 3.7 & 19.3\\
	14 & 708 & 21 & 40 & 21.1 & 101.2 & 296.9 & 66.4 & 0.33 & 19.0 & 4.0 & 21.3\\
	15 & 708 & 23 & 40 & 23.2 & 101.2 & 297.9 & 66.4 & 0.33 & 19.7 & 2.4 & 14.5\\
	16 & 708 & 25 & 25 & 25.4 & 101.1 & 297.4 & 71.8 & 0.30 & 19.5 & 5.1 & 25.7\\
	17 & 708 & 27 & 25 & 27.0 & 101.1 & 297.8 & 70.2 & 0.31 & 18.3 & 5.2 & 27.3\\
	18 & 708 & 29 & 25 & 29.1 & 101.1 & 298.6 & 73.4 & 0.30 & 19.0 & 5.6 & 29.5\\
	19 & 708 & 31 & 25 & 30.9 & 101.1 & 298.9 & 73.4 & 0.30 & 18.6 & 5.4 & 31.3\\
	20 & 708 & 33 & 25 & 33.4 & 101.1 & 299.6 & 73.4 & 0.30 & 19.4 & 6.2 & 33.7\\
	\hline
\end{tabular}
\caption{Summary of experimental measurements at station 2}
\label{table:experiment_results_2}
\end{center}
\end{table}

\begin{table}[H]
\begin{center}
\begin{tabular}{|cccccccccccc|}
	\hline
	Run & $I_Z$ & $V_{nom}$ & $dt$ & $V_{free}$ & $P_{atm}$ & $T_{tunnel}$ & $\phi$ & $\eta_P$ & $R_{core}$ & $\overline{v_{\theta}}_{max}$ & $\overline{v_{\bar{z}}}$\\
	  & $mm$ & $m/s$ & $\mu s$ & $m/s$ & $KPa$ & K & $\%$ & $\mu s$ & $mm$ & $m/s$ & $m/s$\\
	\hline
	21 & 787 & 15 & 40 & 15.2 & 101.1 & 297.9 & 72.0 & 0.29 & 20.8 & 3.0 & 15.4\\
	22 & 787 & 17 & 40 & 17.3 & 101.1 & 297.9 & 72.0 & 0.29 & 18.6 & 3.3 & 17.7\\
	23 & 787 & 19 & 40 & 19.4 & 101.1 & 297.9 & 70.2 & 0.30 & 19.4 & 3.6 & 19.6\\
	24 & 787 & 21 & 40 & 21.1 & 101.1 & 297.9 & 75.3 & 0.29 & 19.0 & 3.9 & 21.3\\
	25 & 787 & 23 & 40 & 23.1 & 101.1 & 298.1 & 75.3 & 0.29 & 18.6 & 4.2 & 23.3\\
	26 & 787 & 25 & 25 & 24.9 & 101.1 & 298.4 & 71.9 & 0.30 & 19.7 & 4.9 & 25.1\\
	27 & 787 & 27 & 25 & 26.9 & 101.1 & 298.6 & 70.2 & 0.30 & 17.7 & 5.1 & 27.5\\
	28 & 787 & 29 & 25 & 29.1 & 101.1 & 299.1 & 71.9 & 0.30 & 18.7 & 5.3 & 29.3\\
	29 & 787 & 31 & 25 & 31.3 & 101.1 & 299.4 & 71.9 & 0.30 & 18.8 & 5.6 & 31.5\\
	30 & 787 & 33 & 25 & 32.9 & 101.0 & 299.8 & 77.0 & 0.28 & 21.2 & 5.9 & 33.3\\
	\hline
\end{tabular}
\caption{Summary of experimental measurements at station 3}
\label{table:experiment_results_3}
\end{center}
\end{table}

\begin{table}[H]
\begin{center}
\begin{tabular}{|cccccccccccc|}
	\hline
	Run & $I_Z$ & $V_{nom}$ & $dt$ & $V_{fs}$ & $Q$ & $P_{atm}$ & $T_{tunnel}$ & $\phi$ & $R_{core}$ & $\overline{t}_{max}$ & $\overline{w}_{mean}$\\
	  & $mm$ & $m/s$ & $\mu s$ & $m/s$ & $Pa$ & $Pa$ & K & $\%$ & $mm$ & $m/s$ & $m/s$\\
	\hline
	31 & 864 & 15.0 & 40 & 14.9 & 133 & 101865 & 295.8 & 63.8 & 19.1 & 2.9 & 15.2\\
	32 & 864 & 17.0 & 40 & 17.4 & 180 & 101855 & 295.9 & 63.8 & 19.2 & 3.2 & 17.5\\
	33 & 864 & 19.0 & 40 & 19.1 & 219 & 101847 & 296.1 & 63.8 & 19.1 & 3.6 & 19.4\\
	34 & 864 & 21.0 & 40 & 21.1 & 267 & 101845 & 296.1 & 63.8 & 18.5 & 3.9 & 21.5\\
	35 & 864 & 23.0 & 40 & 23.3 & 323 & 101844 & 296.4 & 63.8 & 18.5 & 4.3 & 23.7\\
	36 & 864 & 25.0 & 25 & 25.0 & 373 & 101840 & 296.6 & 65.6 & 19.3 & 4.7 & 25.5\\
	37 & 864 & 27.0 & 25 & 27.1 & 438 & 101843 & 297.0 & 65.6 & 17.4 & 4.8 & 27.3\\
	38 & 864 & 29.0 & 25 & 28.8 & 493 & 101848 & 297.6 & 65.6 & 19.9 & 5.0 & 29.0\\
	39 & 864 & 31.0 & 25 & 30.9 & 570 & 101842 & 298.1 & 62.5 & 32.3 & 4.5 & 31.0\\
	40 & 864 & 33.0 & 25 & 33.4 & 661 & 101844 & 298.4 & 62.5 & 16.7 & 4.3 & 32.3\\
	\hline
\end{tabular}
\caption{Summary of experimental measurements at station 4}
\label{table:experiment_results_4}
\end{center}
\end{table}

\begin{table}[H]
\begin{center}
\begin{tabular}{|cccccccccccc|}
	\hline
	Run & $I_Z$ & $V_{nom}$ & $dt$ & $V_{fs}$ & $Q$ & $P_{atm}$ & $T_{tunnel}$ & $\phi$ & $R_{core}$ & $\overline{t}_{max}$ & $\overline{w}_{mean}$\\
	  & $mm$ & $m/s$ & $\mu s$ & $m/s$ & $Pa$ & $Pa$ & K & $\%$ & $mm$ & $m/s$ & $m/s$\\
	\hline
	41 & 914 & 15.0 & 40 & 14.9 & 132 & 101815 & 296.2 & 57.5 & 14.7 & 3.7 & 15.2\\
	42 & 914 & 17.0 & 40 & 17.2 & 180 & 101812 & 296.4 & 55.8 & 15.1 & 4.4 & 17.3\\
	43 & 914 & 19.0 & 40 & 19.1 & 217 & 101812 & 294.4 & 55.8 & 13.5 & 4.6 & 19.4\\
	44 & 914 & 21.0 & 40 & 21.2 & 267 & 101816 & 296.6 & 55.8 & 16.5 & 5.2 & 21.6\\
	45 & 914 & 23.0 & 40 & 23.2 & 323 & 101809 & 296.9 & 55.8 & 14.0 & 5.7 & 23.7\\
	46 & 914 & 25.0 & 25 & 24.9 & 371 & 101802 & 297.1 & 55.8 & 14.8 & 6.2 & 25.4\\
	47 & 914 & 27.0 & 25 & 27.1 & 435 & 101788 & 297.4 & 55.8 & 14.7 & 6.9 & 27.5\\
	48 & 914 & 29.0 & 25 & 29.2 & 506 & 101784 & 297.9 & 55.8 & 13.4 & 7.3 & 29.7\\
	49 & 914 & 31.0 & 25 & 31.3 & 584 & 101786 & 298.1 & 56.1 & 13.6 & 7.5 & 31.8\\
	50 & 914 & 33.0 & 25 & 33.0 & 645 & 101789 & 298.8 & 56.1 & 16.2 & 7.8 & 33.5\\
	\hline
\end{tabular}
\caption{Summary of experimental measurements at station 5}
\label{table:experiment_results_5}
\end{center}
\end{table}

\begin{table}[H]
\begin{center}
\begin{tabular}{|cccccccccccc|}
	\hline
	Run & $I_Z$ & $V_{nom}$ & $dt$ & $V_{fs}$ & $Q$ & $P_{atm}$ & $T_{tunnel}$ & $\phi$ & $R_{core}$ & $\overline{t}_{max}$ & $\overline{w}_{mean}$\\
	  & $mm$ & $m/s$ & $\mu s$ & $m/s$ & $Pa$ & $Pa$ & K & $\%$ & $mm$ & $m/s$ & $m/s$\\
	\hline
	51 & 965 & 15.0 & 40 & 15.3 & 140 & 102039 & 294.9 & 61.2 & 16.5 & 3.7 & 15.5\\
	52 & 965 & 17.0 & 40 & 17.4 & 182 & 102034 & 294.9 & 61.2 & 13.8 & 4.2 & 17.8\\
	53 & 965 & 19.0 & 40 & 19.1 & 219 & 102035 & 295.1 & 59.4 & 14.9 & 4.4 & 19.4\\
	54 & 965 & 21.0 & 40 & 21.2 & 270 & 102037 & 295.4 & 59.4 & 16.0 & 5.1 & 21.7\\
	55 & 965 & 23.0 & 40 & 23.3 & 327 & 102018 & 295.6 & 59.4 & 14.5 & 5.6 & 23.7\\
	56 & 965 & 25.0 & 25 & 25.0 & 375 & 102021 & 295.9 & 59.4 & 16.5 & 6.4 & 25.3\\
	57 & 965 & 27.0 & 25 & 27.1 & 437 & 102008 & 297.9 & 52.6 & 13.9 & 6.7 & 27.6\\
	58 & 965 & 29.0 & 25 & 29.2 & 505 & 102005 & 298.1 & 47.4 & 14.2 & 6.8 & 29.6\\
	59 & 965 & 31.0 & 25 & 30.9 & 571 & 102015 & 297.9 & 53.7 & 17.4 & 7.0 & 31.4\\
	60 & 965 & 33.0 & 25 & 33.0 & 653 & 102009 & 297.4 & 53.7 & 17.7 & 7.9 & 33.7\\
	\hline
\end{tabular}
\caption{Summary of experimental measurements at station 6}
\label{table:experiment_results_6}
\end{center}
\end{table}

\begin{table}[H]
\begin{center}
\begin{tabular}{|cccccccccccc|}
	\hline
	Run & $I_Z$ & $V_{nom}$ & $dt$ & $V_{fs}$ & $Q$ & $P_{atm}$ & $T_{tunnel}$ & $\phi$ & $R_{core}$ & $\overline{t}_{max}$ & $\overline{w}_{core}$\\
	  & $mm$ & $m/s$ & $\mu s$ & $m/s$ & $Pa$ & $Pa$ & K & $\%$ & $mm$ & $m/s$ & $m/s$\\
	\hline
	61 & 1016 & 15.0 & 40 & 15.3 & 140 & 101977 & 296.1 & 52.0 & 15.8 & 3.7 & 12.3\\
	62 & 1016 & 17.0 & 40 & 17.0 & 171 & 101969 & 296.2 & 52.0 & 14.2 & 4.0 & 13.8\\
	63 & 1016 & 19.0 & 40 & 19.1 & 218 & 101961 & 296.3 & 52.0 & 16.1 & 4.5 & 15.1\\
	64 & 1016 & 21.0 & 40 & 21.1 & 269 & 101950 & 296.4 & 49.4 & 15.6 & 5.0 & 17.1\\
	65 & 1016 & 23.0 & 25 & 23.2 & 321 & 101953 & 296.8 & 52.6 & 15.1 & 5.3 & 19.1\\
	66 & 1016 & 25.0 & 25 & 25.0 & 371 & 101951 & 297.1 & 52.6 & 14.6 & 6.0 & 20.0\\
	67 & 1016 & 27.0 & 25 & 27.1 & 436 & 101936 & 297.2 & 52.6 & 14.4 & 6.4 & 22.0\\
	68 & 1016 & 29.0 & 25 & 29.2 & 505 & 101928 & 297.8 & 52.6 & 16.3 & 6.7 & 23.4\\
	69 & 1016 & 31.0 & 25 & 31.3 & 581 & 101921 & 297.9 & 52.6 & 16.9 & 7.3 & 25.3\\
	70 & 1016 & 33.0 & 25 & 33.0 & 647 & 101922 & 298.3 & 52.6 & 15.1 & 7.8 & 26.6\\
	\hline
\end{tabular}
\caption{Summary of experimental measurements at station 7}
\label{table:experiment_results_7}
\end{center}
\end{table}


\subsection{Macro-Turbulent Structures}
The vortex core sizes at all velocities were observed to drop from 
17-20$mm$ in radius to 13-16$mm$ in radius between station 4 at 8.5 chord 
lengths and station 5 at 9 chord lengths. The abrupt contraction of the core 
may be attributable to the instability created by the wake of the vortex 
generator center body. As the core diameter decreased, the azimuthal velocities 
were also observed to increase sharply as the fast moving fluid accelerates 
inward to tighten the core boundary. Figure \ref{fig:run_35_R_contour} shows 
nearly opposed and well defined regions of radial velocity into the core (-) 
and out of the core (+). Figure \ref{fig:run_45_R_contour} shows the dispersion 
of these radial velocity jets just one half chord length further downstream, at 
the same free stream velocity. 

\begin{figure}[H]
\centering
\includegraphics[width=4.25in]{figs/run_35/run_35_R_contour}
\caption{Contour plot of $\overline{r}$ at $z/c$=8.50, $V_{free}$=23.29, station 4.}
\label{fig:run_35_R_contour}
\end{figure}



\begin{figure}[H]
\centering
\includegraphics[width=4.25in]{figs/run_45/run_45_R_contour}
\caption{Contour plot of $\overline{v_{r}}$ at $z/c$=9.00, $V_{free}$=23.24, station 5.}
\label{fig:run_45_R_contour}
\end{figure}




At this transition stage, an abrupt change in 
the distribution turbulent kinetic energy was also observed. Figure 
\ref{fig:run_35_ctke_contour} at station 4 shows turbulence spread widely 
across a region approximately twice the diameter of the core, with several 
areas showing turbulent kinetic energy intensities of 95\% of the peak value in 
the core. Figure \ref{fig:run_45_ctke_contour} at station 5 shows a drastic 
weakening of these zones of high intensity turbulence just outside the core 
region, while also indicating a 30\% increase in turbulent intensity within the 
core. An appearance of fresh turbulent bands radially spoking outwards was also 
observed. Perhaps even more interestingly, small areas of high turbulent 
intensity were found to arise in the same region as before, between 1 and 2 
core radii, just another half chord length down stream as shown in Figure 
\ref{fig:run_55_ctke_contour} at station 6. In this figure, a drop 
in the maximum turbulence intensity was also observed in the center. Despite 
the new strengthening of turbulent structures outside the core region, the core 
radius was not observed to increase between stations 5 and 6. These findings 
are consistent with the vortex gulping phenomenon described by 
\cite{bandyopadhyay1991}. Furthermore, this suggests the ingestion of 
turbulence may be responsible for tightening the core, contributing to the 
longevity of axial wake vortices observed in nature.

\begin{figure}[H]
\centering
\includegraphics[width=4.25in]{figs/run_35/run_35_ctke_contour}
\caption{Contour plot of $k$ at $z/c$=8.50, $V_{free}$=23.29, station 4.}
\label{fig:run_35_ctke_contour}
\end{figure}



\begin{figure}[H]
\centering
\includegraphics[width=4.25in]{figs/run_45/run_45_ctke_contour}
\caption{Contour plot of $k$ at $z/c$=9.00, $V_{free}$=23.24, station 5.}
\label{fig:run_45_ctke_contour}
\end{figure}



\begin{figure}[H]
\centering
\includegraphics[width=4.25in]{figs/run_55/run_55_ctke_contour}
\caption{Contour plot of $k$ at $z/c$=9.50, $V_{free}$=23.33, station 6.}
\label{fig:run_55_ctke_contour}
\end{figure}




The previous discussion is based upon data derived from time averages of many 
snapshots spaced one second apart. To gain better understanding of the time and 
length scales at work in the turbulent regions of the flow, particularly inside 
the core and in the highly turbulent zones just outside the core, the unstable 
components were examined over time. The time series is comprised of just 200 
seconds of samples at 1Hz, enough to resolve motions at frequencies between 
about 0.01 Hz and 0.5 Hz. Understanding of the time and length scales of 
turbulent fluctuations within the core, and how they relate to those seen 
outside of the core is of great interest. Evolution of turbulent scales as the 
interrogation plane moves downstream is of further interest. 

Figures \ref{fig:run_35_ctke_01rdynamic} through 
\ref{fig:run_55_ctke_01rdynamic} show power spectral densities of 
average turbulent kinetic energies within the core boundary for a vortex at the 
same three conditions as figures \ref{fig:run_35_ctke_contour} through 
\ref{fig:run_55_ctke_contour}. Each of these figures contains three plots. The 
left most plot shows the time averaged kinetic energy distribution within the 
sample area. The center plot shows the variation in kinetic energy over time 
with both the mean value of all points within the sample area shown, 
and 5th and 95th percentile boundaries to show the spread. The plot at right 
shows the log of the power spectral density (PSD) of the mean line shown in the 
center plot. 

\begin{figure}[H]
\centering
\includegraphics[width=6in]{figs/run_35/run_35_ctke_01rdynamic}
\caption{Plot showing dynamic variations in turbulent kinetic energy within the core. $z/c$=8.50, $V_{free}$=23.29, station 4.}
\label{fig:run_35_ctke_01rdynamic}
\end{figure}



\begin{figure}[H]
\centering
\includegraphics[width=6in]{figs/run_45/run_45_ctke_01rdynamic}
\caption{Plot showing dynamic variations in turbulent kinetic energy within the core. $z/c$=9.00, $V_{free}$=23.24, station 5.}
\label{fig:run_45_ctke_01rdynamic}
\end{figure}



\begin{figure}[H]
\centering
\includegraphics[width=6in]{figs/run_55/run_55_ctke_01rdynamic}
\caption{Plot showing dynamic variations in turbulent kinetic energy within the core. $z/c$=9.50, $V_{free}$=23.33, station 6.}
\label{fig:run_55_ctke_01rdynamic}
\end{figure}




With the region inside the core boundary highlighted, the increasing
concentration of turbulence between stations 4 and 5 was seen more clearly. The 
power spectral density profile of the core region however did not appear to 
change significantly, with most frequencies remaining between -5 and -10 
$db/Hz$ at all three chord lengths. 

\begin{figure}[H]
\centering
\includegraphics[width=6in]{figs/run_35/run_35_ctke_12rdynamic}
\caption{Plot showing dynamic variations in turbulent kinetic energy between 1 and 2 core radii. $z/c$=8.50, $V_{free}$=23.29, station 4.}
\label{fig:run_35_ctke_12rdynamic}
\end{figure}



\begin{figure}[H]
\centering
\includegraphics[width=6in]{figs/run_45/run_45_ctke_12rdynamic}
\caption{Plot showing dynamic variations in turbulent kinetic energy between 1 and 2 core radii. $z/c$=9.00, $V_{free}$=23.24, station 5.}
\label{fig:run_45_ctke_12rdynamic}
\end{figure}



\begin{figure}[H]
\centering
\includegraphics[width=6in]{figs/run_55/run_55_ctke_12rdynamic}
\caption{Plot showing dynamic variations in turbulent kinetic energy between 1 and 2 core radii. $z/c$=9.50, $V_{free}$=23.33, station 6.}
\label{fig:run_55_ctke_12rdynamic}
\end{figure}




The range over which some insight can be gained was 
expanded slightly with the variation of free stream velocity between 15 and 33 
$m/s$.

(flag, unfinished, more turbulent plots varying by velocity)

The PIV measurements have an uncertainty associated with the stable components 
which benefits from an increased number of measurements. However, the 
uncertainty associated with each individual measurement (N=1) will have a 
significant impact on power spectral density analysis of the fluctuating 
values. An uncertainty analysis has been conducted and indicates that the 
precision of PIV measurements is sufficient to study time averaged quantities 
including Reynolds stresses and quantities related to the root-mean-square of 
velocity fluctuations, but insufficient to study the dynamic nature of 
turbulent structures. The topic is discussed in greater detail in section 
\ref{sec:piv_uncert}.

\section{Reynolds stress and turbulence relationships}

flag