Vortices are common natural fluid phenomena which are extensively scrutinized 
in the study of turbulent flows. While small vortical structures appear and 
disappear without notice, some circumstances promote the formation of strong 
axial vortices that endure. Those types of axial vortices are found trailing 
the wake of aircraft wingtips, and in nature as dust devils, tornadoes, and 
waterspouts. Wingtip vortices behind large aircraft create unsafe conditions 
for other aircraft, especially around airport runways. The presence of axial 
wake vortices is a significant limiting factor on the volume of traffic an 
airport can accommodate \cite{hallock1991}. While the circumstances that result 
in the creation of strong axial vortices are well understood, the mechanisms 
that allow them to stay tightly wound for as long as they do remain mysterious. 
Due to the spontaneous nature of tornadoes, good experimental data is difficult 
to acquire and repeat. Axial wake vortices behave differently for 
small aircraft or in wind tunnels at low Reynolds number than they do for large 
aircraft or at high Reynolds numbers \cite{burnam2013}. Multiple empirical and 
analytical solutions exist to describe the velocity profiles of persistent 
axial vortices, each with varying levels of fidelity.

In this analysis, A bi-wing vortex generator was constructed, and placed in 
a low speed wind tunnel. Stereo particle image velocimetry was employed to 
map three dimensional velocity vectors in planes perpendicular to the free 
stream flow direction. A Reynolds time 
averaging approach was used to represent this velocity data in terms of mean 
and fluctuating components. Free stream velocity and downstream distance 
were varied to create axial vortices over a range of characteristic Reynolds 
numbers, thereby producing a three dimensional perspective. Observed velocity 
profiles were compared with predictions based on multiple common vortex models. 
Relationships between 
Reynolds stresses, turbulent energy, strain rates, and dissipation were
examined to gain understanding of the compact core behavior, and to help 
explain the models performance.