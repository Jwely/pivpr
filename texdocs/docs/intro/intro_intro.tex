Vortices are common natural fluid phenomena which are extensively scrutinized 
in the study of turbulent flows. While small vortical structures appear and 
disappear without notice, some circumstances promote the formation of strong 
axial vortices that endure. Those types of axial vortices are found in the 
trailing wake of aircraft, and in nature as dust devils, tornadoes, and 
waterspouts. Wingtip vortices behind large aircraft create unsafe conditions 
for other aircraft, especially around airport runways. The presence of 
persistent axial wake vortices is a significant limiting factor controlling the 
volume of traffic that a busy airport can accommodate \cite{hallock1991}. While 
the circumstances that result 
in the creation of strong axial vortices are well understood, the mechanisms 
that allow them to remain tightly wound with slowly decaying azimuthal 
velocities for as long as they do remain mysterious. 
Due to the spontaneous nature of tornadoes, good experimental data is difficult 
to acquire and repeat. Axial wake vortices behave differently for 
small aircraft or in wind tunnels at low Reynolds numbers than they do for 
large aircraft or at high Reynolds numbers \cite{burnam2013}. Multiple 
empirical and analytical models exist to describe the velocity profiles of 
persistent axial vortices, each with varying levels of fidelity.

In this analysis, a bi-wing vortex generator was constructed, and mounted in 
a low speed wind tunnel. A mineral oil fog was added to the flow and stereo 
particle image velocimetry was employed to map three dimensional velocity 
vectors in survey planes perpendicular to the free stream flow direction. A 
Reynolds time-averaging approach was employed to represent this velocity data 
in terms of mean and fluctuating components. Free stream velocity and 
downstream distance were varied to create axial vortices over a range of 
characteristic Reynolds numbers, thereby producing a three dimensional 
perspective. Observed velocity profiles were compared with predictions based on 
multiple common vortex models. Relationships between 
Reynolds stresses, turbulent energy, strain rates, and dissipation were
examined to gain understanding of the compact core behavior, and to help 
explain the surprising performance of a simple eddy-viscosity turbulence model.