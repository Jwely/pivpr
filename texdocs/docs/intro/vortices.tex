\section{Established Vortex models}
Several vortex models are already commonly used to describe the behavior of 
axial wake vortices. Among these models are, in order of introduction, the 
Rankine, Lamb-Oseen, Burnham-Hallock, and Proctor models \cite{ahmad2014}. The 
simplest and oldest model, the Rankine vortex, approximates the azimuthal 
velocity of the vortex within the core zone as a rigid body of rotating fluid, 
with velocity trailing off according to the inverse of radius at the core 
boundary \cite{rankine1869}. The azimuthal velocity profile of a Rankine vortex 
is given by Equations \ref{eq:rankine1} and \ref{eq:rankine2}.

\begin{equation}
v_{\theta}(r) = \frac{\Gamma_0}{2 \pi r_{core}} \frac{r}{r_{core}} 
	\text{ for } r \leq r_{core}
	\label{eq:rankine1}
\end{equation}

\begin{equation}
v_{\theta}(r) = \frac{\Gamma_0}{2 \pi r}
	\label{eq:rankine2}
\end{equation}

\noindent
Where $\Gamma_0$ is the circulation and $r_{core}$ is the radius of the core, 
which is defined as the location of the maximum azimuthal velocity, 
$v_{\theta, max}$. The azimuthal velocity $v_{\theta}$, is proportional to 
radius $r$, in the core zone, but proportional to $1/r$ in the area outside the 
core zone. This discontinuity makes the model unsuitable in the vicinity of the 
core region of the vortex. The model is unsuitable for the study of unsteady 
flow or turbulence, and the rigid body assumption at the core has been 
experimentally shown to be invalid.

Significantly later, the Lamb-Oseen vortex model was developed from an 
analytical solution to the Navier-Stokes equations. In this model, a potential 
vortex with an infinite velocity at the centerline was subject to viscous 
decay. The solution produces a continuous, unsteady function of the azimuthal 
velocity profile as shown in Equations \ref{eq:lamb1} and \ref{eq:lamb2} 
\cite{lamb1932}.

\begin{equation}
v_{\theta}(r,t) = \frac{\Gamma_0}{2 \pi r_{core}}[1 - 
							exp(-(r / r_{core})^2)]
	\label{eq:lamb1}
\end{equation}

\noindent
Where the core radius dilates according to 

\begin{equation}
r_{core}(t) = \sqrt{4 \nu t}
	\label{eq:lamb2}
\end{equation}

\noindent
Where $\nu$ is the dynamic viscosity. Alternatively, the azimuthal velocity 
profile can be expressed in terms of the maximum azimuthal velocity, 
$v_{\theta, max}$, as

\begin{equation}
v_{\theta}(r,t) = v_{\theta, max} (1 + \frac{1}{2 \alpha}) \frac{r_{max}}{r} 
[1 - exp(- \alpha \frac{r^2}{r_{max}^2})]
\label{eq:lamb3}
\end{equation}

\noindent
Where the core radius dilates according to 

\begin{equation}
r_{core}(t) = \sqrt{\alpha 4 \nu t}
\label{eq:lamb4}
\end{equation}

\noindent
Where $\alpha = 1.25643$ \cite{davenport1996}. Like the Rankine vortex, a 
Lamb-Oseen vortex does not allow for unsteady flow. In addition, the predicted 
centerline velocity at $r=0$ is undefined. It has utility in estimating average 
velocity profiles, and has been used to initialize large eddy 
simulations \cite{hennemann2011}.

A model which is widely used in the study of wake vortices, and demonstrates 
great skill at predicting the velocity profiles of large experimental vortices 
is the Burnham-Hallock model \cite{burnham1982}. This empirically based model 
has been independently discovered by many, and describes the 
azimuthal velocity profile as

\begin{equation}
v_{\theta}(r) = \frac{\Gamma_0}{2 \pi r_{core}} \frac{r^2}{r^2 + r_{core}^2}
\label{eq:burnham-hallock}
\end{equation}

\noindent
Similar to the Lamb-Oseen vortex, the Burnham-Hallock model has been widely 
used to initialize large eddy simulations and model aircraft response to wake 
encounters \cite{ahmad2014}.

\section{Vortex Model with non-equilibrium Pressure}

Non equilibrium pressure phenomena arises by the application of the 
Hamilton variational principle to describe the volume viscosity in fluids 
\cite{zuckerwar2006}. When subject to conservation of mass, conservation of 
reacting species, and material entropy constraints, the procedure yields two 
dissipative terms in the Navier Stokes equation; a traditional volume viscosity 
term and a term proportional to the material time derivative of the pressure 
gradient \cite{zuckerwar2009}. The resulting modified Navier Stokes equation 
can be expressed as

\begin{equation}
\rho \frac{Dv}{Dt} = - \nabla[P - \eta_p] -
\rho \nabla \Omega + 
\nabla[(\eta_v - \frac{2}{3} \mu)\nabla \cdot \pmb{\text{v}})] + 
\nabla \times (\mu \nabla \times \pmb{\text{v}}) + 
2[\nabla \cdot (\mu \nabla)] \pmb{\text{v}}
\label{eq:modified_ns1}
\end{equation}

\noindent
Where $\eta_p$ is the pressure relaxation coefficient, $\eta_v$ is the volume 
viscosity, and $\mu$ is the dynamic viscosity. If each of these is considered 
constant, and body forces are neglected, conservation of momentum simplifies to 

\begin{equation}
\rho \frac{Dv}{Dt} = - \nabla P + \eta_p \nabla \frac{DP}{Dt} - 
(\eta_v + \frac{4}{3}\nu) \nabla (\nabla \cdot\pmb{\text{v}})
- \mu \nabla \times (\nabla \times \cdot\pmb{\text{v}})
\label{eq:modified_ns2}
\end{equation}

\noindent
Alternatively, Equation \ref{eq:modified_ns2} can be expressed in index 
notation as 

\begin{equation}
\rho \frac{Dv_i}{Dt} = -\frac{\partial P}{\partial x_i} +
\eta_p \frac{D}{Dt} \Bigg( \frac{\partial P}{\partial x_i} \Bigg) + 
\mu \frac{\partial^2 v_i}{\partial x_{k}^2} + 
\eta_p \Bigg[\frac{\partial v_k}{\partial x_i} \frac{\partial P}{\partial x_k} -
\frac{\eta_v + (\frac{1}{3}\mu)}{\eta_p} \frac{\partial}{\partial 
x_i} \Bigg( \frac{1}{\rho} \frac{D\rho}{Dt} \Bigg) \Bigg]
\label{eq:modified_ns3}
\end{equation}

The term in brackets has been examined for a relationship with sound production 
in incompressible flows, but in any scenario should be negligibly small when 
multiplied by the pressure relaxation coefficient. Thus, Equation 
\ref{eq:modified_ns3} can be dramatically simplified \cite{ash2011}.

\begin{equation}
\rho \frac{Dv_i}{Dt} = -\frac{\partial P}{\partial x_i} +
\eta_p \frac{D}{Dt} \Bigg( \frac{\partial P}{\partial x_i} \Bigg) + 
\mu \frac{\partial^2 v_i}{\partial x_{k}^2} + 
\label{eq:modified_ns4}
\end{equation}

When applied to a steady, incompressible axial vortex with zero axial velocity, 
the azimuthal velocity profile is described by

\begin{equation}
v_\theta(r) = \frac{\Gamma_0}{\pi} \frac{\sqrt{2}}{R_\Gamma \sqrt{\nu \eta_p}}
\frac{(r / r_{core})}{(r/r_{core})^2 + 1}
\label{eq:ash_vortex_model}
\end{equation}

\noindent
Where $R_\Gamma$ is the circulation based Reynolds number. The maximum 
azimuthal velocity occurs where $r = r_{core}$, and can be given by 

\begin{equation}
v_{\theta, max} = \frac{\Gamma_0}{\pi} \frac{1}{R_\Gamma \sqrt{2 \nu 
\eta_p}}
\label{eq:ash_vthetamax}
\end{equation}

\noindent
Therefore, in terms of azimuthal velocity, Equation \ref{eq:ash_vortex_model} 
becomes

\begin{equation}
v_\theta(r) = 2 v_{\theta, max}\frac{(r / r_{core})}{(r/r_{core})^2 + 1}
\label{eq:ash_vortex_model2}
\end{equation}

Equation \ref{eq:ash_vortex_model2} is of the exact same form as the 
empirically based Burnham-Hallock model, but is an exact solution to the Navier 
Stokes equations. As the pressure relaxation coefficient shrinks to zero, the 
solution becomes a potential vortex similar to the Lamb-Oseen model. For very 
large pressure relaxation coefficients, the velocity distribution becomes 
similar to that of a rigid body rotation as in the Rankine model. 

(flag, unfinished) talk about more of the implications

\section{Turbulent Structure of Axial Vortices}

Talk about vortex turbulence!
