\section{Vortex models}
Several vortex models are already commonly used to describe the behavior of 
axial wake vortices. Among these models are, in order of introduction, the 
Rankine, Lamb-Oseen, Burnham-Hallock, and Proctor models \cite{ahmad2014}. The 
simplest and oldest model, the Rankine vortex, approximates the azimuthal 
velocity of the vortex within the core zone as a rigid body of rotating fluid, 
with velocity trailing off according to the inverse of radius at the core 
boundary \cite{rankine1869}. The azimuthal velocity profile of a Rankine vortex 
is given by Equations \ref{eq:rankine1} and \ref{eq:rankine2}.

\begin{equation}
v_{\theta}(r) = \frac{\Gamma_0}{2 \pi r_{core}} \frac{r}{r_{core}} 
	\text{ for } r \leq r_{core}
	\label{eq:rankine1}
\end{equation}

\begin{equation}
v_{\theta}(r) = \frac{\Gamma_0}{2 \pi r}
	\label{eq:rankine2}
\end{equation}

\noindent
The azimuthal velocity $v_{\theta}$, is proportional to radius $r$, in the core 
zone, but proportional to $1/r$ in the area outside the core zone. This 
discontinuity makes the model unsuitable in the vicinity of the core region of 
the vortex. The model is unsuitable for the study of unsteady flow or 
turbulence, and the rigid body assumption at the core has been experimentally 
shown to be invalid.

Significantly later, the Lamb-Oseen vortex model was developed from an 
analytical solution to the Navier-Stokes equations. In this model, a potential 
vortex with an infinite velocity at the centerline was subject to viscous 
decay. The solution produces a continuous, unsteady function of the azimuthal 
velocity profile as shown in Equations \ref{eq:lamb1} and \ref{eq:lamb2} 
\cite{lamb1932}.

\begin{equation}
v_{\theta}(r,t) = \frac{\Gamma_0}{2 \pi r_{core}}[1 - 
							exp(-(r / r_{core})^2)]
	\label{eq:lamb1}
\end{equation}

\noindent
Where the core radius dilates according to 

\begin{equation}
r_{core}(t) = \sqrt{4 \nu t}
	\label{eq:lamb2}
\end{equation}

\noindent
Alternatively, in terms of the maximum azimuthal velocity, $v_{\theta, max}$  
can be expressed as

\begin{equation}
v_{\theta}(r,t) = v_{\theta, max} (1 + \frac{1}{2 \alpha}) \frac{r_{max}}{r} 
[1 - exp(- \alpha \frac{r^2}{r_{max}^2})]
\label{eq:lamb3}
\end{equation}

\noindent
Where the core radius dilates according to 

\begin{equation}
r_{core}(t) = \sqrt{\alpha 4 \nu t}
\label{eq:lamb4}
\end{equation}

\noindent
Where $\alpha = 1.25643$ \cite{davenport1996}. Like the Rankine vortex, a 
Lamb-Oseen vortex does not allow for unsteady flow. In addition, the predicted 
centerline velocity at $r=0$ is undefined. It has utility in estimating average 
velocity profiles, and has been used to initialize large eddy simulations.

A model which demonstrates great skill at predicting the velocity profiles of 
large experimental vortices is the Burnham-Hallock model.\cite{burnam2013}.

Modeling the turbulence within a large scale vortex 
is difficult, as all length scales must be considered.