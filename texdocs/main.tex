\tolerance=10000

% general packages that seem to be required by everything
\documentclass[12pt]{report}
\usepackage{geometry}
\usepackage[utf8]{inputenc}
\usepackage[english]{babel}
\usepackage{graphicx}

\usepackage{ODUthesis} 		% custom template specific to ODU thesis.
\usepackage{indentfirst} 	% indents first line of a section (style guide)
\usepackage{hyperref} 		% allows use of hyperlinks
\usepackage{gensymb} 		% allows use of degree symbol with \degree
\usepackage{longtable} 		% formats tables
\usepackage{float} 			% allows use of figure{}[H]
\usepackage{subcaption} 	% allows mutliple figures for one caption
\usepackage{mathtools}		% mathtools including multi lined equations

\usepackage[font=small, labelfont=bf]{caption} 	% formats figure captions

\graphicspath{{figs/}} 		% defines reference path for images and figures

%This style follows conventions used in Physical Review C. The names of 
%figures, tables and captions can be changed by using the commands below with 
%appropriate changes. For example, "FIG." could be replaced by "Fig." or 
%"Figure" to change the labeling of figure captions.

%\renewcommand{\figurename}{FIG.}
%\renewcommand{\tablename}{TABLE}
%\renewcommand{\bibname}{BIBLIOGRAPHY}

\begin{document}

\title{Experimental Investigation of Turbulent Structures and Non-Equilibrium 
Effects in Axial Wake Vortices Via Particle Image Velocimetry}
\author{Jeffry William Ely}
\principaladviser{Robert Ash}
\member{Drew Landman} 
\member{Colin Britcher} 
\degrees{Bachelor of Science in Mechanical Engineering}
\dept{Aerospace Engineering}
\submitdate{May 2016}
\vita{Bachelors of Mechanical Engineering, Old Dominion University, December 
2011}

\phdfalse 			%produces language on title page for Masters Thesis
%\copyrightfalse    %suppresses copyright notice
%\figurespagefalse  %suppresses List of Figures
%\tablespagefalse   %suppresses List of Tables

%=========================================================================
\abstract{
	Vortices are a common phenomenon in fluid flows that arise to dissipate 
	kinetic energy into heat via viscous interaction. They arise naturally at 
	large scales in the form of dust devils, tornadoes, and in the wake of 
	aircraft. It is important to understand the conditions leading to their 
	formation, their duration, and their dissipation to prevent undesirable 
	effects. Among these effects is an decrease in safety of aircraft 
	operations in the wake of other aircraft, an extremely common situation at 
	airports around the world. A large number of mathematical models and 
	experimental data exists to help explain various aspects of axial wake 
	vortex behavior, but current models fail to understand why many vortices 
	remain tightly wound about their cores for as long as they have been 
	observed to do. The current study builds upon the theoretical work of Ash, 
	Zardadkhan and Zuckerwar \cite{ash2011}, and tests specific attributes of a 
	vortex for agreement with non-equilibrium pressure relaxation theory. A 
	bi-wing vortex generator was constructed, and placed in a low speed wind 
	tunnel. Stereo particle image velocimetry was employed to map three 
	dimensional velocity vectors in a plan perpendicular to the free stream 
	flow direction at a rate of 1Hz for 200 seconds. A Reynolds time averaging 
	approach was used to synthesize this velocity data into stable and 
	fluctuating components. Free stream velocity and downstream distance were 
	varied to create a variety of vortices, and to ensure the availability of 
	high quality data.
	}

\beforepreface

%\prefacesection{Acknowledgements}

%text of Acknowledgements goes here any other preface section is inserted 
%similarly by the command

\prefacesection{Notes}
Good practice for preserving the code used to perform this research was 
attempted by use of the Git work flow and a Git repository. All code used to 
perform data analysis and visualization, along with the raw three dimensional 
vector dataset can be downloaded for free from 
\url{https://github.com/Jwely/pivpr}.
	

\afterpreface

%=========================================================================
% introduction chapter and subdocuments
\chapter{Introduction}
\include{docs/intro/prob_statement}
\include{docs/intro/vortex_modeling}
\section{Fundamentals of Particle Image Velocimetry}

Particle Image Velocimetry, or PIV, is a class of methods employed by 
experimental fluid mechanics to measure instantaneous vector velocity fields by 
measuring the displacements of small visible particles which follow the motion 
of the fluid \cite{adrian2011}. Figure \ref{fig:quiver_example} shows a typical 
resulting vector field from measurements of a vortical flow. Each of these 
two-dimensional vectors was measured simultaneously in a thin sheet of fluid. 
The velocity of the fluid is sensed by acquiring images of well entrained 
particles at precise times and measuring the displacement of those particles. 
This is accomplished with the use of cameras for acquiring images and an 
intense laser to illuminate particles within the desired plane. 
This technique can be used to study flows in gases and liquids, and is derived 
from techniques originally developed to measure deformations on the surface of 
solid material.

\begin{figure}[H]
	\centering
	\includegraphics[width=5in]{figs/example_vortex_figs/example_quiver}
	\caption{Velocity vector field of a vortical flow structure as produced by 
	PIV.}
	\label{fig:quiver_example}
\end{figure}

 
\subsection{The Case for PIV}

Unlike other flow measurement techniques, PIV is a non-invasive method to 
directly measure time and displacement, and thus velocity. PIV is also capable 
of resolving vector measurements at many positions within a two dimensional 
slice of the flow field simultaneously, while other measurement techniques 
require taking data at many locations sequentially over a much greater period 
of time. Single camera PIV can measure two components of the velocity vectors 
aligned with the image plane, but the PIV method is capable of resolving many 
dimensions of fluid flow with incremental increases in system complexity. The 
addition of another camera allows the full three dimensional velocity vector to 
be measured, and a sweeping beam laser allows the interrogation of an entire 
volume of flow field instead of a slice.

Stereo PIV is used widely because it provides a full velocity vector, and 
requires only an additional camera, slightly more complex calibration and 
software to process the imagery.  A stereo PIV system with a 
stationary sheet laser can resolve three dimensional velocity vectors and their 
fluctuations within a two dimensional slice of fluid flow. A two point 
correlation tensor containing important information about the turbulent 
structure of a flow can be obtained readily with PIV. The non-invasive nature 
of PIV, combined with the  ability to interrogate a flow volume very quickly 
for information with high dimensionality makes it exceptionally useful in fluid 
mechanics. \cite{adrian1991}

\subsection{Principles of Planar PIV}

A simple planar PIV system consists of a double pulsed laser, light sheet 
forming optics, particle seed, a single lens camera, image digitization 
hardware, and a computer system for data storage and subsequent analysis. The 
underlying concept behind all PIV is that light scattered from the particles as 
they move through the flow field allows a pair of images to capture information 
about the motion of that particle. Double pulsed illumination is commonly used 
in PIV systems due to its relatively low cost and complexity compared with 
multiple laser source systems. The energy required to adequately illuminate an 
area of interest depends upon the size of that area, and the scattering 
properties of the particle seed. Solid state Nd:YAG lasers are typically used 
for this purpose \cite{adrian2011}. An $(X, Y, Z)$ coordinate system 
is defined 
within the light sheet that exists in $(X, Y)$ space, and relates linearly to a 
coordinate system in the image plane of the camera in $(X_p, Y_p)$ pixel space. 
At a time $t$, a laser light sheet is produced for a short pulse and the camera 
captures an image of the light scattering from particles within the flow. At 
some short time later, $t + dt$, a second pulse occurs and a second image is 
taken. By measuring the pixel displacements $(\Delta X_p, \Delta Y_p)$ for a 
particle in the image plane, and transforming the coordinates into the plane of 
the laser one obtains $(\Delta X, \Delta Y, \Delta Z)$. This coordinate 
transform can be obtained from precise information about the optical geometry 
of the system, or by direct measurement of calibration data \cite{fouras2007}.
The method of direct measurement of a calibration target was used in this 
research. 

The detection rate, accuracy, and reliability of PIV depends upon careful 
selection of experimental parameters. Kean and Adrian suggest a set of six 
important dimensionless quantities and ideal operating ranges that improve the 
chances of high quality PIV results . The parameters 
are data validation criterion, particle image density, relative in-plane image 
displacement, relative out-of-plane displacement, velocity gradient, and the 
ratio of the mean image diameter to the interrogation spot diameter 
\cite{kean1990,kean1991}. Steep velocity gradients create random errors as the 
individual particle displacements cover a discreet range within an 
interrogation spot. Particle displacements should be restricted to 25\% of the 
interrogation spot diameter for in-plane velocities, and to 25\% of the light 
sheet thickness in out-of-plane velocities. One way to mitigate both of these 
effects is to use as high resolution PIV as possible, or by scaling the 
interrogation area down by altering the zoom of the cameras. Since the velocity 
profile of an axial vortex can contain high velocity gradients in the vicinity 
of the core, and both in-plane and out-of-plane velocities are expected to be 
high, the present experiments used 50mm lenses to zoom each camera to 
focus on a small an interrogation plane as possible.

\subsection{Principles of Stereo PIV}

Classical single camera PIV is only capable of capturing the projection of the 
velocity vector on the image plane. The out-of-plane component is lost 
completely, and the in-plane components are affected by unrecoverable error. In 
Stereo PIV, a pair of cameras may obliquely view the same plane and the entire 
3d velocity vector may be inferred using the camera geometry. This technique 
includes tilting the backplanes of the cameras to satisfy the Scheimpflug image 
criteria and a dewarping function based on camera geometry to account for 
projective distortion \cite{willert1997}. An example of the geometric set up 
used by Willert to study the movement of a ring vortex through the 
interrogation plane is shown in Figure \ref{fig:stereo_piv}. Uncertainties 
associated with high out-of-plane motion in planar PIV are greatly reduced in 
stereo PIV.

\begin{figure}[H]
	\centering
	\includegraphics[width=5in]{figs/piv_method/stereo_piv_optics}
	\caption{Stereo camera PIV system for mapping three dimensional velocity 
		vectors}
	\label{fig:stereo_piv}
\end{figure}  

\subsection{Particles}

\subsubsection{Particle Dynamics} 

\subsubsection{Error Due to Slip}

\subsubsection{Seeding Particles for PIV}

\subsection{Image processing}

\cite{soloff1997, willert1997}.

\subsection{Interrogation}

\subsection{Measurement of Fluid Flow}

\subsection{PIV in Three Dimensions}

\subsection{Practical PIV Systems}




%=========================================================================
% experiment setup chapter and sub documents
\chapter{Experiment Setup}
A complete particle image velocimetry system was installed in the ODU low speed 
wind tunnel and employed to measure three-dimensional velocity fields produced 
by an axial vortex at multiple nominal velocities $V_{nom}$ and multiple 
interrogation planes. The interrogation planes were defined by their 
distance downstream of trailing edges of the bi-wing the vortex generator 
$I_Z$. Nominal wind tunnel velocity was varied from 15$m/s$ to 33$m/s$, 
sampling 10 distinct velocities in increments of 2$m/s$. The interrogation 
plane was moved at irregular intervals from 546$mm$ to 1016$mm$ as shown in 
table \ref{table:test_matrix_table}. This chapter discuss details of the 
experimental setup used to produce these datasets, including wind tunnel 
control, vortex generator setup, PIV system calibration, data acquisition, data 
processing, and data quality control.

\begin{table}[H]
\begin{center}
\begin{tabular}{|ccc||ccc||ccc|}
	\hline
	Run & $I_Z$  & $V_{nom}$ & Run & $I_Z$  & $V_{nom}$ & Run & $I_Z$  & $V_{nom}$\\
	ID & ($mm$) & ($m/s$) & ID & ($mm$) & ($m/s$) & ID & ($mm$) & ($m/s$)\\
	\hline
	1 & 546 & 15 & 26 & 787 & 25 & 51 & 965 & 15\\
	2 & 546 & 17 & 27 & 787 & 27 & 52 & 965 & 17\\
	3 & 546 & 19 & 28 & 787 & 29 & 53 & 965 & 19\\
	4 & 546 & 21 & 29 & 787 & 31 & 54 & 965 & 21\\
	5 & 546 & 23 & 30 & 787 & 33 & 55 & 965 & 23\\
	6 & 546 & 25 & 31 & 863 & 15 & 56 & 965 & 25\\
	7 & 546 & 27 & 32 & 863 & 17 & 57 & 965 & 27\\
	8 & 546 & 29 & 33 & 863 & 19 & 58 & 965 & 29\\
	9 & 546 & 31 & 34 & 863 & 21 & 59 & 965 & 31\\
	10 & 546 & 33 & 35 & 863 & 23 & 60 & 965 & 33\\
	11 & 708 & 15 & 36 & 863 & 25 & 61 & 1016 & 15\\
	12 & 708 & 17 & 37 & 863 & 27 & 62 & 1016 & 17\\
	13 & 708 & 19 & 38 & 863 & 29 & 63 & 1016 & 19\\
	14 & 708 & 21 & 39 & 863 & 31 & 64 & 1016 & 21\\
	15 & 708 & 23 & 40 & 863 & 33 & 65 & 1016 & 23\\
	16 & 708 & 25 & 41 & 914 & 15 & 66 & 1016 & 25\\
	17 & 708 & 27 & 42 & 914 & 17 & 67 & 1016 & 27\\
	18 & 708 & 29 & 43 & 914 & 19 & 68 & 1016 & 29\\
	19 & 708 & 31 & 44 & 914 & 21 & 69 & 1016 & 31\\
	20 & 708 & 33 & 45 & 914 & 23 & 70 & 1016 & 33\\
	21 & 787 & 15 & 46 & 914 & 25 &   &   &  \\
	22 & 787 & 17 & 47 & 914 & 27 &   &   &  \\
	23 & 787 & 19 & 48 & 914 & 29 &   &   &  \\
	24 & 787 & 21 & 49 & 914 & 31 &   &   &  \\
	25 & 787 & 23 & 50 & 914 & 33 &   &   &  \\
	\hline
\end{tabular}
\caption{Experimental conditions for all 70 experiments.}
\label{table:test_matrix_table}
\end{center}
\end{table}



\section{Low Speed Wind Tunnel}

The Old Dominion University low 
speed wind tunnel (LSWT) was outfitted with a bi-wing axial vortex generator. 
The tunnel has a large test section measuring 2.134$m$ wide by 2.438$m$ 
tall and a small test section measuring 1.219$m$ wide by 0.911$m$ tall. The 
wind tunnel air is propelled with a frequency controlled 125 Horsepower motor. 
Flow velocity was manipulated directly by manually controlling
voltage supplied to the controller. 
The small test section has a total length of 2.438$m$, and the vortex generator 
spaned the 0.911$m$ height of the test section and was 
mounted 0.610$m$ from the front end, leaving 1.829$m$ downstream for the axial 
vortex to develop. The 3x4 foot test section has a functional free stream 
velocity range between 12 and 55 meters per second or between 35 and 120 miles 
per hour.

\begin{figure}[H]
\centering
\includegraphics[width=5in]{figs/setup/odulswt_diagram}
\caption{ODU Low speed wind tunnel.}
\label{fig:odulswt}
\end{figure}

The entire interior of the test section remained vacant and unobstructed with 
the exception of the vortex generator. 
No internal traverse systems or structures 
were present during PIV data acquisition unless otherwise indicated. Tunnel 
velocity was determined by direct measurement of dynamic pressure ($q$), which 
is monitored and controlled by the tunnel control PC. Figure 
\ref{fig:control_diagram} contains a schematic diagram of the systems under the 
wind tunnel control PC.

\begin{figure}[H]
\centering
\includegraphics[width=5in]{figs/setup/odulswt_control}
\caption{Schematic diagram of systems under wind tunnel PC control.}
\label{fig:control_diagram}
\end{figure}



\section{Vortex Generator}

It is desirable to enhance understanding of 
trailing axial vortices generated by aircraft wingtips and their behavior.
While naturally occurring axial vortexes occur in unpredictable and non-uniform 
environments, this experimental investigation required that a repeatable axial 
vortex.  Small-scale aircraft axial wake vortices can be generated in an 
enclosed wind tunnel environment with a single wingtip, but the downwash 
behavior of a wingtip vortex is distorted by the walls of the wind tunnel and 
test section. A bi-wing vortex generator, as pictured in Figures 
\ref{fig:vortex_gen} and \ref{fig:vortex_design}, was designed and constructed 
by undergraduate student researchers to generate a vortex formed by merging two 
juncture vortices of opposite sign and resulting in a single vortex, absent any 
downwash \cite{davis2012}. The vortex generator employed two symmetric NACA-0012
airfoils with chord length of 101.6$mm$ manufactured from foam casts attached 
to a 25.4$mm$ diameter cylindrical center body with hemispherical forward and 
aft end caps. While the vortex generator could be permitted to have variable
angles-of-attack, the wings were locked at an angle of attack of $\pm$8 degrees 
to avoid vortex structural distortions that could result from non-repeatable 
changes in angle-of-attack.

\begin{figure}[H]
\centering
\includegraphics[width=4in]{figs/setup/vortex_generator/picture}
\caption{Picture of the vortex generator set up in the ODU LSWT.}
\label{fig:vortex_gen}
\end{figure}

\begin{figure}[H]
\centering
\includegraphics[width=6.5in]{figs/setup/vortex_generator/design}
\caption{CAD design of the vortex generator used in this study.}
\label{fig:vortex_design}
\end{figure}



\section{PIV Overview}

The present study uses a stereo PIV system which resolves three dimensional 
near-instantaneous velocity vectors gridded on a two dimensional cross section 
of flow. The cameras are capable of taking two images just a few microseconds 
apart. Determining velocity vectors requires two images just a few 
microseconds apart, but pairs of images can be taken at greater time intervals. 
The PIV method used in this study utilizes a "frame-straddling" technique that 
spaces the time interval between two laser pulses so that the camera sensor 
arrays receive the first laser illuminated image data as near as possible to 
the closing of the shutter, followed by a second laser illuminated image as 
close as possible to the beginning of the second shutter opening. This 
technique allows both cameras to capture image sequence pairs simultaneously, 
spaced 25 to 50 microseconds apart. This can be repeated once every second, 
resulting in a true velocity sampling frequency of 1 Hz.

\subsection{Seeding the Flow}

Appropriate particle seeding density and time between straddled frames is the
subject of continued study, and is difficult to predict \textit{a priori}. 
Complete coverage of a two dimensional vector field is highly dependent upon 
uniform optimal particle density conditions which are difficult to obtain, and 
maintain over an extended test interval, due to seed-particle accumulation. For 
stereo PIV, incomplete data in either of the two dimensional vector 
sets from either camera at a given spatial location will result in an 
indeterminate vector displacement in the three dimensional vector data. To 
elevate the likelihood that a displacement vector at a given location can be 
properly determined, an additional data refining technique outlined by 
\cite{hart1998} was employed. The Hart method compares correlation maps 
between adjacent vector spots to produce fewer errors than would otherwise be 
produced by completely independent evaluation of each small image sector. In 
instances where two adjacent regions lack a well-defined peak, the Hart method 
emphasizes shared 
peaks in order to reveal a significant correlation that might otherwise have 
been missed. In instances 
where sub optimal seeding conditions exist and a correlation map 
produces a false peak, the Hart method can isolate and eliminate those 
anomalous peaks based on comparison with adjacent sectors \cite{hart1998}. The 
actual process by which the Hart method reduces erroneous vectors is difficult 
to quantify on a case by case basis, but is expected to have a positive effect 
and reduce overall uncertainty in the PIV measurement.


\section{Stereo PIV Data Acquisition}

The low speed wind tunnel was outfitted with a stereo particle image 
velocimetry system which included two cameras that were mounted with a simple 
frame built from 80$mm$x120$mm$ T-slot extruded aluminum with six sliding 
fastener points on the exterior of the wind tunnel, just outside the test 
section on the left and right sides as shown in figure \ref{fig:pivsetup}. 

\begin{figure}[H]
	\centering
	\includegraphics[width=5in]{figs/piv_method/piv_camera_diagram}
	\caption{Top view schematic of PIV camera positions. A-vortex generator, 
	B-PIV 
		calibration target and interrogation plane, C-left camera, D-right 
		camera, E-station distance dimension.}
	\label{fig:pivsetup}
\end{figure}

The roof of the wind tunnel has a long glass window along the center of the 
test section such that an Nd:YAG laser could be mounted to create a vertically 
oriented laser sheet. This laser sheet functioned to illuminate the fluid flow 
perpendicular to the free stream velocity vector as 
shown in Figure \ref{fig:laser_sheet_picture}. In this image, the fluid flow 
has been freshly seeded with a wand to deposit concentrated fog directly 
downstream of the vortex generator to visualize a vortex with a clearly 
defined, smoke free, vortex core. Bright outer lines mark the edges of the 
light curtain. This was initially performed to aid in the alignment of a 
pressure probe (visible in the image) with the vortex core. PIV data was not 
taken with the fog wand or the pressure probe present in the wind tunnel.


The equipment used for this study was a "TSI Stereo Image Velocimeter System", 
which consists of a pair of TSI PIV 13-8 cameras (Figure 
\ref{fig:camera_picture}), a TSI synchronizer and frame 
grabber specific to the cameras (Figures \ref{fig:synchronizer} and 
\ref{fig:synchronizer2}), a New Wave Dual Mini-YAG Laser (Figure 
\ref{fig:laser}) and Light Guide, a precision 3-D calibration target for stereo 
PIV camera alignment, and a precision laser traverse system. The particle seed 
was generated by an MDG fog generator (Figure \ref{fig:fog_machine}). The TSI 
INSIGHT\textsuperscript{\textcopyright} software was used 
for data acquisition and image to vector processing.

\begin{figure}[H]
	\centering
	\includegraphics[width=5in]{figs/piv_method/laser_sheet_picture}
	\caption{Picture of a light curtain illuminating a cross section of an 
	axial wake vortex.}
	\label{fig:laser_sheet_picture}
\end{figure}

\begin{figure}[H]
	\centering
	\includegraphics[width=5in]{figs/piv_method/piv_cams}
	\caption{Photograph of two TSI PIV 13-8 cameras. Model 630047}
	\label{fig:camera_picture}
\end{figure}

\begin{figure}[H]
	\centering
	\includegraphics[width=5in]{figs/piv_method/synchronizer}
	\caption{Photograph of LaserPulse synchronizer by TSI. Model 610034}
	\label{fig:synchronizer}
\end{figure}

\begin{figure}[H]
	\centering
	\includegraphics[width=5in]{figs/piv_method/synchronizer_rear}
	\caption{Photograph of synchronizer connectors. Model 610034}
	\label{fig:synchronizer2}
\end{figure}

\begin{figure}[H]
	\centering
	\includegraphics[width=5in]{figs/piv_method/laser}
	\caption{Photograph of a PIV laser by SoloPIV. Model 610034}
	\label{fig:laser}
\end{figure}

\begin{figure}[H]
	\centering
	\includegraphics[width=5in]{figs/piv_method/fog_generator}
	\caption{Photograph MDG MAX 3000APS Fog Generator.}
	\label{fig:fog_machine}
\end{figure}

%\begin{figure}[H]
%	\centering
%	\includegraphics[width=5in]{figs/piv_method/INSIGHT_processing}
%	\caption{Photograph of INSIGHT software computing stereo velocity fields}
%	\label{fig:processing_screenshot}
%\end{figure}

\subsection{Synchronizing the Cameras and Laser}
In order to resolve particle displacements on such a small scale in a 
relatively fast moving fluid, the time interval between a laser pulse sequence 
must be tuned to allow sufficient particle displacement occur to obtain a 
meaningful velocity measurement. If the time interval ($dt$) between successive 
exposures was too long, the particles escaped the interrogation plane, and 
could not be tracked. This study varied the laser pulse timing 
between 25 microseconds for higher wind tunnel speeds, and 50 microseconds for 
slower vortex velocities associated with slower wind tunnel speed. As 
introduced in Chapter 1, this PIV system employed a frame straddling technique, 
which initiates the first shutter opening well before the first laser 
pulse begins such that the camera shutter first closes at as the first laser 
pulse is ending, but before the second laser pulse starts. The second exposure 
begins just as the second laser pulse is initiating, though it extends well 
beyond the termination of the first laser pulse as shown schematically in 
Figure \ref{fig:frame_straddling}. 

\begin{figure}
	\centering
	\includegraphics[width=5.5in]{figs/piv_method/frame_straddling}
	\caption{Frame straddling technique. Laser pulses are shown in green, and 
		camera exposures are shown in gray.}
	\label{fig:frame_straddling}
\end{figure}

Timing is achieved with a TSI synchronizer (Figure \ref{fig:synchronizer}) and 
control PC shown in Figure 
\ref{fig:pivblockdiagram}. The control PC sends a command to the synchronizer 
to take an image sample, then the synchronizer sends precisely timed signals to 
operate both cameras and the laser control unit. The cameras each store the 
image pair in an internal memory buffer before returning the data to the 
control PC.

\begin{figure}[H]
	\centering
	\includegraphics[width=4in]{figs/piv_method/experiment_block_diagram}
	\caption{Blockdiagram of PIV hardware components. C-left camera, D-right 
		camera, F-Nd:YAG laser, G-laser control unit and power supply, 
		H-synchronizer, 
		J-control PC running INSIGHT software suite.}
	\label{fig:pivblockdiagram}
\end{figure}

\subsection{Calibration employing a PIV Target}

Dimensional calibration was accomplished with the use of a calibration target, 
which was a 10$cm$ by 10$cm$ black calibration plate with precisely positioned 
white divots and a center fiducial mark ,as shown in figure 
\ref{fig:calibration_target}. The calibration target was positioned inside the 
tunnel test section and aligned with the desired interrogation plane. 
The laser illumination sheet was operated in a continuous firing mode and aimed 
into the test section so that the location of the interrogation sheet is 
clearly visible with the use of polarized eye protecting goggles. The camera 
facing surface of the 
interrogation target is then manually aligned with the interrogation plane, 
which was visible as an intense line of light with a nominal thickness of 
1.5$mm$ observable across the top of the target. Once the calibration target 
was properly aligned, the laser and all other light sources in the laboratory 
were extinguished and a lamp was placed downstream of the test section (to 
create ideal lighting conditions) in order to illuminate the target for camera 
focusing. Cameras outfitted with 50mm lenses were manually aimed at the 
calibration target, taking care to place the fiducial mark near the center of 
the image frame. Focusing was accomplished manually by adjusting each lens, and 
a limited PIV data set was created after focusing to ensure that the focus ws 
sharp enough to resolve a partially complete velocity vector field within the 
interrogation plane.

\begin{figure}[H]
	\centering
	\includegraphics[width=4in]{figs/piv_method/calibration_target}
	\caption{Inverted color and elevated contrast photograph of the calibration 
		target, highlighting the central fiducial mark.}
	\label{fig:calibration_target}
\end{figure}


INSIGHT\textsuperscript{\textcopyright} software was used to take snapshots and 
view them to ensure proper focus 
and alignment. Once satisfactory adjustments were obtained, calibration images 
were taken and the imaging software was used to recognize the fiducial mark, as 
well as each divot, in order to create the necessary two dimensional coordinate 
transformation maps that relate
pixel distance to physical distance. This calibration was employed both to 
determine the magnitudes of the two dimensional velocity vectors unique to each 
camera, and aligning the two separate vector fields, in order to create a three 
dimensional velocity field. Once calibration images were acquired, the 
target was removed and data were taken over a range of velocities in 
that interrogation plane. Each interrogation plane was defined by its position 
downstream relative to the trailing edge of the vortex generator center body. 
Seven interrogation planes were utilized in this study, and every time the 
survey plane was changed, it was necessary to repeat the calibration process. 
Ideally, as dictated by good design of experiments 
practice, the order of experiments should be randomized. This, however, would 
have required exclusive tunnel use for a longer period of time than could be 
scheduled due to the level of demand at the time.

\subsection{Seeding the Flow}

Gathering data with a PIV system requires the flow be properly seeded with 
observable particles. Particles should be sufficiently small to be entrained in 
the flow such that particle motion and fluid motion are the same. In the 
present study, $CO_2$ filled mineral oil bubbles generated with an MDG MAX 3000 
APS fog generator with a Laskin nozzle were used as particle seed. These 
droplets have a nominal diameter of 0.6 micrometers and densities of 1.55 times 
that of air were used \cite{mdgfog}. This was within the optimal range of 
particle densities for good particle entrainment and representation of the 
fluid flow as defined by \cite{mei1996}, as discussed in Chapter 1.

Each velocity measurement 
required a sector of pixels through which displacement could be tracked. If 
a majority of the illuminated particles were allowed to travel outside of a 
given sector between consecutive frame straddled image captures, a meaningful 
correlation map could not be generated. Care was taken to ensure that the 
observable particles 
traversed no more than 50\% of a sector width during the time interval between 
images by adjusting the time interval between image captures ($dt$). 
Utilization of a 50\% border when processing each sector, as described in the 
previous section, helped to ensure that particles did not escape the 
interrogation area. A sufficient quantity of particle groups must be visible in 
each sector in order to yield a meaningful correlation map. If the particle 
seed density was too low, or too high, it would result in a somewhat 
homogeneous image, with ambiguous peaks in the correlation map. A situation 
which results in an increased frequency of spurious measurements with a 
poor signal to noise ratio. Manipulation of sector size 
can increase the number of particles and the reliability of the correlation 
mapping, but only at the expense of reduced velocity field resolution, since 
more pixels are required to resolve each vector.


\subsection{Pressure Measurements}

Pressure measurements within the vortex core could assist in verifying the 
predicted non-eqiulirbium pressure influences on the behavior of axial wake 
vortices, especially when combined with simultaneous three dimensional 
turbulence measurements with PIV \cite{ash2011}.


Initially, it was believed that a pressure probe could be positioned 
along the axis of the vortex core. Without positioning the probe in many 
locations across the flow and post-processing the data set, it was necessary to 
have some other way to verify that the pressure probe was in fact positioned 
within the vortex core. Simultaneous PIV and pressure measurements were 
attempted utilizing a seven hole probe, but the probe and probe mounting system 
was found to severely obstruct the PIV laser sheet and camera views, and 
created bright reflections that saturated the image values in the vicinity of 
the vortex. No usable PIV data could be obtained when the a pressure probe was 
mounted in the tunnel test section. Direct visual inspection of the actual 
vortex core employing a continuous laser sheet, revealed that the core appeared 
to dilate significantly in the presence of the pressure probe. Without a method 
to compensate for this potential core dilation, pressure measurements of the 
vortex core region have not been included in this study. It is possible that, 
with a much larger vortex core which is expected to be less 
sensitive to the presence of an inserted pressure probe, and when coated with 
a non-reflective coating utilizing forward mounted PIV cameras looking 
downrange, simultaneous vortex core pressure measurements and PIV measurements 
could be obtained.

\subsection{Capturing PIV Imagery}

The experiments were conducted with the interrogation plane at seven 
downstream  between and 546$mm$ and 1016$mm$. Each of these downstream 
locations will be referred to as 
a "station". Tests were conducted starting nearest to the vortex 
generator and progressing downstream. Since changing the position of the 
interrogation plane required a complete re-calibration of the PIV system, the 
testing sequence was not randomized. For each station, data were taken for each 
of 
the ten test velocities between 15$m/s$ and 33$m/s$ in ascending order, gaining 
speed over time. Image pairs were acquired once per second, at a 
rate of $1Hz$ for a period of 200 seconds, generating a total of 200 image 
sets. Each test set of images was then processed using INSIGHT software in 
order to produce text files containing three-dimensional vector fields for each 
image set.

Tunnel conditions were monitored closely and recorded for the duration of each 
test. Ancillary wet bulb and dry bulb temperatures were taken with a sling 
psychrometer. The amount of mineral oil fog (seed) was not controllable in a 
quantitative manner over an extended period of testing, because the dissipation 
rate was not constant. Additional fog was added to the tunnel through a hose in 
the tunnel wall, far upstream of the high speed test section, on a qualitative 
basis. Prior to actual PIV data acquisition, a few test PIV images 
were captured, and a low quality two dimensional computation was performed real 
time for a heads-up evaluation of data quality. 
On occasion, low data quality indicated a low particle density in the tunnel, 
so additional particle fog was added. 
Specific details for each test including the relative humidity ($\phi$) derived 
from wet bulb and dry bulb measurements and associated pressure relaxation 
coefficient 
($\eta_P$), are summarized in Tables \ref{table:station_1_measurements} through 
\ref{table:station_7_measurements}. 

Stations are defined by the position of the interrogation plane, measured 
downstream from the wing edge of the bi-wing vortex generator, expressed as
$I_Z$. Nominal velocity is denoted by $V_{nom}$, while actual mean free stream 
velocity is denoted by $V_{fs}$, and variance in the measurement for the 
duration of the test is labeled $\sigma_{V_{fs}}$. Dynamic pressure is $Q$, 
atmospheric 
pressure is $P_{atm}$, and the tunnel temperature is $T_{tunnel}$. Relative 
humidity $\phi$ was calculated from wet bulb and dry bulb temperatures ($T_w$ 
and $T$, respectively) by 
equations \ref{eq:rh_es} through \ref{eq:rh_rh} according to \cite{owen1977}. 
Finally, pressure relaxation coefficient is calculated according to the table 
in \cite{ash2011}.

\begin{equation}
e_S = 6.112 exp \left( \frac{17.67 T}{T + 243.5} \right)
\label{eq:rh_es}
\end{equation}

\begin{equation}
e_W = 6.112 exp \left( \frac{17.67 T_W}{T_W + 243.5} \right)
\label{eq:rh_ew}
\end{equation}

\begin{equation}
e = e_W - P_{atm} (T - T_W) 0.00066(1 +( 0.00115T_W))
\label{eq:rh_e}
\end{equation}

\begin{equation}
\phi = 100 \frac{e}{e_S}
\label{eq:rh_rh}
\end{equation}




\renewcommand\baselinestretch{1.3}\selectfont
\begin{table}[H]
\begin{center}
\begin{tabular}{|ccccccccccc|}
	\hline
	Run & $I_Z$ & $V_{nom}$ & $dt$ & $V_{fs}$ & $\sigma_{V_{fs}}$ & $Q$ & $P_{atm}$ & $T_{tunnel}$ & $\phi$ & $\eta_P$\\
	$ID$ & $(mm)$ & $(m/s)$ & $(\mu s)$ & $(m/s)$ & $(m/s)$ & $(Pa)$ & $(Pa)$ & $(\degree K)$ & $(\%)$ & $(\mu s)$\\
	\hline
	1 & 546 & 15 & 50 & 15.22 & 0.02 & 135 & 102036 & 299.85 & 60.4 & 0.35\\
	2 & 546 & 17 & 50 & 16.88 & 0.02 & 170 & 102115 & 297.55 & 66.3 & 0.329\\
	3 & 546 & 19 & 50 & 19.43 & 0.01 & 225 & 102105 & 297.55 & 66.3 & 0.329\\
	4 & 546 & 21 & 50 & 21.06 & 0.02 & 264 & 102100 & 297.75 & 66.3 & 0.324\\
	5 & 546 & 23 & 35 & 23.21 & 0.04 & 321 & 102097 & 297.95 & 66.3 & 0.324\\
	6 & 546 & 25 & 35 & 24.86 & 0.05 & 371 & 102093 & 298.15 & 66.3 & 0.324\\
	7 & 546 & 27 & 35 & 27.02 & 0.03 & 434 & 102092 & 298.3 & 66.3 & 0.324\\
	8 & 546 & 29 & 25 & 29.12 & 0.04 & 505 & 102080 & 298.35 & 66.3 & 0.324\\
	9 & 546 & 31 & 25 & 30.86 & 0.06 & 564 & 102050 & 299.15 & 66.3 & 0.324\\
	10 & 546 & 33 & 25 & 32.98 & 0.05 & 641 & 102054 & 299.9 & 60.4 & 0.35\\
	\hline
\end{tabular}
\caption{Experimental measurements for station 1}
\label{table:station_1_measurements}
\end{center}
\end{table}
\renewcommand\baselinestretch{2}\selectfont

\renewcommand\baselinestretch{1.3}\selectfont
\begin{table}[H]
\begin{center}
\begin{tabular}{|ccccccccccc|}
	\hline
	Run & $I_Z$ & $V_{nom}$ & $dt$ & $V_{fs}$ & $\sigma_{V_{fs}}$ & $Q$ & $P_{atm}$ & $T_{tunnel}$ & $\phi$ & $\eta_P$\\
	$ID$ & $(in)$ & $(m/s)$ & $(\mu s)$ & $(m/s)$ & $(m/s)$ & $(Pa)$ & $(Pa)$ & $(\degree K)$ & $(\%)$ & $(\mu s)$\\
	\hline
	11 & 708 & 15 & 40 & 15.27 & 0.02 & 138 & 101185 & 296.05 & 69.8 & 0.312\\
	12 & 708 & 17 & 40 & 16.89 & 0.02 & 169 & 101218 & 296.55 & 69.8 & 0.312\\
	13 & 708 & 19 & 40 & 19.03 & 0.02 & 215 & 101219 & 296.55 & 69.8 & 0.312\\
	14 & 708 & 21 & 40 & 21.13 & 0.02 & 264 & 101186 & 296.85 & 66.3 & 0.329\\
	15 & 708 & 23 & 40 & 23.21 & 0.04 & 321 & 101150 & 297.85 & 66.8 & 0.329\\
	16 & 708 & 25 & 25 & 25.36 & 0.04 & 380 & 101120 & 297.45 & 71.7 & 0.301\\
	17 & 708 & 27 & 25 & 27.03 & 0.03 & 432 & 101120 & 297.75 & 70.1 & 0.306\\
	18 & 708 & 29 & 25 & 29.12 & 0.06 & 498 & 101106 & 298.55 & 73.3 & 0.297\\
	19 & 708 & 31 & 25 & 30.87 & 0.04 & 562 & 101109 & 298.95 & 73.3 & 0.297\\
	20 & 708 & 33 & 25 & 33.39 & 0.04 & 653 & 101101 & 299.65 & 73.3 & 0.297\\
	\hline
\end{tabular}
\caption{Experimental measurements for station 2}
\label{table:station_2_measurements}
\end{center}
\end{table}
\renewcommand\baselinestretch{2}\selectfont

\input{tables/station_3_measurements}
\renewcommand\baselinestretch{1.3}\selectfont
\begin{table}[H]
\begin{center}
\begin{tabular}{|ccccccccccc|}
	\hline
	Run & $I_Z$ & $V_{nom}$ & $dt$ & $V_{fs}$ & $\sigma_{V_{fs}}$ & $Q$ & $P_{atm}$ & $T_{tunnel}$ & $\phi$ & $\eta_P$\\
	$ID$ & $(in)$ & $(m/s)$ & $(\mu s)$ & $(m/s)$ & $(m/s)$ & $(Pa)$ & $(Pa)$ & $(\degree K)$ & $(\%)$ & $(\mu s)$\\
	\hline
	31 & 863 & 15 & 40 & 14.86 & 0.02 & 133 & 101865 & 295.75 & 63.8 & 0.354\\
	32 & 863 & 17 & 40 & 17.39 & 0.02 & 180 & 101855 & 295.95 & 63.8 & 0.354\\
	33 & 863 & 19 & 40 & 19.08 & 0.02 & 219 & 101847 & 296.1 & 63.8 & 0.354\\
	34 & 863 & 21 & 40 & 21.13 & 0.05 & 267 & 101845 & 296.15 & 63.8 & 0.354\\
	35 & 863 & 23 & 40 & 23.29 & 0.02 & 323 & 101844 & 296.45 & 63.8 & 0.354\\
	36 & 863 & 25 & 25 & 24.98 & 0.05 & 373 & 101840 & 296.65 & 65.6 & 0.344\\
	37 & 863 & 27 & 25 & 27.09 & 0.05 & 438 & 101843 & 297 & 65.6 & 0.344\\
	38 & 863 & 29 & 25 & 28.81 & 0.03 & 493 & 101848 & 297.55 & 65.6 & 0.344\\
	39 & 863 & 31 & 25 & 30.87 & 0.04 & 570 & 101842 & 298.15 & - & -\\
	40 & 863 & 33 & 25 & 33.44 & 0.06 & 661 & 101844 & 298.35 & - & -\\
	\hline
\end{tabular}
\caption{Experimental measurements for station 4}
\label{table:station_4_measurements}
\end{center}
\end{table}
\renewcommand\baselinestretch{2}\selectfont

\begin{table}[H]
\begin{center}
\begin{tabular}{|ccccccccccc|}
	\hline
	Run & $I_Z$ & $V_{nom}$ & $dt$ & $V_{fs}$ & $\sigma_{V_{fs}}$ & $Q$ & $P_{atm}$ & $T_{tunnel}$ & $\phi$ & $\eta_P$\\
	$ID$ & $(in)$ & $(m/s)$ & $(\mu s)$ & $(m/s)$ & $(m/s)$ & $(Pa)$ & $(Pa)$ & $(\degree K)$ & $(\%)$ & $(\mu s)$\\
	\hline
	41 & 914 & 15 & 40 & 14.88 & 0.02 & 132 & 101815 & 296.25 & 57.5 & 0.386\\
	42 & 914 & 17 & 40 & 17.24 & 0.03 & 180 & 101812 & 296.35 & 55.8 & 0.398\\
	43 & 914 & 19 & 40 & 19.08 & 0.03 & 217 & 101812 & 294.45 & 55.8 & 0.398\\
	44 & 914 & 21 & 40 & 21.18 & 0.03 & 267 & 101816 & 296.65 & 55.8 & 0.398\\
	45 & 914 & 23 & 40 & 23.24 & 0.03 & 323 & 101809 & 296.95 & 55.8 & 0.398\\
	46 & 914 & 25 & 25 & 24.9 & 0.03 & 371 & 101802 & 297.15 & 55.8 & 0.398\\
	47 & 914 & 27 & 25 & 27.08 & 0.04 & 435 & 101788 & 297.45 & 55.8 & 0.398\\
	48 & 914 & 29 & 25 & 29.19 & 0.03 & 506 & 101784 & 297.85 & 55.8 & 0.398\\
	49 & 914 & 31 & 25 & 31.28 & 0.05 & 584 & 101786 & 298.15 & 56.1 & 0.393\\
	50 & 914 & 33 & 25 & 33.05 & 0.05 & 645 & 101789 & 298.75 & 56.1 & 0.393\\
	\hline
\end{tabular}
\caption{Experimental conditions for tests at station 5}
\label{table:station_5_measurements}
\end{center}
\end{table}

\begin{table}[H]
\begin{center}
\begin{tabular}{|ccccccccccc|}
	\hline
	Run & $I_Z$ & $V_{nom}$ & $dt$ & $V_{fs}$ & $\sigma_{V_{fs}}$ & $Q$ & $P_{atm}$ & $T_{tunnel}$ & $\phi$ & $\eta_P$\\
	$ID$ & $(in)$ & $(m/s)$ & $(\mu s)$ & $(m/s)$ & $(m/s)$ & $(Pa)$ & $(Pa)$ & $(\degree K)$ & $(\%)$ & $(\mu s)$\\
	\hline
	51 & 965 & 15 & 40 & 15.31 & 0.02 & 140 & 102039 & 294.85 & 61.2 & 0.381\\
	52 & 965 & 17 & 40 & 17.36 & 0.03 & 182 & 102034 & 294.95 & 61.2 & 0.381\\
	53 & 965 & 19 & 40 & 19.08 & 0.03 & 219 & 102035 & 295.15 & 59.5 & 0.392\\
	54 & 965 & 21 & 40 & 21.23 & 0.03 & 270 & 102037 & 295.35 & 59.5 & 0.392\\
	55 & 965 & 23 & 40 & 23.33 & 0.03 & 327 & 102018 & 295.65 & 59.5 & 0.392\\
	56 & 965 & 25 & 25 & 24.97 & 0.03 & 375 & 102021 & 295.95 & 59.5 & 0.392\\
	57 & 965 & 27 & 25 & 27.05 & 0.06 & 437 & 102008 & 297.85 & 52.6 & 0.422\\
	58 & 965 & 29 & 25 & 29.17 & 0.04 & 505 & 102005 & 298.15 & 47.4 & 0.454\\
	59 & 965 & 31 & 25 & 30.9 & 0.06 & 571 & 102015 & 297.85 & 53.7 & 0.421\\
	60 & 965 & 33 & 25 & 33.04 & 0.06 & 653 & 102009 & 297.35 & 53.7 & 0.421\\
	\hline
\end{tabular}
\caption{Experimental measurements for station 6}
\label{table:station_6_measurements}
\end{center}
\end{table}

\renewcommand\baselinestretch{1.3}\selectfont
\begin{table}[H]
\begin{center}
\begin{tabular}{|ccccccccccc|}
	\hline
	Run & $I_Z$ & $V_{nom}$ & $dt$ & $V_{fs}$ & $\sigma_{V_{fs}}$ & $Q$ & $P_{atm}$ & $T_{tunnel}$ & $\phi$ & $\eta_P$\\
	$ID$ & $(in)$ & $(m/s)$ & $(\mu s)$ & $(m/s)$ & $(m/s)$ & $(Pa)$ & $(Pa)$ & $(\degree K)$ & $(\%)$ & $(\mu s)$\\
	\hline
	61 & 1016 & 15 & 40 & 15.31 & 0.03 & 140 & 101977 & 296.15 & 52 & 0.434\\
	62 & 1016 & 17 & 40 & 16.98 & 0.04 & 171 & 101969 & 296.25 & 52 & 0.434\\
	63 & 1016 & 19 & 40 & 19.1 & 0.03 & 218 & 101961 & 296.3 & 52 & 0.434\\
	64 & 1016 & 21 & 40 & 21.15 & 0.05 & 269 & 101950 & 296.45 & 49.4 & 0.45\\
	65 & 1016 & 23 & 25 & 23.23 & 0.05 & 321 & 101953 & 296.77 & 52.6 & 0.422\\
	66 & 1016 & 25 & 25 & 24.96 & 0.05 & 371 & 101951 & 297.07 & 52.6 & 0.422\\
	67 & 1016 & 27 & 25 & 27.05 & 0.5 & 436 & 101936 & 297.25 & 52.6 & 0.422\\
	68 & 1016 & 29 & 25 & 29.17 & 0.03 & 505 & 101928 & 297.76 & 52.6 & 0.422\\
	69 & 1016 & 31 & 25 & 31.31 & 0.05 & 581 & 101921 & 297.85 & 52.6 & 0.422\\
	70 & 1016 & 33 & 25 & 33.01 & 0.05 & 647 & 101922 & 298.3 & 52.6 & 0.422\\
	\hline
\end{tabular}
\caption{Experimental measurements for station 7}
\label{table:station_7_measurements}
\end{center}
\end{table}
\renewcommand\baselinestretch{2}\selectfont


\section{Stereo PIV Data processing}

Raw PIV data was processed into text files containing lists of vectors and 
positions from raw image data with commercial INSIGHT software. 
This format is considered to be the 
starting point for the present analysis. Each vector is recorded as a row in a 
"v3d" file, with a set of position coordinates,velocity 
vector components, and two quality control flags that were used to reduce 
spurious vector count. An example of this data is shown in Table 
\ref{table:v3d_row_example}.

\begin{table}[H]
\begin{center}
\begin{tabular}{|cccccccc|}
	\hline
	X mm & Y mm  & Z mm & U m/s & V m/s & W m/s & CHC & Residual Pixels\\
	\hline
	8.02046 & 0.174553 & 0 & 0.415872 & -2.25951 & 13.5873 & 1 & 0.127058\\
	9.74656 & 0.174553 & 0 & 0.386507 & -2.4523 & 13.9244 & 1 & 0.166965\\
	11.4727 & 0.174553 & 0 & 1.01919 & -2.8773 & 14.9454 & 1 & 0.0480147\\
	13.1988 & 0.174553 & 0 & 1.30872 & -3.02836 & 15.2081 & 1 & 0.0560525\\
	\hline
\end{tabular}
\caption{Example rows from v3d files with raw 3d vector data}
\label{table:v3d_row_example}
\end{center}
\end{table}


The position coordinates are expressed in units of millimeters from 
the fiducial mark on the target used for calibration. These coordinates are 
expressed from the viewpoint of the cameras, which were pointed upstream; 
positive $X$ coordinates are to the right, positive $Y$ is upwards, and 
positive $Z$ is towards the cameras. Vector components are recorded as $U$, 
$V$ and $W$. 

\subsection{Notation}
Since each run contained 200 separate vector sets, this data must be 
synthesized into 
values that compare well with the Reynolds Averaged Navier Stokes equations in 
cylindrical coordinates. This processing was performed in Python 2.7 and 
Matlab, with much of the code entirely replicated in each language. 
First, text files with tables of vector data were loaded into a parser to 
construct a mesh grid of the $X$, $Y$ coordinate space. These mesh grids 
establish the relationship between matrix indices and real coordinate 
space with units of $mm$.
Then, for each test as shown in Table \ref{table:test_matrix_table}, data from 
each of the 200 snapshots is Reynolds averaged to produce an average velocity 
component, and a fluctuating velocity component in each dimension for each 
vector, and expressed as equations \ref{eq:ubar} to \ref{eq:wprime}. 
Each component is given an individual variable name, with stable nonfluctuating 
components taken from an average of all sets expressed as capitol letters ($U$, 
$V$, $W$), and time averaged fluctuating components expressed as lower case 
letters ($u$, $v$, $w$). 
Velocities referring to just one of the 200 sets will use a 
subscript $i$ as ($U_i$, $V_i$, $W_i$). Each of these symbols represents an 
entire matrix of values across the interrogation plane. This is done primarily 
for simplicity and coherence between the theoretical basis, and the software 
implementation of 
the mathematics where special characters cannot be used in variable names.

\begin{equation}
U  = \frac{1}{N} \sum_{i=1}^{200} U_i
\label{eq:ubar}
\end{equation}

\begin{equation}
V  = \frac{1}{N} \sum_{i=1}^{200} V_i
\end{equation}

\begin{equation}
W  = \frac{1}{N} \sum_{i=1}^{200} W_i
\end{equation}

\begin{equation}
u = \frac{1}{N} \sum_{i=1}^{200} |U_i - U|
\end{equation}

\begin{equation}
v = \frac{1}{N} \sum_{i=1}^{200} |V_i - V|
\end{equation}

\begin{equation}
w = \frac{1}{N} \sum_{i=1}^{200} |W_i - W|
\label{eq:wprime}
\end{equation}

Where $U$, $V$ and $W$ are the time averaged velocity 
components in the $X$, $Y$, $Z$ directions respectively, and  $u$, 
$v$ and $w$ are the time averaged fluctuating velocity components.
Grid points with fewer than 20 measurements ($i < 20$) were assumed to be no 
data and thrown out. These fluctuating components can represent turbulent 
energies as $uu$, $vv$, and $ww$, and Reynolds stresses as $uv$, $vw$, and 
$uw$. Turbulent kinetic energy $k$ can be calculated as one half 
of the sum of the three components as in equation \ref{eq:tke}.

\begin{equation}
k = \frac{1}{2} \left(u^2 + v^2 + w^2\right)
\label{eq:tke}
\end{equation}


Once all the statistics for Cartesian coordinates have been generated, the 
vortex core must be found in order to convert all values to cylindrical 
coordinates. The core is found by finding the minimum in-plane velocity 
magnitudes near the center of the interrogation plane, excluding the stream 
wise $W$ component. In practice, to avoid confusion in identifying the vortex 
core due to spurious edge values, a threshold value is defined that limits the 
search for in-plane velocity minima to an area near the center of the field of 
view. Once the lowest value is found, sub-pixel accuracy is achieved by taking 
a cubic interpolation of the grid points surrounding the minimum, and resolving 
on the finer mesh grid. 

With a location for the vortex core, new mesh grids in radial and tangential 
coordinates, script $\mathcal{R}$ and $\theta$, are created from the $X$ and 
$Y$ mesh grids. velocity components in the $r$ and $t$ directions are then 
calculated by equations \ref{eq:uv_r} and \ref{eq:uv_t}.

\begin{equation}
R_i = u_i \cos{(\theta)} + v_i \sin{(\theta)}
\label{eq:uv_r}
\end{equation}

\begin{equation}
T_i = u_i \sin{(\theta)} - v_i \cos{(\theta)}
\label{eq:uv_t}
\end{equation}

Where $R_i$ and $T_i$ represent a radial and tangential velocity matrix for 
just one of the 200 total surveys, and $\theta$ is the mesh grid of angles 
about the vortex core. These equations are applied on an element wise basis to 
every $j,k$ grid point in the vector field. Once this is done, the same formula 
is applied to separate the radial and tangential velocity components $r$ and 
$t$ into stable and fluctuating components in equations \ref{eq:rbar} through 
\ref{eq:rbar}.

\begin{equation}
R  = \frac{1}{N} \sum_{i=1}^N R_i
\label{eq:rbar}
\end{equation}

\begin{equation}
T  = \frac{1}{N} \sum_{i=1}^N T_i
\label{eq:tbar}
\end{equation}

\begin{equation}
r = \frac{1}{N} \sum_{i=1}^N |R_i - R|
\end{equation}

\begin{equation}
t = \frac{1}{N} \sum_{i=1}^N |T_i - T|
\end{equation}

Where $R$ and $T$ are the time averaged velocity 
components in the $\mathcal{R}$ and $\theta$ directions respectively, and  $r$
and $t$ are the time averaged fluctuating velocity components.

Once this processing is complete, we have large set of reynolds averaged 
velocity data available for each of the 70 test cases. Plots of this data from 
test case 55 will be shown as examples. Stream plots can be produced as in 
figure \ref{fig:examp_stream}. Cartesian average velocity components for test 
case is shown in figures \ref{fig:examp_U} through \ref{fig:examp_W}, though 
they are not particularly useful when compared to values in cylindrical 
coordinates. 
Cylindrical average velocity components are shown in figures \ref{fig:examp_R} 
and \ref{fig:examp_T}. Reynolds stresses are shown in figures 
\ref{fig:examp_rt} through \ref{fig:examp_tw}. Turbulent energies are shown in 
figures \ref{fig:examp_rr} through \ref{fig:examp_ww}. Total turbulent kinetic 
energy $k$ is shown in figure \ref{fig:examp_tke}. Finally, scatter plots that 
flatten the dataset into one dimension can be generated as in the tangential 
velocity vs distance to core plot shown in figure \ref{fig:examp_Tscatter}.

\begin{figure}[H]
	\centering
	\includegraphics[width=5in]{figs/example_vortex_figs/example_stream}
	\caption{Example stream plot of run ID 55.}
	\label{fig:examp_stream}
\end{figure}

\begin{figure}[H]
	\centering
	\includegraphics[width=5in]{figs/example_vortex_figs/example_U_contour}
\caption{Example contour plot of $U$ for run ID 55.}
\label{fig:examp_U}
\end{figure}

\begin{figure}[H]
	\centering
	\includegraphics[width=5in]{figs/example_vortex_figs/example_V_contour}
\caption{Example contour plot of $V$ for run ID 55.}
\label{fig:examp_V}
\end{figure}

\begin{figure}[H]
	\centering
	\includegraphics[width=5in]{figs/example_vortex_figs/example_W_contour}
\caption{Example contour plot of $W$ for run ID 55.}
\label{fig:examp_W}
\end{figure}

\begin{figure}[H]
	\centering
	\includegraphics[width=5in]{figs/example_vortex_figs/example_R_contour}
\caption{Example contour plot of $R$ for run ID 55.}
\label{fig:examp_R}
\end{figure}

\begin{figure}[H]
	\centering
	\includegraphics[width=5in]{figs/example_vortex_figs/example_T_contour}
\caption{Example contour plot of $T$ for run ID 55.}
\label{fig:examp_T}
\end{figure}

\begin{figure}[H]
	\centering
	\includegraphics[width=5in]{figs/example_vortex_figs/example_rt_contour}
\caption{Example contour plot of $rt$ for run ID 55.}
\label{fig:examp_rt}
\end{figure}

\begin{figure}[H]
	\centering
	\includegraphics[width=5in]{figs/example_vortex_figs/example_rw_contour}
\caption{Example contour plot of $rw$ for run ID 55.}
\label{fig:examp_rw}
\end{figure}

\begin{figure}[H]
	\centering
	\includegraphics[width=5in]{figs/example_vortex_figs/example_tw_contour}
\caption{Example contour plot of $tw$ for run ID 55.}
\label{fig:examp_tw}
\end{figure}

\begin{figure}[H]
	\centering
	\includegraphics[width=5in]{figs/example_vortex_figs/example_rr_contour}
\caption{Example contour plot of $rr$ for run ID 55.}
\label{fig:examp_rr}
\end{figure}

\begin{figure}[H]
	\centering
	\includegraphics[width=5in]{figs/example_vortex_figs/example_tt_contour}
\caption{Example contour plot of $tt$ for run ID 55.}
\label{fig:examp_tt}
\end{figure}

\begin{figure}[H]
	\centering
	\includegraphics[width=5in]{figs/example_vortex_figs/example_ww_contour}
\caption{Example contour plot of $ww$ for run ID 55.}
\label{fig:examp_ww}
\end{figure}

\begin{figure}[H]
	\centering
	\includegraphics[width=5in]{figs/example_vortex_figs/example_ctke_contour}
\caption{Example contour plot of turbulent kinetic energy $k$ for run ID 55.}
\label{fig:examp_tke}
\end{figure}

\begin{figure}[H]
	\centering
	\includegraphics[width=7in]{figs/example_vortex_figs/example_TscatterTKE}
\caption{Example scatter plot of $T$, colored by $k$ for run ID 55.}
\label{fig:examp_Tscatter}
\end{figure}

\section{Uncertainty in PIV measurements}
 
There are a number of factors that contribute to uncertainty in PIV 
measurements. Both bias and precision errors can be estimated by considering 
detailed information about the optical geometry of the PIV setup. Monte Carlo 
based error estimation techniques can be applied by creating artificially 
simulated images with randomly distributed particles \cite{adeyinka2005}. 
The distribution of these particles can be modeled using a Gaussian intensity  
profile by (flag, reference) as described in Equation 
\ref{eq:piv_gaussian_uncertainty}.

\begin{equation}
	I(x,y) = I_0exp \left( \frac{-(x_{img} - x_p)^2 - (y_{img} - y_p)^2}
	{\frac{1}{8}d_\tau^2} \right)
	\label{eq:piv_gaussian_uncertainty}
\end{equation}
%
Where $x_p$ and $y_p$ are the locations of the particle centroid, $d_\tau$ is 
the diameter of of the particle, and $I_0$ is particle intensity. Particle 
intensity is directly related to the intensity of the light sheet, which is 
modeled as a Gaussian distribution \cite{PIVuncertAIAA}. This assumption allows 
us 
to express particle intensity as \ref{eq:particle_intensity_gaus}

\begin{equation}
	I_0(z_p) = (q)exp\left(- \frac{z_p^2}{\frac{1}{8}\Delta Z_L^2}\right)
	\label{eq:particle_intensity_gaus}
\end{equation}
%
Where $z_p$ is the particles position within the thickness of the light sheet, 
$q$ is the particle light scattering efficiency, and $\Delta Z_L$ is the 
thickness of the light sheet.

These formula are used to generate artificial image pairs for a single camera. 
A sufficient number of particles are created with $x$, $y$ and $z$ coordinates 
to meet particle density parameters, these coordinates are then 
used to generate light intensities according to Equation 
\ref{eq:particle_intensity_gaus}, which populate the image plane of the first 
image $A$. Next, a displacement image, $B$, is generated by shifting all the 
particles in a predetermined direction in three dimensional space. It is worth 
noting that for a single camera setup, particle movements in the $z$ direction 
do not produce pixel displacements, but simply determine the intensity of the 
light reflected from the particle. The known particle displacements can then be 
compared against outputs calculated with PIV capture and processing software.

To translate this concept to stereo PIV, an additional step is required. 
Instead of directly placing particles with known coordinates onto the image 
plane of one camera, they are placed on a conceptual version of the 
interrogation plane. The coordinate transforms obtained from PIV calibration 
are used to map the displacements from the conceptual plane into the image 
plane of each camera. These coordinate transforms are unique to each camera, 
and depend upon the optical geometry of the PIV setup. Uncertainty is 
calculated using the recommended AIAA calibration procedure outlined in 
\cite{PIVuncertAIAA}. To determine the system bias, the mean difference between 
the velocity standard established by the Monte Carlo simulation and the 
velocity calculated by the PIV software are compared as follows

\begin{equation}
\overline{\Delta U} = \frac{1}{N} \left(\sum_{i=1}^N \Delta U_i \right),
\label{eq:Uerror}
\end{equation}

\begin{equation}
\overline{\Delta V} = \frac{1}{N} \left(\sum_{i=1}^N \Delta V_i \right),
\label{eq:Verror}
\end{equation}

\begin{equation}
\overline{\Delta W} = \frac{1}{N} \left(\sum_{i=1}^N \Delta W_i \right)
\label{eq:Werror}
\end{equation}
%
Which is simply the average difference between the known velocity components 
and the measured velocity components $\Delta U$, $\Delta V$, and $\Delta W$ for 
a large number of simulations. This is referred to as the bias, and the three 
bias components are denoted as

\begin{equation}
\beta_{U} = \overline{\Delta U}
\label{eq:Ubias}
\end{equation}
\begin{equation}
\beta_{V} = \overline{\Delta V}
\label{eq:Vbias}
\end{equation}
\begin{equation}
\beta_{W} = \overline{\Delta W}
\label{eq:Wbias}
\end{equation}

The measurement precision is reported as the root-mean-square of the  
standard deviation, calculated as in 
	
\begin{equation}
S_{\Delta U} = \sqrt{\frac{1}{N-1} \left(\sum_{i=1}^N (\Delta U_i - 
\overline{\Delta U})^2 \right)}
\label{eq:Usd}
\end{equation}

\begin{equation}
S_{\Delta V} = \sqrt{\frac{1}{N-1} \left(\sum_{i=1}^N (\Delta V_i - 
	\overline{\Delta V})^2 \right)}
\label{eq:Vsd}
\end{equation}

\begin{equation}
S_{\Delta W} = \sqrt{\frac{1}{N-1} \left(\sum_{i=1}^N (\Delta W_i - 
	\overline{\Delta W})^2 \right)}
\label{eq:Wsd}
\end{equation}
%
Resulting in precision calculations given by 
%	
\begin{equation}
P_{\overline{U}} = \frac{2 S_{\Delta U}}{\sqrt{N}}
\label{eq:Uprec}
\end{equation}

\begin{equation}
P_{\overline{V}} = \frac{2 S_{\Delta V}}{\sqrt{N}}
\label{eq:Vprec}
\end{equation}

\begin{equation}
P_{\overline{W}} = \frac{2 S_{\Delta W}}{\sqrt{N}}
\label{eq:Wprec}
\end{equation}

Total uncertainty for each component at the 95\% confidence level is calculated 
by combining the bias and precision via to obtain

\begin{equation}
U_{\overline{\Delta U}} = \sqrt{\beta_{U}^2 + P_{\overline{U}}^2}
\label{eq:Uuncert}
\end{equation}
\begin{equation}
U_{\overline{\Delta V}} = \sqrt{\beta_{V}^2 + P_{\overline{V}}^2}
\label{eq:Vuncert}
\end{equation}
\begin{equation}
U_{\overline{\Delta W}} = \sqrt{\beta_{W}^2 + P_{\overline{W}}^2}
\label{eq:Wuncert}
\end{equation}


For these experiments, uncertainty analysis was conducted after the 
experimental data was taken. Vortices were characterized by velocities at key 
locations that allow each vortex to be described by one of the common vortex 
models. Characterization velocities of particular interest include the maximum 
tangential velocity about the vortex core, the distance of this high tangential 
velocity region from the core, which defined the core radius, and the minimum 
axial velocity observed in the vortex core. Understanding the uncertainty of 
these measurements required a Monte Carlo approach from synthetically created 
particle imagery. Artificial pixel displacements were specified to approximate 
the typical displacement associated with the velocities of greatest interest. 

To add complexity, the time between frame captures, $dt$ was expected to have a 
significant impact on uncertainty. Since the range of velocities used in this 
study required the use of multiple values of $dt$, artificial images were 
generated for a scenario at each value of $dt$ and the associated velocities. 
With the exception of measurements at station one, the furthest upstream 
observations at $I_Z$ of 546$mm$ down stream, all tests were conducted with a 
$dt$ value of 25, or 40 $\mu s$. At station one $dt$ values of 35 were also 
used. During the experimentation period, great difficulty 
was encountered in tuning PIV parameters to achieve well resolved vector fields 
at station one. The uncertainty analysis will demonstrate that the geometry of 
viewing angles was also unfavorable at this station, and the quality of 
measurements taken this far upstream was poor. Measurements taken at station 
one were largely discarded.

In the stream wise direction, the velocity of interest is that of the core, 
which was reported as lower than free stream. (flag)

Even though many samples were taken at every vector location, the uncertainty 
in each individual measurement was of great importance for studying the 
unstable component of the velocity, and thus turbulent phenomena. Uncertainty 
in the fluctuating component were best represented by using an $N$ 
value of one in precision Equations \ref{eq:Uprec} through \ref{eq:Wprec}. 
Uncertainty in the stable component was lower, since this measurement is a 
result of averaging many measurements, and was calculated by using an $N$ value 
of 200 in the precision Equations \ref{eq:Uprec} through \ref{eq:Wprec}.


The Monte Carlo analysis shows....

(flag, unfinished, need to run new Monte Carlo code)
%=========================================================================

% experiment results and sub documents
\chapter{Experiment Results}
words here

%=========================================================================
% constructs the bibliography
\bibliographystyle{apalike}
\bibliography{bib/vortices,bib/other,bib/particles,bib/piv}

\begin{thebibliography}{99}
%This command adds some extra space in the table of contents
\addtocontents{toc}{\vspace*{12pt}}  

% adds an entry for the bibliography in the table of contents
\addcontentsline{toc}{chapter}{BIBLIOGRAPHY}  
\end{thebibliography}
%=========================================================================

\appendix
%
\chapter{PIV Data}

\section{Station 1 PIV Data: 546mm Downstream}
\renewcommand\baselinestretch{1.3}\selectfont
\begin{table}[H]
\begin{center}
\begin{tabular}{|ccccccccccc|}
	\hline
	Run & $I_Z$ & $V_{nom}$ & $dt$ & $V_{fs}$ & $\sigma_{V_{fs}}$ & $Q$ & $P_{atm}$ & $T_{tunnel}$ & $\phi$ & $\eta_P$\\
	$ID$ & $(mm)$ & $(m/s)$ & $(\mu s)$ & $(m/s)$ & $(m/s)$ & $(Pa)$ & $(Pa)$ & $(\degree K)$ & $(\%)$ & $(\mu s)$\\
	\hline
	1 & 546 & 15 & 50 & 15.22 & 0.02 & 135 & 102036 & 299.85 & 60.4 & 0.35\\
	2 & 546 & 17 & 50 & 16.88 & 0.02 & 170 & 102115 & 297.55 & 66.3 & 0.329\\
	3 & 546 & 19 & 50 & 19.43 & 0.01 & 225 & 102105 & 297.55 & 66.3 & 0.329\\
	4 & 546 & 21 & 50 & 21.06 & 0.02 & 264 & 102100 & 297.75 & 66.3 & 0.324\\
	5 & 546 & 23 & 35 & 23.21 & 0.04 & 321 & 102097 & 297.95 & 66.3 & 0.324\\
	6 & 546 & 25 & 35 & 24.86 & 0.05 & 371 & 102093 & 298.15 & 66.3 & 0.324\\
	7 & 546 & 27 & 35 & 27.02 & 0.03 & 434 & 102092 & 298.3 & 66.3 & 0.324\\
	8 & 546 & 29 & 25 & 29.12 & 0.04 & 505 & 102080 & 298.35 & 66.3 & 0.324\\
	9 & 546 & 31 & 25 & 30.86 & 0.06 & 564 & 102050 & 299.15 & 66.3 & 0.324\\
	10 & 546 & 33 & 25 & 32.98 & 0.05 & 641 & 102054 & 299.9 & 60.4 & 0.35\\
	\hline
\end{tabular}
\caption{Experimental measurements for station 1}
\label{table:station_1_measurements}
\end{center}
\end{table}
\renewcommand\baselinestretch{2}\selectfont

\input{figs/tex/run_1}
\input{figs/tex/run_2}
\input{figs/tex/run_3}
\input{figs/tex/run_4}
\input{figs/tex/run_5}
\input{figs/tex/run_6}
\input{figs/tex/run_7}
\input{figs/tex/run_8}
\begin{figure}[H]
\centering
\includegraphics[width=6in]{figs/run_9/run_9_comparison}
\caption{Theoretical vortex profile fits to experimental data at $z/c$=5.37, $V_{free}$=30.86, station 1.}
\end{figure}


\begin{figure}[H]
\centering
\includegraphics[width=4.25in]{figs/run_9/run_9_stream}
\caption{Stream plot at $z/c$=5.37, $V_{free}$=30.86, station 1.}
\end{figure}


\begin{figure}[H]
\centering
\includegraphics[width=4.25in]{figs/run_9/run_9_R_contour}
\caption{Contour plot of $\overline{v_{r}}$ at $z/c$=5.37, $V_{free}$=30.86, station 1.}
\end{figure}


\begin{figure}[H]
\centering
\includegraphics[width=4.25in]{figs/run_9/run_9_T_contour}
\caption{Contour plot of $\overline{v_{\theta}}$ at $z/c$=5.37, $V_{free}$=30.86, station 1.}
\end{figure}


\begin{figure}[H]
\centering
\includegraphics[width=4.25in]{figs/run_9/run_9_W_contour}
\caption{Contour plot of $\overline{v_{z}}$ at $z/c$=5.37, $V_{free}$=30.86, station 1.}
\end{figure}


\begin{figure}[H]
\centering
\includegraphics[width=4.25in]{figs/run_9/run_9_rt_contour}
\caption{Contour plot of $\overline{v_{r}^{\prime} v_{\theta}^{\prime}}$ at $z/c$=5.37, $V_{free}$=30.86, station 1.}
\end{figure}


\begin{figure}[H]
\centering
\includegraphics[width=4.25in]{figs/run_9/run_9_rw_contour}
\caption{Contour plot of $\overline{v_{r}^{\prime} v_{z}^{\prime}}$ at $z/c$=5.37, $V_{free}$=30.86, station 1.}
\end{figure}


\begin{figure}[H]
\centering
\includegraphics[width=4.25in]{figs/run_9/run_9_tw_contour}
\caption{Contour plot of $\overline{v_{\theta}^{\prime} v_{z}^{\prime}}$ at $z/c$=5.37, $V_{free}$=30.86, station 1.}
\end{figure}


\begin{figure}[H]
\centering
\includegraphics[width=4.25in]{figs/run_9/run_9_rr_contour}
\caption{Contour plot of $\overline{v_{r}^{\prime} v_{r}^{\prime}}$ at $z/c$=5.37, $V_{free}$=30.86, station 1.}
\end{figure}


\begin{figure}[H]
\centering
\includegraphics[width=4.25in]{figs/run_9/run_9_tt_contour}
\caption{Contour plot of $\overline{v_{\theta}^{\prime} v_{\theta}^{\prime}}$ at $z/c$=5.37, $V_{free}$=30.86, station 1.}
\end{figure}


\begin{figure}[H]
\centering
\includegraphics[width=4.25in]{figs/run_9/run_9_ww_contour}
\caption{Contour plot of $\overline{v_{z}^{\prime} v_{z}^{\prime}}$ at $z/c$=5.37, $V_{free}$=30.86, station 1.}
\end{figure}


\begin{figure}[H]
\centering
\includegraphics[width=4.25in]{figs/run_9/run_9_ctke_contour}
\caption{Contour plot of $k$ at $z/c$=5.37, $V_{free}$=30.86, station 1.}
\end{figure}


\begin{figure}[H]
\centering
\includegraphics[width=4.25in]{figs/run_9/run_9_num_contour}
\caption{Contour plot of $N$ at $z/c$=5.37, $V_{free}$=30.86, station 1.}
\end{figure}


\begin{figure}[H]
\centering
\includegraphics[width=6in]{figs/run_9/run_9_T_vs_r_meshscatter}
\caption{Scatter plot of azimuthal velocity vs radius at $z/c$=5.37, $V_{free}$=30.86, station 1.}
\end{figure}


\begin{figure}[H]
\centering
\includegraphics[width=6in]{figs/run_9/run_9_T_vs_r_mesh_ctkescatter}
\caption{Scatter plot of azimuthal velocity vs radius, turbulent kinetic energy, at $z/c$=5.37, $V_{free}$=30.86, station 1.}
\end{figure}


\begin{figure}[H]
\centering
\includegraphics[width=6in]{figs/run_9/run_9_ctke_vs_r_meshscatter}
\caption{Scatter plot of turbulent kinetic energy vs radius at $z/c$=5.37, $V_{free}$=30.86, station 1.}
\end{figure}


\begin{figure}[H]
\centering
\includegraphics[width=6in]{figs/run_9/run_9_turb_visc_reynolds_vs_r_meshscatter}
\caption{Scatter plot of $
u_T$ reynolds stress term vs radius at $z/c$=5.37, $V_{free}$=30.86, station 1.}
\end{figure}


\begin{figure}[H]
\centering
\includegraphics[width=6in]{figs/run_9/run_9_turb_visc_vel_grad_vs_r_meshscatter}
\caption{Scatter plot of $\nu_T$ velocity gradient term vs radius at $z/c$=5.37, $V_{free}$=30.86, station 1.}
\end{figure}


\begin{figure}[H]
\centering
\includegraphics[width=6in]{figs/run_9/run_9_turb_visc_ettap_vs_r_meshscatter}
\caption{Scatter plot of $\nu_T$ pressure relaxation term vs radius at $z/c$=5.37, $V_{free}$=30.86, station 1.}
\end{figure}


\begin{figure}[H]
\centering
\includegraphics[width=6in]{figs/run_9/run_9_turb_visc_total_vs_r_meshscatter}
\caption{Scatter plot of non-equilibrium based $\nu_T$ vs radius at $z/c$=5.37, $V_{free}$=30.86, station 1.}
\end{figure}


\begin{figure}[H]
\centering
\includegraphics[width=6in]{figs/run_9/run_9_turb_visc_vs_r_meshscatter}
\caption{Scatter plot of $\nu_T$ vs radius as calculated from the turbulent viscosity hypothesis. $z/c$=5.37, $V_{free}$=30.86, station 1.}
\end{figure}


\begin{figure}[H]
\centering
\includegraphics[width=4.25in]{figs/run_9/run_9_dPdr_contour}
\caption{Diverging contour plot of $\frac{d\bar{P}}{dr}$ from non-equilibrium theory. $z/c$=5.37, $V_{free}$=30.86, station 1.}
\end{figure}



\begin{figure}[H]
\centering
\includegraphics[width=6in]{figs/run_10/run_10_comparison}
\caption{Theoretical vortex profile fits to experimental data at $z/c$=5.37, $V_{free}$=32.98, station 1.}
\end{figure}


\begin{figure}[H]
\centering
\includegraphics[width=4.25in]{figs/run_10/run_10_stream}
\caption{Stream plot at $z/c$=5.37, $V_{free}$=32.98, station 1.}
\end{figure}


\begin{figure}[H]
\centering
\includegraphics[width=4.25in]{figs/run_10/run_10_R_contour}
\caption{Contour plot of $\overline{v_{r}}$ at $z/c$=5.37, $V_{free}$=32.98, station 1.}
\end{figure}


\begin{figure}[H]
\centering
\includegraphics[width=4.25in]{figs/run_10/run_10_T_contour}
\caption{Contour plot of $\overline{v_{\theta}}$ at $z/c$=5.37, $V_{free}$=32.98, station 1.}
\end{figure}


\begin{figure}[H]
\centering
\includegraphics[width=4.25in]{figs/run_10/run_10_W_contour}
\caption{Contour plot of $\overline{v_{z}}$ at $z/c$=5.37, $V_{free}$=32.98, station 1.}
\end{figure}


\begin{figure}[H]
\centering
\includegraphics[width=4.25in]{figs/run_10/run_10_rt_contour}
\caption{Contour plot of $\overline{v_{r}^{\prime} v_{\theta}^{\prime}}$ at $z/c$=5.37, $V_{free}$=32.98, station 1.}
\end{figure}


\begin{figure}[H]
\centering
\includegraphics[width=4.25in]{figs/run_10/run_10_rw_contour}
\caption{Contour plot of $\overline{v_{r}^{\prime} v_{z}^{\prime}}$ at $z/c$=5.37, $V_{free}$=32.98, station 1.}
\end{figure}


\begin{figure}[H]
\centering
\includegraphics[width=4.25in]{figs/run_10/run_10_tw_contour}
\caption{Contour plot of $\overline{v_{\theta}^{\prime} v_{z}^{\prime}}$ at $z/c$=5.37, $V_{free}$=32.98, station 1.}
\end{figure}


\begin{figure}[H]
\centering
\includegraphics[width=4.25in]{figs/run_10/run_10_rr_contour}
\caption{Contour plot of $\overline{v_{r}^{\prime} v_{r}^{\prime}}$ at $z/c$=5.37, $V_{free}$=32.98, station 1.}
\end{figure}


\begin{figure}[H]
\centering
\includegraphics[width=4.25in]{figs/run_10/run_10_tt_contour}
\caption{Contour plot of $\overline{v_{\theta}^{\prime} v_{\theta}^{\prime}}$ at $z/c$=5.37, $V_{free}$=32.98, station 1.}
\end{figure}


\begin{figure}[H]
\centering
\includegraphics[width=4.25in]{figs/run_10/run_10_ww_contour}
\caption{Contour plot of $\overline{v_{z}^{\prime} v_{z}^{\prime}}$ at $z/c$=5.37, $V_{free}$=32.98, station 1.}
\end{figure}


\begin{figure}[H]
\centering
\includegraphics[width=4.25in]{figs/run_10/run_10_ctke_contour}
\caption{Contour plot of $k$ at $z/c$=5.37, $V_{free}$=32.98, station 1.}
\end{figure}


\begin{figure}[H]
\centering
\includegraphics[width=4.25in]{figs/run_10/run_10_num_contour}
\caption{Contour plot of $N$ at $z/c$=5.37, $V_{free}$=32.98, station 1.}
\end{figure}


\begin{figure}[H]
\centering
\includegraphics[width=6in]{figs/run_10/run_10_T_vs_r_meshscatter}
\caption{Scatter plot of azimuthal velocity vs radius at $z/c$=5.37, $V_{free}$=32.98, station 1.}
\end{figure}


\begin{figure}[H]
\centering
\includegraphics[width=6in]{figs/run_10/run_10_T_vs_r_mesh_ctkescatter}
\caption{Scatter plot of azimuthal velocity vs radius, turbulent kinetic energy, at $z/c$=5.37, $V_{free}$=32.98, station 1.}
\end{figure}


\begin{figure}[H]
\centering
\includegraphics[width=6in]{figs/run_10/run_10_ctke_vs_r_meshscatter}
\caption{Scatter plot of turbulent kinetic energy vs radius at $z/c$=5.37, $V_{free}$=32.98, station 1.}
\end{figure}


\begin{figure}[H]
\centering
\includegraphics[width=6in]{figs/run_10/run_10_turb_visc_reynolds_vs_r_meshscatter}
\caption{Scatter plot of $
u_T$ reynolds stress term vs radius at $z/c$=5.37, $V_{free}$=32.98, station 1.}
\end{figure}


\begin{figure}[H]
\centering
\includegraphics[width=6in]{figs/run_10/run_10_turb_visc_vel_grad_vs_r_meshscatter}
\caption{Scatter plot of $\nu_T$ velocity gradient term vs radius at $z/c$=5.37, $V_{free}$=32.98, station 1.}
\end{figure}


\begin{figure}[H]
\centering
\includegraphics[width=6in]{figs/run_10/run_10_turb_visc_by_vtheta_vs_r_meshscatter}
\caption{Scatter plot of mean total viscosity $\nu$ vs radius at $z/c$=5.37, $V_{free}$=32.98, station 1.}
\end{figure}


\begin{figure}[H]
\centering
\includegraphics[width=6in]{figs/run_10/run_10_momentum_vel_grad_vs_r_meshscatter}
\caption{Scatter plot of the momentum based velocity gradient term vs radius at $z/c$=5.37, $V_{free}$=32.98, station 1.}
\end{figure}


\begin{figure}[H]
\centering
\includegraphics[width=6in]{figs/run_10/run_10_turb_visc_ettap_vs_r_meshscatter}
\caption{Scatter plot of $\nu_T$ pressure relaxation term vs radius at $z/c$=5.37, $V_{free}$=32.98, station 1.}
\end{figure}


\begin{figure}[H]
\centering
\includegraphics[width=6in]{figs/run_10/run_10_turb_visc_total_vs_r_meshscatter}
\caption{Scatter plot of non-equilibrium based $\nu_T$ vs radius at $z/c$=5.37, $V_{free}$=32.98, station 1.}
\end{figure}


\begin{figure}[H]
\centering
\includegraphics[width=6in]{figs/run_10/run_10_turb_visc_vs_r_meshscatter}
\caption{Scatter plot of $\nu_T$ vs radius as calculated from the turbulent viscosity hypothesis. $z/c$=5.37, $V_{free}$=32.98, station 1.}
\end{figure}




\section{Station 2 PIV Data: 708mm Downstream}
\renewcommand\baselinestretch{1.3}\selectfont
\begin{table}[H]
\begin{center}
\begin{tabular}{|ccccccccccc|}
	\hline
	Run & $I_Z$ & $V_{nom}$ & $dt$ & $V_{fs}$ & $\sigma_{V_{fs}}$ & $Q$ & $P_{atm}$ & $T_{tunnel}$ & $\phi$ & $\eta_P$\\
	$ID$ & $(in)$ & $(m/s)$ & $(\mu s)$ & $(m/s)$ & $(m/s)$ & $(Pa)$ & $(Pa)$ & $(\degree K)$ & $(\%)$ & $(\mu s)$\\
	\hline
	11 & 708 & 15 & 40 & 15.27 & 0.02 & 138 & 101185 & 296.05 & 69.8 & 0.312\\
	12 & 708 & 17 & 40 & 16.89 & 0.02 & 169 & 101218 & 296.55 & 69.8 & 0.312\\
	13 & 708 & 19 & 40 & 19.03 & 0.02 & 215 & 101219 & 296.55 & 69.8 & 0.312\\
	14 & 708 & 21 & 40 & 21.13 & 0.02 & 264 & 101186 & 296.85 & 66.3 & 0.329\\
	15 & 708 & 23 & 40 & 23.21 & 0.04 & 321 & 101150 & 297.85 & 66.8 & 0.329\\
	16 & 708 & 25 & 25 & 25.36 & 0.04 & 380 & 101120 & 297.45 & 71.7 & 0.301\\
	17 & 708 & 27 & 25 & 27.03 & 0.03 & 432 & 101120 & 297.75 & 70.1 & 0.306\\
	18 & 708 & 29 & 25 & 29.12 & 0.06 & 498 & 101106 & 298.55 & 73.3 & 0.297\\
	19 & 708 & 31 & 25 & 30.87 & 0.04 & 562 & 101109 & 298.95 & 73.3 & 0.297\\
	20 & 708 & 33 & 25 & 33.39 & 0.04 & 653 & 101101 & 299.65 & 73.3 & 0.297\\
	\hline
\end{tabular}
\caption{Experimental measurements for station 2}
\label{table:station_2_measurements}
\end{center}
\end{table}
\renewcommand\baselinestretch{2}\selectfont

\begin{figure}[H]
\centering
\includegraphics[width=6in]{figs/run_11/run_11_comparison}
\caption{Theoretical vortex profile fits to experimental data at $z/c$=6.97, $V_{free}$=15.27, station 2.}
\end{figure}


\begin{figure}[H]
\centering
\includegraphics[width=4.25in]{figs/run_11/run_11_stream}
\caption{Stream plot at $z/c$=6.97, $V_{free}$=15.27, station 2.}
\end{figure}


\begin{figure}[H]
\centering
\includegraphics[width=4.25in]{figs/run_11/run_11_R_contour}
\caption{Contour plot of $\overline{v_{r}}$ at $z/c$=6.97, $V_{free}$=15.27, station 2.}
\end{figure}


\begin{figure}[H]
\centering
\includegraphics[width=4.25in]{figs/run_11/run_11_T_contour}
\caption{Contour plot of $\overline{v_{\theta}}$ at $z/c$=6.97, $V_{free}$=15.27, station 2.}
\end{figure}


\begin{figure}[H]
\centering
\includegraphics[width=4.25in]{figs/run_11/run_11_W_contour}
\caption{Contour plot of $\overline{v_{z}}$ at $z/c$=6.97, $V_{free}$=15.27, station 2.}
\end{figure}


\begin{figure}[H]
\centering
\includegraphics[width=4.25in]{figs/run_11/run_11_rt_contour}
\caption{Contour plot of $\overline{v_{r}^{\prime} v_{\theta}^{\prime}}$ at $z/c$=6.97, $V_{free}$=15.27, station 2.}
\end{figure}


\begin{figure}[H]
\centering
\includegraphics[width=4.25in]{figs/run_11/run_11_rw_contour}
\caption{Contour plot of $\overline{v_{r}^{\prime} v_{z}^{\prime}}$ at $z/c$=6.97, $V_{free}$=15.27, station 2.}
\end{figure}


\begin{figure}[H]
\centering
\includegraphics[width=4.25in]{figs/run_11/run_11_tw_contour}
\caption{Contour plot of $\overline{v_{\theta}^{\prime} v_{z}^{\prime}}$ at $z/c$=6.97, $V_{free}$=15.27, station 2.}
\end{figure}


\begin{figure}[H]
\centering
\includegraphics[width=4.25in]{figs/run_11/run_11_rr_contour}
\caption{Contour plot of $\overline{v_{r}^{\prime} v_{r}^{\prime}}$ at $z/c$=6.97, $V_{free}$=15.27, station 2.}
\end{figure}


\begin{figure}[H]
\centering
\includegraphics[width=4.25in]{figs/run_11/run_11_tt_contour}
\caption{Contour plot of $\overline{v_{\theta}^{\prime} v_{\theta}^{\prime}}$ at $z/c$=6.97, $V_{free}$=15.27, station 2.}
\end{figure}


\begin{figure}[H]
\centering
\includegraphics[width=4.25in]{figs/run_11/run_11_ww_contour}
\caption{Contour plot of $\overline{v_{z}^{\prime} v_{z}^{\prime}}$ at $z/c$=6.97, $V_{free}$=15.27, station 2.}
\end{figure}


\begin{figure}[H]
\centering
\includegraphics[width=4.25in]{figs/run_11/run_11_ctke_contour}
\caption{Contour plot of $k$ at $z/c$=6.97, $V_{free}$=15.27, station 2.}
\end{figure}


\begin{figure}[H]
\centering
\includegraphics[width=4.25in]{figs/run_11/run_11_num_contour}
\caption{Contour plot of $N$ at $z/c$=6.97, $V_{free}$=15.27, station 2.}
\end{figure}


\begin{figure}[H]
\centering
\includegraphics[width=6in]{figs/run_11/run_11_T_vs_r_meshscatter}
\caption{Scatter plot of azimuthal velocity vs radius at $z/c$=6.97, $V_{free}$=15.27, station 2.}
\end{figure}


\begin{figure}[H]
\centering
\includegraphics[width=6in]{figs/run_11/run_11_T_vs_r_mesh_ctkescatter}
\caption{Scatter plot of azimuthal velocity vs radius, turbulent kinetic energy, at $z/c$=6.97, $V_{free}$=15.27, station 2.}
\end{figure}


\begin{figure}[H]
\centering
\includegraphics[width=6in]{figs/run_11/run_11_ctke_vs_r_meshscatter}
\caption{Scatter plot of turbulent kinetic energy vs radius at $z/c$=6.97, $V_{free}$=15.27, station 2.}
\end{figure}


\begin{figure}[H]
\centering
\includegraphics[width=6in]{figs/run_11/run_11_turb_visc_reynolds_vs_r_meshscatter}
\caption{Scatter plot of $
u_T$ reynolds stress term vs radius at $z/c$=6.97, $V_{free}$=15.27, station 2.}
\end{figure}


\begin{figure}[H]
\centering
\includegraphics[width=6in]{figs/run_11/run_11_turb_visc_vel_grad_vs_r_meshscatter}
\caption{Scatter plot of $\nu_T$ velocity gradient term vs radius at $z/c$=6.97, $V_{free}$=15.27, station 2.}
\end{figure}


\begin{figure}[H]
\centering
\includegraphics[width=6in]{figs/run_11/run_11_turb_visc_ettap_vs_r_meshscatter}
\caption{Scatter plot of $\nu_T$ pressure relaxation term vs radius at $z/c$=6.97, $V_{free}$=15.27, station 2.}
\end{figure}


\begin{figure}[H]
\centering
\includegraphics[width=6in]{figs/run_11/run_11_turb_visc_total_vs_r_meshscatter}
\caption{Scatter plot of non-equilibrium based $\nu_T$ vs radius at $z/c$=6.97, $V_{free}$=15.27, station 2.}
\end{figure}


\begin{figure}[H]
\centering
\includegraphics[width=6in]{figs/run_11/run_11_turb_visc_vs_r_meshscatter}
\caption{Scatter plot of $\nu_T$ vs radius as calculated from the turbulent viscosity hypothesis. $z/c$=6.97, $V_{free}$=15.27, station 2.}
\end{figure}


\begin{figure}[H]
\centering
\includegraphics[width=4.25in]{figs/run_11/run_11_dPdr_contour}
\caption{Diverging contour plot of $\frac{d\bar{P}}{dr}$ from non-equilibrium theory. $z/c$=6.97, $V_{free}$=15.27, station 2.}
\end{figure}



\begin{figure}[H]
\centering
\includegraphics[width=6in]{figs/run_12/run_12_comparison}
\caption{Theoretical vortex profile fits to experimental data at $z/c$=6.97, $V_{free}$=16.89, station 2.}
\end{figure}


\begin{figure}[H]
\centering
\includegraphics[width=4.25in]{figs/run_12/run_12_stream}
\caption{Stream plot at $z/c$=6.97, $V_{free}$=16.89, station 2.}
\end{figure}


\begin{figure}[H]
\centering
\includegraphics[width=4.25in]{figs/run_12/run_12_R_contour}
\caption{Contour plot of $\overline{r}$ at $z/c$=6.97, $V_{free}$=16.89, station 2.}
\end{figure}


\begin{figure}[H]
\centering
\includegraphics[width=4.25in]{figs/run_12/run_12_T_contour}
\caption{Contour plot of $\overline{t}$ at $z/c$=6.97, $V_{free}$=16.89, station 2.}
\end{figure}


\begin{figure}[H]
\centering
\includegraphics[width=4.25in]{figs/run_12/run_12_W_contour}
\caption{Contour plot of $\overline{w}$ at $z/c$=6.97, $V_{free}$=16.89, station 2.}
\end{figure}


\begin{figure}[H]
\centering
\includegraphics[width=4.25in]{figs/run_12/run_12_rt_contour}
\caption{Contour plot of $\overline{r^\prime t^\prime}$ at $z/c$=6.97, $V_{free}$=16.89, station 2.}
\end{figure}


\begin{figure}[H]
\centering
\includegraphics[width=4.25in]{figs/run_12/run_12_rw_contour}
\caption{Contour plot of $\overline{r^\prime w^\prime}$ at $z/c$=6.97, $V_{free}$=16.89, station 2.}
\end{figure}


\begin{figure}[H]
\centering
\includegraphics[width=4.25in]{figs/run_12/run_12_tw_contour}
\caption{Contour plot of $\overline{t^\prime w^\prime}$ at $z/c$=6.97, $V_{free}$=16.89, station 2.}
\end{figure}


\begin{figure}[H]
\centering
\includegraphics[width=4.25in]{figs/run_12/run_12_rr_contour}
\caption{Contour plot of $\overline{r^\prime r^\prime}$ at $z/c$=6.97, $V_{free}$=16.89, station 2.}
\end{figure}


\begin{figure}[H]
\centering
\includegraphics[width=4.25in]{figs/run_12/run_12_tt_contour}
\caption{Contour plot of $\overline{t^\prime t^\prime}$ at $z/c$=6.97, $V_{free}$=16.89, station 2.}
\end{figure}


\begin{figure}[H]
\centering
\includegraphics[width=4.25in]{figs/run_12/run_12_ww_contour}
\caption{Contour plot of $\overline{w^\prime w^\prime}$ at $z/c$=6.97, $V_{free}$=16.89, station 2.}
\end{figure}


\begin{figure}[H]
\centering
\includegraphics[width=4.25in]{figs/run_12/run_12_ctke_contour}
\caption{Contour plot of $k$ at $z/c$=6.97, $V_{free}$=16.89, station 2.}
\end{figure}


\begin{figure}[H]
\centering
\includegraphics[width=4.25in]{figs/run_12/run_12_num_contour}
\caption{Contour plot of $N$ at $z/c$=6.97, $V_{free}$=16.89, station 2.}
\end{figure}


\begin{figure}[H]
\centering
\includegraphics[width=6in]{figs/run_12/run_12_T_vs_r_meshscatter}
\caption{Scatter plot of azimuthal velocity vs radius at $z/c$=6.97, $V_{free}$=16.89, station2.}
\end{figure}


\begin{figure}[H]
\centering
\includegraphics[width=6in]{figs/run_12/run_12_T_vs_r_mesh_ctkescatter}
\caption{Scatter plot of azimuthal velocity vs radius, turbulent kinetic energy, at $z/c$=6.97, $V_{free}$=16.89, station 2.}
\end{figure}


\begin{figure}[H]
\centering
\includegraphics[width=6in]{figs/run_12/run_12_ctke_vs_r_meshscatter}
\caption{Scatter plot of turbulent kinetic energy vs radius at $z/c$=6.97, $V_{free}$=16.89, station 2.}
\end{figure}


\begin{figure}[H]
\centering
\includegraphics[width=6in]{figs/run_12/run_12_turb_visc_ettap_top_vs_r_meshscatter}
\caption{Scatter plot of $
u_T$ reynolds stress term vs radius at $z/c$=6.97, $V_{free}$=16.89, station 2.}
\end{figure}


\begin{figure}[H]
\centering
\includegraphics[width=6in]{figs/run_12/run_12_turb_visc_ettap_bot_vs_r_meshscatter}
\caption{Scatter plot of $\nu_T$ velocity gradient term vs radius at $z/c$=6.97, $V_{free}$=16.89, station 2.}
\end{figure}


\begin{figure}[H]
\centering
\includegraphics[width=6in]{figs/run_12/run_12_turb_visc_ettap_vs_r_meshscatter}
\caption{Scatter plot of $\nu_T$ pressure relaxation term vs radius at $z/c$=6.97, $V_{free}$=16.89, station 2.}
\end{figure}


\begin{figure}[H]
\centering
\includegraphics[width=6in]{figs/run_12/run_12_turb_visc_vs_r_meshscatter}
\caption{Scatter plot of $\nu_T$ as calculated from the turbulent viscosity hypothesis. $z/c$=6.97, $V_{free}$=16.89, station 2.}
\end{figure}


\begin{figure}[H]
\centering
\includegraphics[width=6in]{figs/run_12/run_12_ctke_01rdynamic}
\caption{Plot showing dynamic variations in turbulent kinetic energy within the core. $z/c$=6.97, $V_{free}$=16.89, station 2.}
\end{figure}


\begin{figure}[H]
\centering
\includegraphics[width=6in]{figs/run_12/run_12_ctke_12rdynamic}
\caption{Plot showing dynamic variations in turbulent kinetic energy between 1 and 2 core radii. $z/c$=6.97, $V_{free}$=16.89, station 2.}
\end{figure}


\begin{figure}[H]
\centering
\includegraphics[width=6in]{figs/run_12/run_12_ctke_005rdynamic}
\caption{Plot showing dynamic variations in turbulent kinetic energy between 0 and 0.5 core radii. $z/c$=6.97, $V_{free}$=16.89, station 2.}
\end{figure}



\input{figs/tex/run_13}
\begin{figure}[H]
\centering
\includegraphics[width=6in]{figs/run_14/run_14_comparison}
\caption{Theoretical vortex profile fits to experimental data at $z/c$=6.97, $V_{free}$=21.13, station 2.}
\end{figure}


\begin{figure}[H]
\centering
\includegraphics[width=4.25in]{figs/run_14/run_14_stream}
\caption{Stream plot at $z/c$=6.97, $V_{free}$=21.13, station 2.}
\end{figure}


\begin{figure}[H]
\centering
\includegraphics[width=4.25in]{figs/run_14/run_14_R_contour}
\caption{Contour plot of $\overline{v_{r}}$ at $z/c$=6.97, $V_{free}$=21.13, station 2.}
\end{figure}


\begin{figure}[H]
\centering
\includegraphics[width=4.25in]{figs/run_14/run_14_T_contour}
\caption{Contour plot of $\overline{v_{\theta}}$ at $z/c$=6.97, $V_{free}$=21.13, station 2.}
\end{figure}


\begin{figure}[H]
\centering
\includegraphics[width=4.25in]{figs/run_14/run_14_W_contour}
\caption{Contour plot of $\overline{v_{z}}$ at $z/c$=6.97, $V_{free}$=21.13, station 2.}
\end{figure}


\begin{figure}[H]
\centering
\includegraphics[width=4.25in]{figs/run_14/run_14_rt_contour}
\caption{Contour plot of $\overline{v_{r}^{\prime} v_{\theta}^{\prime}}$ at $z/c$=6.97, $V_{free}$=21.13, station 2.}
\end{figure}


\begin{figure}[H]
\centering
\includegraphics[width=4.25in]{figs/run_14/run_14_rw_contour}
\caption{Contour plot of $\overline{v_{r}^{\prime} v_{z}^{\prime}}$ at $z/c$=6.97, $V_{free}$=21.13, station 2.}
\end{figure}


\begin{figure}[H]
\centering
\includegraphics[width=4.25in]{figs/run_14/run_14_tw_contour}
\caption{Contour plot of $\overline{v_{\theta}^{\prime} v_{z}^{\prime}}$ at $z/c$=6.97, $V_{free}$=21.13, station 2.}
\end{figure}


\begin{figure}[H]
\centering
\includegraphics[width=4.25in]{figs/run_14/run_14_rr_contour}
\caption{Contour plot of $\overline{v_{r}^{\prime} v_{r}^{\prime}}$ at $z/c$=6.97, $V_{free}$=21.13, station 2.}
\end{figure}


\begin{figure}[H]
\centering
\includegraphics[width=4.25in]{figs/run_14/run_14_tt_contour}
\caption{Contour plot of $\overline{v_{\theta}^{\prime} v_{\theta}^{\prime}}$ at $z/c$=6.97, $V_{free}$=21.13, station 2.}
\end{figure}


\begin{figure}[H]
\centering
\includegraphics[width=4.25in]{figs/run_14/run_14_ww_contour}
\caption{Contour plot of $\overline{v_{z}^{\prime} v_{z}^{\prime}}$ at $z/c$=6.97, $V_{free}$=21.13, station 2.}
\end{figure}


\begin{figure}[H]
\centering
\includegraphics[width=4.25in]{figs/run_14/run_14_ctke_contour}
\caption{Contour plot of $k$ at $z/c$=6.97, $V_{free}$=21.13, station 2.}
\end{figure}


\begin{figure}[H]
\centering
\includegraphics[width=4.25in]{figs/run_14/run_14_num_contour}
\caption{Contour plot of $N$ at $z/c$=6.97, $V_{free}$=21.13, station 2.}
\end{figure}


\begin{figure}[H]
\centering
\includegraphics[width=6in]{figs/run_14/run_14_T_vs_r_meshscatter}
\caption{Scatter plot of azimuthal velocity vs radius at $z/c$=6.97, $V_{free}$=21.13, station 2.}
\end{figure}


\begin{figure}[H]
\centering
\includegraphics[width=6in]{figs/run_14/run_14_T_vs_r_mesh_ctkescatter}
\caption{Scatter plot of azimuthal velocity vs radius, turbulent kinetic energy, at $z/c$=6.97, $V_{free}$=21.13, station 2.}
\end{figure}


\begin{figure}[H]
\centering
\includegraphics[width=6in]{figs/run_14/run_14_ctke_vs_r_meshscatter}
\caption{Scatter plot of turbulent kinetic energy vs radius at $z/c$=6.97, $V_{free}$=21.13, station 2.}
\end{figure}


\begin{figure}[H]
\centering
\includegraphics[width=6in]{figs/run_14/run_14_turb_visc_reynolds_vs_r_meshscatter}
\caption{Scatter plot of $
u_T$ reynolds stress term vs radius at $z/c$=6.97, $V_{free}$=21.13, station 2.}
\end{figure}


\begin{figure}[H]
\centering
\includegraphics[width=6in]{figs/run_14/run_14_turb_visc_vel_grad_vs_r_meshscatter}
\caption{Scatter plot of $\nu_T$ velocity gradient term vs radius at $z/c$=6.97, $V_{free}$=21.13, station 2.}
\end{figure}


\begin{figure}[H]
\centering
\includegraphics[width=6in]{figs/run_14/run_14_momentum_vel_grad_vs_r_meshscatter}
\caption{Scatter plot of the momentum based velocity gradient term vs radius at $z/c$=6.97, $V_{free}$=21.13, station 2.}
\end{figure}


\begin{figure}[H]
\centering
\includegraphics[width=6in]{figs/run_14/run_14_turb_visc_ettap_vs_r_meshscatter}
\caption{Scatter plot of $\nu_T$ pressure relaxation term vs radius at $z/c$=6.97, $V_{free}$=21.13, station 2.}
\end{figure}


\begin{figure}[H]
\centering
\includegraphics[width=6in]{figs/run_14/run_14_turb_visc_total_vs_r_meshscatter}
\caption{Scatter plot of non-equilibrium based $\nu_T$ vs radius at $z/c$=6.97, $V_{free}$=21.13, station 2.}
\end{figure}


\begin{figure}[H]
\centering
\includegraphics[width=6in]{figs/run_14/run_14_turb_visc_vs_r_meshscatter}
\caption{Scatter plot of $\nu_T$ vs radius as calculated from the turbulent viscosity hypothesis. $z/c$=6.97, $V_{free}$=21.13, station 2.}
\end{figure}


\begin{figure}[H]
\centering
\includegraphics[width=4.25in]{figs/run_14/run_14_dPdr_contour}
\caption{Diverging contour plot of $\frac{d\bar{P}}{dr}$ from non-equilibrium theory. $z/c$=6.97, $V_{free}$=21.13, station 2.}
\end{figure}



\input{figs/tex/run_15}
\begin{figure}[H]
\centering
\includegraphics[width=6in]{figs/run_16/run_16_comparison}
\caption{Theoretical vortex profile fits to experimental data at $z/c$=6.97, $V_{free}$=25.36, station 2.}
\end{figure}


\begin{figure}[H]
\centering
\includegraphics[width=4.25in]{figs/run_16/run_16_stream}
\caption{Stream plot at $z/c$=6.97, $V_{free}$=25.36, station 2.}
\end{figure}


\begin{figure}[H]
\centering
\includegraphics[width=4.25in]{figs/run_16/run_16_R_contour}
\caption{Contour plot of $\overline{v_{r}}$ at $z/c$=6.97, $V_{free}$=25.36, station 2.}
\end{figure}


\begin{figure}[H]
\centering
\includegraphics[width=4.25in]{figs/run_16/run_16_T_contour}
\caption{Contour plot of $\overline{v_{\theta}}$ at $z/c$=6.97, $V_{free}$=25.36, station 2.}
\end{figure}


\begin{figure}[H]
\centering
\includegraphics[width=4.25in]{figs/run_16/run_16_W_contour}
\caption{Contour plot of $\overline{v_{z}}$ at $z/c$=6.97, $V_{free}$=25.36, station 2.}
\end{figure}


\begin{figure}[H]
\centering
\includegraphics[width=4.25in]{figs/run_16/run_16_rt_contour}
\caption{Contour plot of $\overline{v_{r}^{\prime} v_{\theta}^{\prime}}$ at $z/c$=6.97, $V_{free}$=25.36, station 2.}
\end{figure}


\begin{figure}[H]
\centering
\includegraphics[width=4.25in]{figs/run_16/run_16_rw_contour}
\caption{Contour plot of $\overline{v_{r}^{\prime} v_{z}^{\prime}}$ at $z/c$=6.97, $V_{free}$=25.36, station 2.}
\end{figure}


\begin{figure}[H]
\centering
\includegraphics[width=4.25in]{figs/run_16/run_16_tw_contour}
\caption{Contour plot of $\overline{v_{\theta}^{\prime} v_{z}^{\prime}}$ at $z/c$=6.97, $V_{free}$=25.36, station 2.}
\end{figure}


\begin{figure}[H]
\centering
\includegraphics[width=4.25in]{figs/run_16/run_16_rr_contour}
\caption{Contour plot of $\overline{v_{r}^{\prime} v_{r}^{\prime}}$ at $z/c$=6.97, $V_{free}$=25.36, station 2.}
\end{figure}


\begin{figure}[H]
\centering
\includegraphics[width=4.25in]{figs/run_16/run_16_tt_contour}
\caption{Contour plot of $\overline{v_{\theta}^{\prime} v_{\theta}^{\prime}}$ at $z/c$=6.97, $V_{free}$=25.36, station 2.}
\end{figure}


\begin{figure}[H]
\centering
\includegraphics[width=4.25in]{figs/run_16/run_16_ww_contour}
\caption{Contour plot of $\overline{v_{z}^{\prime} v_{z}^{\prime}}$ at $z/c$=6.97, $V_{free}$=25.36, station 2.}
\end{figure}


\begin{figure}[H]
\centering
\includegraphics[width=4.25in]{figs/run_16/run_16_ctke_contour}
\caption{Contour plot of $k$ at $z/c$=6.97, $V_{free}$=25.36, station 2.}
\end{figure}


\begin{figure}[H]
\centering
\includegraphics[width=4.25in]{figs/run_16/run_16_num_contour}
\caption{Contour plot of $N$ at $z/c$=6.97, $V_{free}$=25.36, station 2.}
\end{figure}


\begin{figure}[H]
\centering
\includegraphics[width=6in]{figs/run_16/run_16_T_vs_r_meshscatter}
\caption{Scatter plot of azimuthal velocity vs radius at $z/c$=6.97, $V_{free}$=25.36, station 2.}
\end{figure}


\begin{figure}[H]
\centering
\includegraphics[width=6in]{figs/run_16/run_16_T_vs_r_mesh_ctkescatter}
\caption{Scatter plot of azimuthal velocity vs radius, turbulent kinetic energy, at $z/c$=6.97, $V_{free}$=25.36, station 2.}
\end{figure}


\begin{figure}[H]
\centering
\includegraphics[width=6in]{figs/run_16/run_16_ctke_vs_r_meshscatter}
\caption{Scatter plot of turbulent kinetic energy vs radius at $z/c$=6.97, $V_{free}$=25.36, station 2.}
\end{figure}


\begin{figure}[H]
\centering
\includegraphics[width=6in]{figs/run_16/run_16_turb_visc_reynolds_vs_r_meshscatter}
\caption{Scatter plot of $
u_T$ reynolds stress term vs radius at $z/c$=6.97, $V_{free}$=25.36, station 2.}
\end{figure}


\begin{figure}[H]
\centering
\includegraphics[width=6in]{figs/run_16/run_16_turb_visc_vel_grad_vs_r_meshscatter}
\caption{Scatter plot of $\nu_T$ velocity gradient term vs radius at $z/c$=6.97, $V_{free}$=25.36, station 2.}
\end{figure}


\begin{figure}[H]
\centering
\includegraphics[width=6in]{figs/run_16/run_16_turb_visc_by_vtheta_vs_r_meshscatter}
\caption{Scatter plot of total viscosity $\nu$ calculated by mean $\bar{v}_{\theta}$ equation with non-equilibrium pressure at $z/c$=6.97, $V_{free}$=25.36, station 2.}
\end{figure}


\begin{figure}[H]
\centering
\includegraphics[width=6in]{figs/run_16/run_16_momentum_vel_grad_vs_r_meshscatter}
\caption{Scatter plot of the momentum based velocity gradient term vs radius at $z/c$=6.97, $V_{free}$=25.36, station 2.}
\end{figure}


\begin{figure}[H]
\centering
\includegraphics[width=6in]{figs/run_16/run_16_turb_visc_ettap_vs_r_meshscatter}
\caption{Scatter plot of $\nu_T$ pressure relaxation term vs radius at $z/c$=6.97, $V_{free}$=25.36, station 2.}
\end{figure}


\begin{figure}[H]
\centering
\includegraphics[width=6in]{figs/run_16/run_16_turb_visc_total_vs_r_meshscatter}
\caption{Scatter plot of non-equilibrium based $\nu_T$ vs radius at $z/c$=6.97, $V_{free}$=25.36, station 2.}
\end{figure}


\begin{figure}[H]
\centering
\includegraphics[width=6in]{figs/run_16/run_16_turb_visc_vs_r_meshscatter}
\caption{Scatter plot of $\nu_T$ vs radius as calculated from the turbulent viscosity hypothesis. $z/c$=6.97, $V_{free}$=25.36, station 2.}
\end{figure}


\begin{figure}[H]
\centering
\includegraphics[width=6in]{figs/run_16/run_16_ctke_01rdynamic}
\caption{Plot showing dynamic variations in turbulent kinetic energy within the core. $z/c$=6.97, $V_{free}$=25.36, station 2.}
\end{figure}


\begin{figure}[H]
\centering
\includegraphics[width=6in]{figs/run_16/run_16_ctke_12rdynamic}
\caption{Plot showing dynamic variations in turbulent kinetic energy between 1 and 2 core radii. $z/c$=6.97, $V_{free}$=25.36, station 2.}
\end{figure}


\begin{figure}[H]
\centering
\includegraphics[width=6in]{figs/run_16/run_16_ctke_005rdynamic}
\caption{Plot showing dynamic variations in turbulent kinetic energy between 0 and 0.5 core radii. $z/c$=6.97, $V_{free}$=25.36, station 2.}
\end{figure}



\input{figs/tex/run_17}
\input{figs/tex/run_18}
\input{figs/tex/run_19}
\input{figs/tex/run_20}

\section{Station 3 PIV Data: 787mm Downstream}
\input{tables/station_3_measurements}
\input{figs/tex/run_21}
\begin{figure}[H]
\centering
\includegraphics[width=6in]{figs/run_22/run_22_comparison}
\caption{Theoretical vortex profile fits to experimental data at $z/c$=7.75, $V_{free}$=17.27, station 3.}
\end{figure}


\begin{figure}[H]
\centering
\includegraphics[width=4.25in]{figs/run_22/run_22_stream}
\caption{Stream plot at $z/c$=7.75, $V_{free}$=17.27, station 3.}
\end{figure}


\begin{figure}[H]
\centering
\includegraphics[width=4.25in]{figs/run_22/run_22_R_contour}
\caption{Contour plot of $\overline{v_{r}}$ at $z/c$=7.75, $V_{free}$=17.27, station 3.}
\end{figure}


\begin{figure}[H]
\centering
\includegraphics[width=4.25in]{figs/run_22/run_22_T_contour}
\caption{Contour plot of $\overline{v_{\theta}}$ at $z/c$=7.75, $V_{free}$=17.27, station 3.}
\end{figure}


\begin{figure}[H]
\centering
\includegraphics[width=4.25in]{figs/run_22/run_22_W_contour}
\caption{Contour plot of $\overline{v_{z}}$ at $z/c$=7.75, $V_{free}$=17.27, station 3.}
\end{figure}


\begin{figure}[H]
\centering
\includegraphics[width=4.25in]{figs/run_22/run_22_rt_contour}
\caption{Contour plot of $\overline{v_{r}^{\prime} v_{\theta}^{\prime}}$ at $z/c$=7.75, $V_{free}$=17.27, station 3.}
\end{figure}


\begin{figure}[H]
\centering
\includegraphics[width=4.25in]{figs/run_22/run_22_rw_contour}
\caption{Contour plot of $\overline{v_{r}^{\prime} v_{z}^{\prime}}$ at $z/c$=7.75, $V_{free}$=17.27, station 3.}
\end{figure}


\begin{figure}[H]
\centering
\includegraphics[width=4.25in]{figs/run_22/run_22_tw_contour}
\caption{Contour plot of $\overline{v_{\theta}^{\prime} v_{z}^{\prime}}$ at $z/c$=7.75, $V_{free}$=17.27, station 3.}
\end{figure}


\begin{figure}[H]
\centering
\includegraphics[width=4.25in]{figs/run_22/run_22_rr_contour}
\caption{Contour plot of $\overline{v_{r}^{\prime} v_{r}^{\prime}}$ at $z/c$=7.75, $V_{free}$=17.27, station 3.}
\end{figure}


\begin{figure}[H]
\centering
\includegraphics[width=4.25in]{figs/run_22/run_22_tt_contour}
\caption{Contour plot of $\overline{v_{\theta}^{\prime} v_{\theta}^{\prime}}$ at $z/c$=7.75, $V_{free}$=17.27, station 3.}
\end{figure}


\begin{figure}[H]
\centering
\includegraphics[width=4.25in]{figs/run_22/run_22_ww_contour}
\caption{Contour plot of $\overline{v_{z}^{\prime} v_{z}^{\prime}}$ at $z/c$=7.75, $V_{free}$=17.27, station 3.}
\end{figure}


\begin{figure}[H]
\centering
\includegraphics[width=4.25in]{figs/run_22/run_22_ctke_contour}
\caption{Contour plot of $k$ at $z/c$=7.75, $V_{free}$=17.27, station 3.}
\end{figure}


\begin{figure}[H]
\centering
\includegraphics[width=4.25in]{figs/run_22/run_22_num_contour}
\caption{Contour plot of $N$ at $z/c$=7.75, $V_{free}$=17.27, station 3.}
\end{figure}


\begin{figure}[H]
\centering
\includegraphics[width=6in]{figs/run_22/run_22_T_vs_r_meshscatter}
\caption{Scatter plot of azimuthal velocity vs radius at $z/c$=7.75, $V_{free}$=17.27, station 3.}
\end{figure}


\begin{figure}[H]
\centering
\includegraphics[width=6in]{figs/run_22/run_22_T_vs_r_mesh_ctkescatter}
\caption{Scatter plot of azimuthal velocity vs radius, turbulent kinetic energy, at $z/c$=7.75, $V_{free}$=17.27, station 3.}
\end{figure}


\begin{figure}[H]
\centering
\includegraphics[width=6in]{figs/run_22/run_22_ctke_vs_r_meshscatter}
\caption{Scatter plot of turbulent kinetic energy vs radius at $z/c$=7.75, $V_{free}$=17.27, station 3.}
\end{figure}


\begin{figure}[H]
\centering
\includegraphics[width=6in]{figs/run_22/run_22_turb_visc_reynolds_vs_r_meshscatter}
\caption{Scatter plot of $
u_T$ reynolds stress term vs radius at $z/c$=7.75, $V_{free}$=17.27, station 3.}
\end{figure}


\begin{figure}[H]
\centering
\includegraphics[width=6in]{figs/run_22/run_22_turb_visc_vel_grad_vs_r_meshscatter}
\caption{Scatter plot of $\nu_T$ velocity gradient term vs radius at $z/c$=7.75, $V_{free}$=17.27, station 3.}
\end{figure}


\begin{figure}[H]
\centering
\includegraphics[width=6in]{figs/run_22/run_22_turb_visc_by_vtheta_vs_r_meshscatter}
\caption{Scatter plot of mean total viscosity $\nu$ vs radius at $z/c$=7.75, $V_{free}$=17.27, station 3.}
\end{figure}


\begin{figure}[H]
\centering
\includegraphics[width=6in]{figs/run_22/run_22_momentum_vel_grad_vs_r_meshscatter}
\caption{Scatter plot of the momentum based velocity gradient term vs radius at $z/c$=7.75, $V_{free}$=17.27, station 3.}
\end{figure}


\begin{figure}[H]
\centering
\includegraphics[width=6in]{figs/run_22/run_22_turb_visc_ettap_vs_r_meshscatter}
\caption{Scatter plot of $\nu_T$ pressure relaxation term vs radius at $z/c$=7.75, $V_{free}$=17.27, station 3.}
\end{figure}


\begin{figure}[H]
\centering
\includegraphics[width=6in]{figs/run_22/run_22_turb_visc_total_vs_r_meshscatter}
\caption{Scatter plot of non-equilibrium based $\nu_T$ vs radius at $z/c$=7.75, $V_{free}$=17.27, station 3.}
\end{figure}


\begin{figure}[H]
\centering
\includegraphics[width=6in]{figs/run_22/run_22_turb_visc_vs_r_meshscatter}
\caption{Scatter plot of $\nu_T$ vs radius as calculated from the turbulent viscosity hypothesis. $z/c$=7.75, $V_{free}$=17.27, station 3.}
\end{figure}



\begin{figure}[H]
\centering
\includegraphics[width=6in]{figs/run_23/run_23_comparison}
\caption{Theoretical vortex profile fits to experimental data at $z/c$=7.75, $V_{free}$=19.36, station 3.}
\end{figure}


\begin{figure}[H]
\centering
\includegraphics[width=4.25in]{figs/run_23/run_23_stream}
\caption{Stream plot at $z/c$=7.75, $V_{free}$=19.36, station 3.}
\end{figure}


\begin{figure}[H]
\centering
\includegraphics[width=4.25in]{figs/run_23/run_23_R_contour}
\caption{Contour plot of $\overline{v_{r}}$ at $z/c$=7.75, $V_{free}$=19.36, station 3.}
\end{figure}


\begin{figure}[H]
\centering
\includegraphics[width=4.25in]{figs/run_23/run_23_T_contour}
\caption{Contour plot of $\overline{v_{\theta}}$ at $z/c$=7.75, $V_{free}$=19.36, station 3.}
\end{figure}


\begin{figure}[H]
\centering
\includegraphics[width=4.25in]{figs/run_23/run_23_W_contour}
\caption{Contour plot of $\overline{v_{z}}$ at $z/c$=7.75, $V_{free}$=19.36, station 3.}
\end{figure}


\begin{figure}[H]
\centering
\includegraphics[width=4.25in]{figs/run_23/run_23_rt_contour}
\caption{Contour plot of $\overline{v_{r}^{\prime} v_{\theta}^{\prime}}$ at $z/c$=7.75, $V_{free}$=19.36, station 3.}
\end{figure}


\begin{figure}[H]
\centering
\includegraphics[width=4.25in]{figs/run_23/run_23_rw_contour}
\caption{Contour plot of $\overline{v_{r}^{\prime} v_{z}^{\prime}}$ at $z/c$=7.75, $V_{free}$=19.36, station 3.}
\end{figure}


\begin{figure}[H]
\centering
\includegraphics[width=4.25in]{figs/run_23/run_23_tw_contour}
\caption{Contour plot of $\overline{v_{\theta}^{\prime} v_{z}^{\prime}}$ at $z/c$=7.75, $V_{free}$=19.36, station 3.}
\end{figure}


\begin{figure}[H]
\centering
\includegraphics[width=4.25in]{figs/run_23/run_23_rr_contour}
\caption{Contour plot of $\overline{v_{r}^{\prime} v_{r}^{\prime}}$ at $z/c$=7.75, $V_{free}$=19.36, station 3.}
\end{figure}


\begin{figure}[H]
\centering
\includegraphics[width=4.25in]{figs/run_23/run_23_tt_contour}
\caption{Contour plot of $\overline{v_{\theta}^{\prime} v_{\theta}^{\prime}}$ at $z/c$=7.75, $V_{free}$=19.36, station 3.}
\end{figure}


\begin{figure}[H]
\centering
\includegraphics[width=4.25in]{figs/run_23/run_23_ww_contour}
\caption{Contour plot of $\overline{v_{z}^{\prime} v_{z}^{\prime}}$ at $z/c$=7.75, $V_{free}$=19.36, station 3.}
\end{figure}


\begin{figure}[H]
\centering
\includegraphics[width=4.25in]{figs/run_23/run_23_ctke_contour}
\caption{Contour plot of $k$ at $z/c$=7.75, $V_{free}$=19.36, station 3.}
\end{figure}


\begin{figure}[H]
\centering
\includegraphics[width=4.25in]{figs/run_23/run_23_num_contour}
\caption{Contour plot of $N$ at $z/c$=7.75, $V_{free}$=19.36, station 3.}
\end{figure}


\begin{figure}[H]
\centering
\includegraphics[width=6in]{figs/run_23/run_23_T_vs_r_meshscatter}
\caption{Scatter plot of azimuthal velocity vs radius at $z/c$=7.75, $V_{free}$=19.36, station 3.}
\end{figure}


\begin{figure}[H]
\centering
\includegraphics[width=6in]{figs/run_23/run_23_T_vs_r_mesh_ctkescatter}
\caption{Scatter plot of azimuthal velocity vs radius, turbulent kinetic energy, at $z/c$=7.75, $V_{free}$=19.36, station 3.}
\end{figure}


\begin{figure}[H]
\centering
\includegraphics[width=6in]{figs/run_23/run_23_ctke_vs_r_meshscatter}
\caption{Scatter plot of turbulent kinetic energy vs radius at $z/c$=7.75, $V_{free}$=19.36, station 3.}
\end{figure}


\begin{figure}[H]
\centering
\includegraphics[width=6in]{figs/run_23/run_23_turb_visc_reynolds_vs_r_meshscatter}
\caption{Scatter plot of $
u_T$ reynolds stress term vs radius at $z/c$=7.75, $V_{free}$=19.36, station 3.}
\end{figure}


\begin{figure}[H]
\centering
\includegraphics[width=6in]{figs/run_23/run_23_turb_visc_vel_grad_vs_r_meshscatter}
\caption{Scatter plot of $\nu_T$ velocity gradient term vs radius at $z/c$=7.75, $V_{free}$=19.36, station 3.}
\end{figure}


\begin{figure}[H]
\centering
\includegraphics[width=6in]{figs/run_23/run_23_turb_visc_by_vtheta_vs_r_meshscatter}
\caption{Scatter plot of total viscosity $\nu$ calculated by mean $\bar{v}_{\theta}$ equation with non-equilibrium pressure at $z/c$=7.75, $V_{free}$=19.36, station 3.}
\end{figure}


\begin{figure}[H]
\centering
\includegraphics[width=6in]{figs/run_23/run_23_momentum_vel_grad_vs_r_meshscatter}
\caption{Scatter plot of the momentum based velocity gradient term vs radius at $z/c$=7.75, $V_{free}$=19.36, station 3.}
\end{figure}


\begin{figure}[H]
\centering
\includegraphics[width=6in]{figs/run_23/run_23_turb_visc_ettap_vs_r_meshscatter}
\caption{Scatter plot of $\nu_T$ pressure relaxation term vs radius at $z/c$=7.75, $V_{free}$=19.36, station 3.}
\end{figure}


\begin{figure}[H]
\centering
\includegraphics[width=6in]{figs/run_23/run_23_turb_visc_total_vs_r_meshscatter}
\caption{Scatter plot of non-equilibrium based $\nu_T$ vs radius at $z/c$=7.75, $V_{free}$=19.36, station 3.}
\end{figure}


\begin{figure}[H]
\centering
\includegraphics[width=6in]{figs/run_23/run_23_turb_visc_vs_r_meshscatter}
\caption{Scatter plot of $\nu_T$ vs radius as calculated from the turbulent viscosity hypothesis. $z/c$=7.75, $V_{free}$=19.36, station 3.}
\end{figure}



\begin{figure}[H]
\centering
\includegraphics[width=6in]{figs/run_24/run_24_comparison}
\caption{Theoretical vortex profile fits to experimental data at $z/c$=7.75, $V_{free}$=21.09, station 3.}
\end{figure}


\begin{figure}[H]
\centering
\includegraphics[width=4.25in]{figs/run_24/run_24_stream}
\caption{Stream plot at $z/c$=7.75, $V_{free}$=21.09, station 3.}
\end{figure}


\begin{figure}[H]
\centering
\includegraphics[width=4.25in]{figs/run_24/run_24_R_contour}
\caption{Contour plot of $\overline{r}$ at $z/c$=7.75, $V_{free}$=21.09, station 3.}
\end{figure}


\begin{figure}[H]
\centering
\includegraphics[width=4.25in]{figs/run_24/run_24_T_contour}
\caption{Contour plot of $\overline{t}$ at $z/c$=7.75, $V_{free}$=21.09, station 3.}
\end{figure}


\begin{figure}[H]
\centering
\includegraphics[width=4.25in]{figs/run_24/run_24_W_contour}
\caption{Contour plot of $\overline{w}$ at $z/c$=7.75, $V_{free}$=21.09, station 3.}
\end{figure}


\begin{figure}[H]
\centering
\includegraphics[width=4.25in]{figs/run_24/run_24_rt_contour}
\caption{Contour plot of $\overline{r^\prime t^\prime}$ at $z/c$=7.75, $V_{free}$=21.09, station 3.}
\end{figure}


\begin{figure}[H]
\centering
\includegraphics[width=4.25in]{figs/run_24/run_24_rw_contour}
\caption{Contour plot of $\overline{r^\prime w^\prime}$ at $z/c$=7.75, $V_{free}$=21.09, station 3.}
\end{figure}


\begin{figure}[H]
\centering
\includegraphics[width=4.25in]{figs/run_24/run_24_tw_contour}
\caption{Contour plot of $\overline{t^\prime w^\prime}$ at $z/c$=7.75, $V_{free}$=21.09, station 3.}
\end{figure}


\begin{figure}[H]
\centering
\includegraphics[width=4.25in]{figs/run_24/run_24_rr_contour}
\caption{Contour plot of $\overline{r^\prime r^\prime}$ at $z/c$=7.75, $V_{free}$=21.09, station 3.}
\end{figure}


\begin{figure}[H]
\centering
\includegraphics[width=4.25in]{figs/run_24/run_24_tt_contour}
\caption{Contour plot of $\overline{t^\prime t^\prime}$ at $z/c$=7.75, $V_{free}$=21.09, station 3.}
\end{figure}


\begin{figure}[H]
\centering
\includegraphics[width=4.25in]{figs/run_24/run_24_ww_contour}
\caption{Contour plot of $\overline{w^\prime w^\prime}$ at $z/c$=7.75, $V_{free}$=21.09, station 3.}
\end{figure}


\begin{figure}[H]
\centering
\includegraphics[width=4.25in]{figs/run_24/run_24_ctke_contour}
\caption{Contour plot of $k$ at $z/c$=7.75, $V_{free}$=21.09, station 3.}
\end{figure}


\begin{figure}[H]
\centering
\includegraphics[width=4.25in]{figs/run_24/run_24_num_contour}
\caption{Contour plot of $N$ at $z/c$=7.75, $V_{free}$=21.09, station 3.}
\end{figure}


\begin{figure}[H]
\centering
\includegraphics[width=6in]{figs/run_24/run_24_T_vs_r_meshscatter}
\caption{Scatter plot of azimuthal velocity vs radius at $z/c$=7.75, $V_{free}$=21.09, station3.}
\end{figure}


\begin{figure}[H]
\centering
\includegraphics[width=6in]{figs/run_24/run_24_T_vs_r_mesh_ctkescatter}
\caption{Scatter plot of azimuthal velocity vs radius, turbulent kinetic energy, at $z/c$=7.75, $V_{free}$=21.09, station 3.}
\end{figure}


\begin{figure}[H]
\centering
\includegraphics[width=6in]{figs/run_24/run_24_ctke_vs_r_meshscatter}
\caption{Scatter plot of turbulent kinetic energy vs radius at $z/c$=7.75, $V_{free}$=21.09, station 3.}
\end{figure}


\begin{figure}[H]
\centering
\includegraphics[width=6in]{figs/run_24/run_24_turb_visc_ettap_top_vs_r_meshscatter}
\caption{Scatter plot of $
u_T$ reynolds stress term vs radius at $z/c$=7.75, $V_{free}$=21.09, station 3.}
\end{figure}


\begin{figure}[H]
\centering
\includegraphics[width=6in]{figs/run_24/run_24_turb_visc_ettap_bot_vs_r_meshscatter}
\caption{Scatter plot of $\nu_T$ velocity gradient term vs radius at $z/c$=7.75, $V_{free}$=21.09, station 3.}
\end{figure}


\begin{figure}[H]
\centering
\includegraphics[width=6in]{figs/run_24/run_24_turb_visc_ettap_vs_r_meshscatter}
\caption{Scatter plot of $\nu_T$ pressure relaxation term vs radius at $z/c$=7.75, $V_{free}$=21.09, station 3.}
\end{figure}


\begin{figure}[H]
\centering
\includegraphics[width=6in]{figs/run_24/run_24_turb_visc_vs_r_meshscatter}
\caption{Scatter plot of $\nu_T$ as calculated from the turbulent viscosity hypothesis. $z/c$=7.75, $V_{free}$=21.09, station 3.}
\end{figure}


\begin{figure}[H]
\centering
\includegraphics[width=6in]{figs/run_24/run_24_ctke_01rdynamic}
\caption{Plot showing dynamic variations in turbulent kinetic energy within the core. $z/c$=7.75, $V_{free}$=21.09, station 3.}
\end{figure}


\begin{figure}[H]
\centering
\includegraphics[width=6in]{figs/run_24/run_24_ctke_12rdynamic}
\caption{Plot showing dynamic variations in turbulent kinetic energy between 1 and 2 core radii. $z/c$=7.75, $V_{free}$=21.09, station 3.}
\end{figure}


\begin{figure}[H]
\centering
\includegraphics[width=6in]{figs/run_24/run_24_ctke_005rdynamic}
\caption{Plot showing dynamic variations in turbulent kinetic energy between 0 and 0.5 core radii. $z/c$=7.75, $V_{free}$=21.09, station 3.}
\end{figure}



\input{figs/tex/run_25}
\input{figs/tex/run_26}
\input{figs/tex/run_27}
\input{figs/tex/run_28}
\input{figs/tex/run_29}
\input{figs/tex/run_30}

\section{Station 4 PIV Data: 863mm Downstream}
\renewcommand\baselinestretch{1.3}\selectfont
\begin{table}[H]
\begin{center}
\begin{tabular}{|ccccccccccc|}
	\hline
	Run & $I_Z$ & $V_{nom}$ & $dt$ & $V_{fs}$ & $\sigma_{V_{fs}}$ & $Q$ & $P_{atm}$ & $T_{tunnel}$ & $\phi$ & $\eta_P$\\
	$ID$ & $(in)$ & $(m/s)$ & $(\mu s)$ & $(m/s)$ & $(m/s)$ & $(Pa)$ & $(Pa)$ & $(\degree K)$ & $(\%)$ & $(\mu s)$\\
	\hline
	31 & 863 & 15 & 40 & 14.86 & 0.02 & 133 & 101865 & 295.75 & 63.8 & 0.354\\
	32 & 863 & 17 & 40 & 17.39 & 0.02 & 180 & 101855 & 295.95 & 63.8 & 0.354\\
	33 & 863 & 19 & 40 & 19.08 & 0.02 & 219 & 101847 & 296.1 & 63.8 & 0.354\\
	34 & 863 & 21 & 40 & 21.13 & 0.05 & 267 & 101845 & 296.15 & 63.8 & 0.354\\
	35 & 863 & 23 & 40 & 23.29 & 0.02 & 323 & 101844 & 296.45 & 63.8 & 0.354\\
	36 & 863 & 25 & 25 & 24.98 & 0.05 & 373 & 101840 & 296.65 & 65.6 & 0.344\\
	37 & 863 & 27 & 25 & 27.09 & 0.05 & 438 & 101843 & 297 & 65.6 & 0.344\\
	38 & 863 & 29 & 25 & 28.81 & 0.03 & 493 & 101848 & 297.55 & 65.6 & 0.344\\
	39 & 863 & 31 & 25 & 30.87 & 0.04 & 570 & 101842 & 298.15 & - & -\\
	40 & 863 & 33 & 25 & 33.44 & 0.06 & 661 & 101844 & 298.35 & - & -\\
	\hline
\end{tabular}
\caption{Experimental measurements for station 4}
\label{table:station_4_measurements}
\end{center}
\end{table}
\renewcommand\baselinestretch{2}\selectfont

\begin{figure}[H]
\centering
\includegraphics[width=6in]{figs/run_31/run_31_comparison}
\caption{Theoretical vortex profile fits to experimental data at $z/c$=8.50, $V_{free}$=14.86, station 4.}
\end{figure}


\begin{figure}[H]
\centering
\includegraphics[width=4.25in]{figs/run_31/run_31_stream}
\caption{Stream plot at $z/c$=8.50, $V_{free}$=14.86, station 4.}
\end{figure}


\begin{figure}[H]
\centering
\includegraphics[width=4.25in]{figs/run_31/run_31_R_contour}
\caption{Contour plot of $\overline{r}$ at $z/c$=8.50, $V_{free}$=14.86, station 4.}
\end{figure}


\begin{figure}[H]
\centering
\includegraphics[width=4.25in]{figs/run_31/run_31_T_contour}
\caption{Contour plot of $\overline{t}$ at $z/c$=8.50, $V_{free}$=14.86, station 4.}
\end{figure}


\begin{figure}[H]
\centering
\includegraphics[width=4.25in]{figs/run_31/run_31_W_contour}
\caption{Contour plot of $\overline{w}$ at $z/c$=8.50, $V_{free}$=14.86, station 4.}
\end{figure}


\begin{figure}[H]
\centering
\includegraphics[width=4.25in]{figs/run_31/run_31_rt_contour}
\caption{Contour plot of $\overline{r^\prime t^\prime}$ at $z/c$=8.50, $V_{free}$=14.86, station 4.}
\end{figure}


\begin{figure}[H]
\centering
\includegraphics[width=4.25in]{figs/run_31/run_31_rw_contour}
\caption{Contour plot of $\overline{r^\prime w^\prime}$ at $z/c$=8.50, $V_{free}$=14.86, station 4.}
\end{figure}


\begin{figure}[H]
\centering
\includegraphics[width=4.25in]{figs/run_31/run_31_tw_contour}
\caption{Contour plot of $\overline{t^\prime w^\prime}$ at $z/c$=8.50, $V_{free}$=14.86, station 4.}
\end{figure}


\begin{figure}[H]
\centering
\includegraphics[width=4.25in]{figs/run_31/run_31_rr_contour}
\caption{Contour plot of $\overline{r^\prime r^\prime}$ at $z/c$=8.50, $V_{free}$=14.86, station 4.}
\end{figure}


\begin{figure}[H]
\centering
\includegraphics[width=4.25in]{figs/run_31/run_31_tt_contour}
\caption{Contour plot of $\overline{t^\prime t^\prime}$ at $z/c$=8.50, $V_{free}$=14.86, station 4.}
\end{figure}


\begin{figure}[H]
\centering
\includegraphics[width=4.25in]{figs/run_31/run_31_ww_contour}
\caption{Contour plot of $\overline{w^\prime w^\prime}$ at $z/c$=8.50, $V_{free}$=14.86, station 4.}
\end{figure}


\begin{figure}[H]
\centering
\includegraphics[width=4.25in]{figs/run_31/run_31_ctke_contour}
\caption{Contour plot of $k$ at $z/c$=8.50, $V_{free}$=14.86, station 4.}
\end{figure}


\begin{figure}[H]
\centering
\includegraphics[width=4.25in]{figs/run_31/run_31_num_contour}
\caption{Contour plot of $N$ at $z/c$=8.50, $V_{free}$=14.86, station 4.}
\end{figure}


\begin{figure}[H]
\centering
\includegraphics[width=6in]{figs/run_31/run_31_T_vs_r_meshscatter}
\caption{Scatter plot of azimuthal velocity vs radius at $z/c$=8.50, $V_{free}$=14.86, station4.}
\end{figure}


\begin{figure}[H]
\centering
\includegraphics[width=6in]{figs/run_31/run_31_T_vs_r_mesh_ctkescatter}
\caption{Scatter plot of azimuthal velocity vs radius, turbulent kinetic energy, at $z/c$=8.50, $V_{free}$=14.86, station 4.}
\end{figure}


\begin{figure}[H]
\centering
\includegraphics[width=6in]{figs/run_31/run_31_ctke_vs_r_meshscatter}
\caption{Scatter plot of turbulent kinetic energy vs radius at $z/c$=8.50, $V_{free}$=14.86, station 4.}
\end{figure}


\begin{figure}[H]
\centering
\includegraphics[width=6in]{figs/run_31/run_31_turb_visc_reynolds_vs_r_meshscatter}
\caption{Scatter plot of $
u_T$ reynolds stress term vs radius at $z/c$=8.50, $V_{free}$=14.86, station 4.}
\end{figure}


\begin{figure}[H]
\centering
\includegraphics[width=6in]{figs/run_31/run_31_turb_visc_vel_grad_vs_r_meshscatter}
\caption{Scatter plot of $\nu_T$ velocity gradient term vs radius at $z/c$=8.50, $V_{free}$=14.86, station 4.}
\end{figure}


\begin{figure}[H]
\centering
\includegraphics[width=6in]{figs/run_31/run_31_turb_visc_ettap_vs_r_meshscatter}
\caption{Scatter plot of $\nu_T$ pressure relaxation term vs radius at $z/c$=8.50, $V_{free}$=14.86, station 4.}
\end{figure}


\begin{figure}[H]
\centering
\includegraphics[width=6in]{figs/run_31/run_31_turb_visc_total_vs_r_meshscatter}
\caption{Scatter plot of non-equilibrium based $\nu_T$ vs radius at $z/c$=8.50, $V_{free}$=14.86, station 4.}
\end{figure}


\begin{figure}[H]
\centering
\includegraphics[width=6in]{figs/run_31/run_31_turb_visc_vs_r_meshscatter}
\caption{Scatter plot of $\nu_T$ vs radius as calculated from the turbulent viscosity hypothesis. $z/c$=8.50, $V_{free}$=14.86, station 4.}
\end{figure}


\begin{figure}[H]
\centering
\includegraphics[width=6in]{figs/run_31/run_31_ctke_01rdynamic}
\caption{Plot showing dynamic variations in turbulent kinetic energy within the core. $z/c$=8.50, $V_{free}$=14.86, station 4.}
\end{figure}


\begin{figure}[H]
\centering
\includegraphics[width=6in]{figs/run_31/run_31_ctke_12rdynamic}
\caption{Plot showing dynamic variations in turbulent kinetic energy between 1 and 2 core radii. $z/c$=8.50, $V_{free}$=14.86, station 4.}
\end{figure}


\begin{figure}[H]
\centering
\includegraphics[width=6in]{figs/run_31/run_31_ctke_005rdynamic}
\caption{Plot showing dynamic variations in turbulent kinetic energy between 0 and 0.5 core radii. $z/c$=8.50, $V_{free}$=14.86, station 4.}
\end{figure}



\input{figs/tex/run_32}
\begin{figure}[H]
\centering
\includegraphics[width=6in]{figs/run_33/run_33_comparison}
\caption{Theoretical vortex profile fits to experimental data at $z/c$=8.50, $V_{free}$=19.08, station 4.}
\end{figure}


\begin{figure}[H]
\centering
\includegraphics[width=4.25in]{figs/run_33/run_33_stream}
\caption{Stream plot at $z/c$=8.50, $V_{free}$=19.08, station 4.}
\end{figure}


\begin{figure}[H]
\centering
\includegraphics[width=4.25in]{figs/run_33/run_33_R_contour}
\caption{Contour plot of $\overline{r}$ at $z/c$=8.50, $V_{free}$=19.08, station 4.}
\end{figure}


\begin{figure}[H]
\centering
\includegraphics[width=4.25in]{figs/run_33/run_33_T_contour}
\caption{Contour plot of $\overline{t}$ at $z/c$=8.50, $V_{free}$=19.08, station 4.}
\end{figure}


\begin{figure}[H]
\centering
\includegraphics[width=4.25in]{figs/run_33/run_33_W_contour}
\caption{Contour plot of $\overline{w}$ at $z/c$=8.50, $V_{free}$=19.08, station 4.}
\end{figure}


\begin{figure}[H]
\centering
\includegraphics[width=4.25in]{figs/run_33/run_33_rt_contour}
\caption{Contour plot of $\overline{r^\prime t^\prime}$ at $z/c$=8.50, $V_{free}$=19.08, station 4.}
\end{figure}


\begin{figure}[H]
\centering
\includegraphics[width=4.25in]{figs/run_33/run_33_rw_contour}
\caption{Contour plot of $\overline{r^\prime w^\prime}$ at $z/c$=8.50, $V_{free}$=19.08, station 4.}
\end{figure}


\begin{figure}[H]
\centering
\includegraphics[width=4.25in]{figs/run_33/run_33_tw_contour}
\caption{Contour plot of $\overline{t^\prime w^\prime}$ at $z/c$=8.50, $V_{free}$=19.08, station 4.}
\end{figure}


\begin{figure}[H]
\centering
\includegraphics[width=4.25in]{figs/run_33/run_33_rr_contour}
\caption{Contour plot of $\overline{r^\prime r^\prime}$ at $z/c$=8.50, $V_{free}$=19.08, station 4.}
\end{figure}


\begin{figure}[H]
\centering
\includegraphics[width=4.25in]{figs/run_33/run_33_tt_contour}
\caption{Contour plot of $\overline{t^\prime t^\prime}$ at $z/c$=8.50, $V_{free}$=19.08, station 4.}
\end{figure}


\begin{figure}[H]
\centering
\includegraphics[width=4.25in]{figs/run_33/run_33_ww_contour}
\caption{Contour plot of $\overline{w^\prime w^\prime}$ at $z/c$=8.50, $V_{free}$=19.08, station 4.}
\end{figure}


\begin{figure}[H]
\centering
\includegraphics[width=4.25in]{figs/run_33/run_33_ctke_contour}
\caption{Contour plot of $k$ at $z/c$=8.50, $V_{free}$=19.08, station 4.}
\end{figure}


\begin{figure}[H]
\centering
\includegraphics[width=4.25in]{figs/run_33/run_33_num_contour}
\caption{Contour plot of $N$ at $z/c$=8.50, $V_{free}$=19.08, station 4.}
\end{figure}


\begin{figure}[H]
\centering
\includegraphics[width=6in]{figs/run_33/run_33_T_vs_r_meshscatter}
\caption{Scatter plot of azimuthal velocity vs radius at $z/c$=8.50, $V_{free}$=19.08, station4.}
\end{figure}


\begin{figure}[H]
\centering
\includegraphics[width=6in]{figs/run_33/run_33_T_vs_r_mesh_ctkescatter}
\caption{Scatter plot of azimuthal velocity vs radius, turbulent kinetic energy, at $z/c$=8.50, $V_{free}$=19.08, station 4.}
\end{figure}


\begin{figure}[H]
\centering
\includegraphics[width=6in]{figs/run_33/run_33_ctke_vs_r_meshscatter}
\caption{Scatter plot of turbulent kinetic energy vs radius at $z/c$=8.50, $V_{free}$=19.08, station 4.}
\end{figure}


\begin{figure}[H]
\centering
\includegraphics[width=6in]{figs/run_33/run_33_turb_visc_ettap_top_vs_r_meshscatter}
\caption{Scatter plot of $
u_T$ reynolds stress term vs radius at $z/c$=8.50, $V_{free}$=19.08, station 4.}
\end{figure}


\begin{figure}[H]
\centering
\includegraphics[width=6in]{figs/run_33/run_33_turb_visc_ettap_bot_vs_r_meshscatter}
\caption{Scatter plot of $\nu_T$ velocity gradient term vs radius at $z/c$=8.50, $V_{free}$=19.08, station 4.}
\end{figure}


\begin{figure}[H]
\centering
\includegraphics[width=6in]{figs/run_33/run_33_turb_visc_ettap_vs_r_meshscatter}
\caption{Scatter plot of $\nu_T$ pressure relaxation term vs radius at $z/c$=8.50, $V_{free}$=19.08, station 4.}
\end{figure}


\begin{figure}[H]
\centering
\includegraphics[width=6in]{figs/run_33/run_33_turb_visc_vs_r_meshscatter}
\caption{Scatter plot of $\nu_T$ as calculated from the turbulent viscosity hypothesis. $z/c$=8.50, $V_{free}$=19.08, station 4.}
\end{figure}


\begin{figure}[H]
\centering
\includegraphics[width=6in]{figs/run_33/run_33_ctke_01rdynamic}
\caption{Plot showing dynamic variations in turbulent kinetic energy within the core. $z/c$=8.50, $V_{free}$=19.08, station 4.}
\end{figure}


\begin{figure}[H]
\centering
\includegraphics[width=6in]{figs/run_33/run_33_ctke_12rdynamic}
\caption{Plot showing dynamic variations in turbulent kinetic energy between 1 and 2 core radii. $z/c$=8.50, $V_{free}$=19.08, station 4.}
\end{figure}


\begin{figure}[H]
\centering
\includegraphics[width=6in]{figs/run_33/run_33_ctke_005rdynamic}
\caption{Plot showing dynamic variations in turbulent kinetic energy between 0 and 0.5 core radii. $z/c$=8.50, $V_{free}$=19.08, station 4.}
\end{figure}



\input{figs/tex/run_34}
\input{figs/tex/run_35}
\begin{figure}[H]
\centering
\includegraphics[width=6in]{figs/run_36/run_36_comparison}
\caption{Theoretical vortex profile fits to experimental data at $z/c$=8.50, $V_{free}$=24.98, station 4.}
\end{figure}


\begin{figure}[H]
\centering
\includegraphics[width=4.25in]{figs/run_36/run_36_stream}
\caption{Stream plot at $z/c$=8.50, $V_{free}$=24.98, station 4.}
\end{figure}


\begin{figure}[H]
\centering
\includegraphics[width=4.25in]{figs/run_36/run_36_R_contour}
\caption{Contour plot of $\overline{v_{r}}$ at $z/c$=8.50, $V_{free}$=24.98, station 4.}
\end{figure}


\begin{figure}[H]
\centering
\includegraphics[width=4.25in]{figs/run_36/run_36_T_contour}
\caption{Contour plot of $\overline{v_{\theta}}$ at $z/c$=8.50, $V_{free}$=24.98, station 4.}
\end{figure}


\begin{figure}[H]
\centering
\includegraphics[width=4.25in]{figs/run_36/run_36_W_contour}
\caption{Contour plot of $\overline{v_{z}}$ at $z/c$=8.50, $V_{free}$=24.98, station 4.}
\end{figure}


\begin{figure}[H]
\centering
\includegraphics[width=4.25in]{figs/run_36/run_36_rt_contour}
\caption{Contour plot of $\overline{v_{r}^{\prime} v_{\theta}^{\prime}}$ at $z/c$=8.50, $V_{free}$=24.98, station 4.}
\end{figure}


\begin{figure}[H]
\centering
\includegraphics[width=4.25in]{figs/run_36/run_36_rw_contour}
\caption{Contour plot of $\overline{v_{r}^{\prime} v_{z}^{\prime}}$ at $z/c$=8.50, $V_{free}$=24.98, station 4.}
\end{figure}


\begin{figure}[H]
\centering
\includegraphics[width=4.25in]{figs/run_36/run_36_tw_contour}
\caption{Contour plot of $\overline{v_{\theta}^{\prime} v_{z}^{\prime}}$ at $z/c$=8.50, $V_{free}$=24.98, station 4.}
\end{figure}


\begin{figure}[H]
\centering
\includegraphics[width=4.25in]{figs/run_36/run_36_rr_contour}
\caption{Contour plot of $\overline{v_{r}^{\prime} v_{r}^{\prime}}$ at $z/c$=8.50, $V_{free}$=24.98, station 4.}
\end{figure}


\begin{figure}[H]
\centering
\includegraphics[width=4.25in]{figs/run_36/run_36_tt_contour}
\caption{Contour plot of $\overline{v_{\theta}^{\prime} v_{\theta}^{\prime}}$ at $z/c$=8.50, $V_{free}$=24.98, station 4.}
\end{figure}


\begin{figure}[H]
\centering
\includegraphics[width=4.25in]{figs/run_36/run_36_ww_contour}
\caption{Contour plot of $\overline{v_{z}^{\prime} v_{z}^{\prime}}$ at $z/c$=8.50, $V_{free}$=24.98, station 4.}
\end{figure}


\begin{figure}[H]
\centering
\includegraphics[width=4.25in]{figs/run_36/run_36_ctke_contour}
\caption{Contour plot of $k$ at $z/c$=8.50, $V_{free}$=24.98, station 4.}
\end{figure}


\begin{figure}[H]
\centering
\includegraphics[width=4.25in]{figs/run_36/run_36_num_contour}
\caption{Contour plot of $N$ at $z/c$=8.50, $V_{free}$=24.98, station 4.}
\end{figure}


\begin{figure}[H]
\centering
\includegraphics[width=6in]{figs/run_36/run_36_T_vs_r_meshscatter}
\caption{Scatter plot of azimuthal velocity vs radius at $z/c$=8.50, $V_{free}$=24.98, station 4.}
\end{figure}


\begin{figure}[H]
\centering
\includegraphics[width=6in]{figs/run_36/run_36_T_vs_r_mesh_ctkescatter}
\caption{Scatter plot of azimuthal velocity vs radius, turbulent kinetic energy, at $z/c$=8.50, $V_{free}$=24.98, station 4.}
\end{figure}


\begin{figure}[H]
\centering
\includegraphics[width=6in]{figs/run_36/run_36_ctke_vs_r_meshscatter}
\caption{Scatter plot of turbulent kinetic energy vs radius at $z/c$=8.50, $V_{free}$=24.98, station 4.}
\end{figure}


\begin{figure}[H]
\centering
\includegraphics[width=6in]{figs/run_36/run_36_turb_visc_reynolds_vs_r_meshscatter}
\caption{Scatter plot of $
u_T$ reynolds stress term vs radius at $z/c$=8.50, $V_{free}$=24.98, station 4.}
\end{figure}


\begin{figure}[H]
\centering
\includegraphics[width=6in]{figs/run_36/run_36_turb_visc_vel_grad_vs_r_meshscatter}
\caption{Scatter plot of $\nu_T$ velocity gradient term vs radius at $z/c$=8.50, $V_{free}$=24.98, station 4.}
\end{figure}


\begin{figure}[H]
\centering
\includegraphics[width=6in]{figs/run_36/run_36_turb_visc_by_vtheta_vs_r_meshscatter}
\caption{Scatter plot of mean total viscosity $\nu$ vs radius at $z/c$=8.50, $V_{free}$=24.98, station 4.}
\end{figure}


\begin{figure}[H]
\centering
\includegraphics[width=6in]{figs/run_36/run_36_momentum_vel_grad_vs_r_meshscatter}
\caption{Scatter plot of the momentum based velocity gradient term vs radius at $z/c$=8.50, $V_{free}$=24.98, station 4.}
\end{figure}


\begin{figure}[H]
\centering
\includegraphics[width=6in]{figs/run_36/run_36_turb_visc_ettap_vs_r_meshscatter}
\caption{Scatter plot of $\nu_T$ pressure relaxation term vs radius at $z/c$=8.50, $V_{free}$=24.98, station 4.}
\end{figure}


\begin{figure}[H]
\centering
\includegraphics[width=6in]{figs/run_36/run_36_turb_visc_total_vs_r_meshscatter}
\caption{Scatter plot of non-equilibrium based $\nu_T$ vs radius at $z/c$=8.50, $V_{free}$=24.98, station 4.}
\end{figure}


\begin{figure}[H]
\centering
\includegraphics[width=6in]{figs/run_36/run_36_turb_visc_vs_r_meshscatter}
\caption{Scatter plot of $\nu_T$ vs radius as calculated from the turbulent viscosity hypothesis. $z/c$=8.50, $V_{free}$=24.98, station 4.}
\end{figure}



\begin{figure}[H]
\centering
\includegraphics[width=6in]{figs/run_37/run_37_comparison}
\caption{Theoretical vortex profile fits to experimental data at $z/c$=8.50, $V_{free}$=27.09, station 4.}
\end{figure}


\begin{figure}[H]
\centering
\includegraphics[width=4.25in]{figs/run_37/run_37_stream}
\caption{Stream plot at $z/c$=8.50, $V_{free}$=27.09, station 4.}
\end{figure}


\begin{figure}[H]
\centering
\includegraphics[width=4.25in]{figs/run_37/run_37_R_contour}
\caption{Contour plot of $\overline{v_{r}}$ at $z/c$=8.50, $V_{free}$=27.09, station 4.}
\end{figure}


\begin{figure}[H]
\centering
\includegraphics[width=4.25in]{figs/run_37/run_37_T_contour}
\caption{Contour plot of $\overline{v_{\theta}}$ at $z/c$=8.50, $V_{free}$=27.09, station 4.}
\end{figure}


\begin{figure}[H]
\centering
\includegraphics[width=4.25in]{figs/run_37/run_37_W_contour}
\caption{Contour plot of $\overline{v_{z}}$ at $z/c$=8.50, $V_{free}$=27.09, station 4.}
\end{figure}


\begin{figure}[H]
\centering
\includegraphics[width=4.25in]{figs/run_37/run_37_rt_contour}
\caption{Contour plot of $\overline{v_{r}^{\prime} v_{\theta}^{\prime}}$ at $z/c$=8.50, $V_{free}$=27.09, station 4.}
\end{figure}


\begin{figure}[H]
\centering
\includegraphics[width=4.25in]{figs/run_37/run_37_rw_contour}
\caption{Contour plot of $\overline{v_{r}^{\prime} v_{z}^{\prime}}$ at $z/c$=8.50, $V_{free}$=27.09, station 4.}
\end{figure}


\begin{figure}[H]
\centering
\includegraphics[width=4.25in]{figs/run_37/run_37_tw_contour}
\caption{Contour plot of $\overline{v_{\theta}^{\prime} v_{z}^{\prime}}$ at $z/c$=8.50, $V_{free}$=27.09, station 4.}
\end{figure}


\begin{figure}[H]
\centering
\includegraphics[width=4.25in]{figs/run_37/run_37_rr_contour}
\caption{Contour plot of $\overline{v_{r}^{\prime} v_{r}^{\prime}}$ at $z/c$=8.50, $V_{free}$=27.09, station 4.}
\end{figure}


\begin{figure}[H]
\centering
\includegraphics[width=4.25in]{figs/run_37/run_37_tt_contour}
\caption{Contour plot of $\overline{v_{\theta}^{\prime} v_{\theta}^{\prime}}$ at $z/c$=8.50, $V_{free}$=27.09, station 4.}
\end{figure}


\begin{figure}[H]
\centering
\includegraphics[width=4.25in]{figs/run_37/run_37_ww_contour}
\caption{Contour plot of $\overline{v_{z}^{\prime} v_{z}^{\prime}}$ at $z/c$=8.50, $V_{free}$=27.09, station 4.}
\end{figure}


\begin{figure}[H]
\centering
\includegraphics[width=4.25in]{figs/run_37/run_37_ctke_contour}
\caption{Contour plot of $k$ at $z/c$=8.50, $V_{free}$=27.09, station 4.}
\end{figure}


\begin{figure}[H]
\centering
\includegraphics[width=4.25in]{figs/run_37/run_37_num_contour}
\caption{Contour plot of $N$ at $z/c$=8.50, $V_{free}$=27.09, station 4.}
\end{figure}


\begin{figure}[H]
\centering
\includegraphics[width=6in]{figs/run_37/run_37_T_vs_r_meshscatter}
\caption{Scatter plot of azimuthal velocity vs radius at $z/c$=8.50, $V_{free}$=27.09, station 4.}
\end{figure}


\begin{figure}[H]
\centering
\includegraphics[width=6in]{figs/run_37/run_37_T_vs_r_mesh_ctkescatter}
\caption{Scatter plot of azimuthal velocity vs radius, turbulent kinetic energy, at $z/c$=8.50, $V_{free}$=27.09, station 4.}
\end{figure}


\begin{figure}[H]
\centering
\includegraphics[width=6in]{figs/run_37/run_37_ctke_vs_r_meshscatter}
\caption{Scatter plot of turbulent kinetic energy vs radius at $z/c$=8.50, $V_{free}$=27.09, station 4.}
\end{figure}


\begin{figure}[H]
\centering
\includegraphics[width=6in]{figs/run_37/run_37_turb_visc_reynolds_vs_r_meshscatter}
\caption{Scatter plot of $
u_T$ reynolds stress term vs radius at $z/c$=8.50, $V_{free}$=27.09, station 4.}
\end{figure}


\begin{figure}[H]
\centering
\includegraphics[width=6in]{figs/run_37/run_37_turb_visc_vel_grad_vs_r_meshscatter}
\caption{Scatter plot of $\nu_T$ velocity gradient term vs radius at $z/c$=8.50, $V_{free}$=27.09, station 4.}
\end{figure}


\begin{figure}[H]
\centering
\includegraphics[width=6in]{figs/run_37/run_37_turb_visc_ettap_vs_r_meshscatter}
\caption{Scatter plot of $\nu_T$ pressure relaxation term vs radius at $z/c$=8.50, $V_{free}$=27.09, station 4.}
\end{figure}


\begin{figure}[H]
\centering
\includegraphics[width=6in]{figs/run_37/run_37_turb_visc_total_vs_r_meshscatter}
\caption{Scatter plot of non-equilibrium based $\nu_T$ vs radius at $z/c$=8.50, $V_{free}$=27.09, station 4.}
\end{figure}


\begin{figure}[H]
\centering
\includegraphics[width=6in]{figs/run_37/run_37_turb_visc_vs_r_meshscatter}
\caption{Scatter plot of $\nu_T$ vs radius as calculated from the turbulent viscosity hypothesis. $z/c$=8.50, $V_{free}$=27.09, station 4.}
\end{figure}


\begin{figure}[H]
\centering
\includegraphics[width=4.25in]{figs/run_37/run_37_dPdr_contour}
\caption{Diverging contour plot of $\frac{d\bar{P}}{dr}$ from non-equilibrium theory. $z/c$=8.50, $V_{free}$=27.09, station 4.}
\end{figure}



\input{figs/tex/run_38}
\input{figs/tex/run_39}
\input{figs/tex/run_40}

\section{Station 5 PIV Data: 914mm Downstream}
\begin{table}[H]
\begin{center}
\begin{tabular}{|ccccccccccc|}
	\hline
	Run & $I_Z$ & $V_{nom}$ & $dt$ & $V_{fs}$ & $\sigma_{V_{fs}}$ & $Q$ & $P_{atm}$ & $T_{tunnel}$ & $\phi$ & $\eta_P$\\
	$ID$ & $(in)$ & $(m/s)$ & $(\mu s)$ & $(m/s)$ & $(m/s)$ & $(Pa)$ & $(Pa)$ & $(\degree K)$ & $(\%)$ & $(\mu s)$\\
	\hline
	41 & 914 & 15 & 40 & 14.88 & 0.02 & 132 & 101815 & 296.25 & 57.5 & 0.386\\
	42 & 914 & 17 & 40 & 17.24 & 0.03 & 180 & 101812 & 296.35 & 55.8 & 0.398\\
	43 & 914 & 19 & 40 & 19.08 & 0.03 & 217 & 101812 & 294.45 & 55.8 & 0.398\\
	44 & 914 & 21 & 40 & 21.18 & 0.03 & 267 & 101816 & 296.65 & 55.8 & 0.398\\
	45 & 914 & 23 & 40 & 23.24 & 0.03 & 323 & 101809 & 296.95 & 55.8 & 0.398\\
	46 & 914 & 25 & 25 & 24.9 & 0.03 & 371 & 101802 & 297.15 & 55.8 & 0.398\\
	47 & 914 & 27 & 25 & 27.08 & 0.04 & 435 & 101788 & 297.45 & 55.8 & 0.398\\
	48 & 914 & 29 & 25 & 29.19 & 0.03 & 506 & 101784 & 297.85 & 55.8 & 0.398\\
	49 & 914 & 31 & 25 & 31.28 & 0.05 & 584 & 101786 & 298.15 & 56.1 & 0.393\\
	50 & 914 & 33 & 25 & 33.05 & 0.05 & 645 & 101789 & 298.75 & 56.1 & 0.393\\
	\hline
\end{tabular}
\caption{Experimental conditions for tests at station 5}
\label{table:station_5_measurements}
\end{center}
\end{table}

\begin{figure}[H]
\centering
\includegraphics[width=6in]{figs/run_41/run_41_comparison}
\caption{Theoretical vortex profile fits to experimental data at $z/c$=9.00, $V_{free}$=14.88, station 5.}
\end{figure}


\begin{figure}[H]
\centering
\includegraphics[width=4.25in]{figs/run_41/run_41_stream}
\caption{Stream plot at $z/c$=9.00, $V_{free}$=14.88, station 5.}
\end{figure}


\begin{figure}[H]
\centering
\includegraphics[width=4.25in]{figs/run_41/run_41_R_contour}
\caption{Contour plot of $\overline{v_{r}}$ at $z/c$=9.00, $V_{free}$=14.88, station 5.}
\end{figure}


\begin{figure}[H]
\centering
\includegraphics[width=4.25in]{figs/run_41/run_41_T_contour}
\caption{Contour plot of $\overline{v_{\theta}}$ at $z/c$=9.00, $V_{free}$=14.88, station 5.}
\end{figure}


\begin{figure}[H]
\centering
\includegraphics[width=4.25in]{figs/run_41/run_41_W_contour}
\caption{Contour plot of $\overline{v_{z}}$ at $z/c$=9.00, $V_{free}$=14.88, station 5.}
\end{figure}


\begin{figure}[H]
\centering
\includegraphics[width=4.25in]{figs/run_41/run_41_rt_contour}
\caption{Contour plot of $\overline{v_{r}^{\prime} v_{\theta}^{\prime}}$ at $z/c$=9.00, $V_{free}$=14.88, station 5.}
\end{figure}


\begin{figure}[H]
\centering
\includegraphics[width=4.25in]{figs/run_41/run_41_rw_contour}
\caption{Contour plot of $\overline{v_{r}^{\prime} v_{z}^{\prime}}$ at $z/c$=9.00, $V_{free}$=14.88, station 5.}
\end{figure}


\begin{figure}[H]
\centering
\includegraphics[width=4.25in]{figs/run_41/run_41_tw_contour}
\caption{Contour plot of $\overline{v_{\theta}^{\prime} v_{z}^{\prime}}$ at $z/c$=9.00, $V_{free}$=14.88, station 5.}
\end{figure}


\begin{figure}[H]
\centering
\includegraphics[width=4.25in]{figs/run_41/run_41_rr_contour}
\caption{Contour plot of $\overline{v_{r}^{\prime} v_{r}^{\prime}}$ at $z/c$=9.00, $V_{free}$=14.88, station 5.}
\end{figure}


\begin{figure}[H]
\centering
\includegraphics[width=4.25in]{figs/run_41/run_41_tt_contour}
\caption{Contour plot of $\overline{v_{\theta}^{\prime} v_{\theta}^{\prime}}$ at $z/c$=9.00, $V_{free}$=14.88, station 5.}
\end{figure}


\begin{figure}[H]
\centering
\includegraphics[width=4.25in]{figs/run_41/run_41_ww_contour}
\caption{Contour plot of $\overline{v_{z}^{\prime} v_{z}^{\prime}}$ at $z/c$=9.00, $V_{free}$=14.88, station 5.}
\end{figure}


\begin{figure}[H]
\centering
\includegraphics[width=4.25in]{figs/run_41/run_41_ctke_contour}
\caption{Contour plot of $k$ at $z/c$=9.00, $V_{free}$=14.88, station 5.}
\end{figure}


\begin{figure}[H]
\centering
\includegraphics[width=4.25in]{figs/run_41/run_41_num_contour}
\caption{Contour plot of $N$ at $z/c$=9.00, $V_{free}$=14.88, station 5.}
\end{figure}


\begin{figure}[H]
\centering
\includegraphics[width=6in]{figs/run_41/run_41_T_vs_r_meshscatter}
\caption{Scatter plot of azimuthal velocity vs radius at $z/c$=9.00, $V_{free}$=14.88, station 5.}
\end{figure}


\begin{figure}[H]
\centering
\includegraphics[width=6in]{figs/run_41/run_41_T_vs_r_mesh_ctkescatter}
\caption{Scatter plot of azimuthal velocity vs radius, turbulent kinetic energy, at $z/c$=9.00, $V_{free}$=14.88, station 5.}
\end{figure}


\begin{figure}[H]
\centering
\includegraphics[width=6in]{figs/run_41/run_41_ctke_vs_r_meshscatter}
\caption{Scatter plot of turbulent kinetic energy vs radius at $z/c$=9.00, $V_{free}$=14.88, station 5.}
\end{figure}


\begin{figure}[H]
\centering
\includegraphics[width=6in]{figs/run_41/run_41_turb_visc_reynolds_vs_r_meshscatter}
\caption{Scatter plot of $
u_T$ reynolds stress term vs radius at $z/c$=9.00, $V_{free}$=14.88, station 5.}
\end{figure}


\begin{figure}[H]
\centering
\includegraphics[width=6in]{figs/run_41/run_41_turb_visc_vel_grad_vs_r_meshscatter}
\caption{Scatter plot of $\nu_T$ velocity gradient term vs radius at $z/c$=9.00, $V_{free}$=14.88, station 5.}
\end{figure}


\begin{figure}[H]
\centering
\includegraphics[width=6in]{figs/run_41/run_41_momentum_vel_grad_vs_r_meshscatter}
\caption{Scatter plot of the momentum based velocity gradient term vs radius at $z/c$=9.00, $V_{free}$=14.88, station 5.}
\end{figure}


\begin{figure}[H]
\centering
\includegraphics[width=6in]{figs/run_41/run_41_turb_visc_ettap_vs_r_meshscatter}
\caption{Scatter plot of $\nu_T$ pressure relaxation term vs radius at $z/c$=9.00, $V_{free}$=14.88, station 5.}
\end{figure}


\begin{figure}[H]
\centering
\includegraphics[width=6in]{figs/run_41/run_41_turb_visc_total_vs_r_meshscatter}
\caption{Scatter plot of non-equilibrium based $\nu_T$ vs radius at $z/c$=9.00, $V_{free}$=14.88, station 5.}
\end{figure}


\begin{figure}[H]
\centering
\includegraphics[width=6in]{figs/run_41/run_41_turb_visc_vs_r_meshscatter}
\caption{Scatter plot of $\nu_T$ vs radius as calculated from the turbulent viscosity hypothesis. $z/c$=9.00, $V_{free}$=14.88, station 5.}
\end{figure}


\begin{figure}[H]
\centering
\includegraphics[width=4.25in]{figs/run_41/run_41_dPdr_contour}
\caption{Diverging contour plot of $\frac{d\bar{P}}{dr}$ from non-equilibrium theory. $z/c$=9.00, $V_{free}$=14.88, station 5.}
\end{figure}



\begin{figure}[H]
\centering
\includegraphics[width=6in]{figs/run_42/run_42_comparison}
\caption{Theoretical vortex profile fits to experimental data at $z/c$=9.00, $V_{free}$=17.24, station 5.}
\end{figure}


\begin{figure}[H]
\centering
\includegraphics[width=4.25in]{figs/run_42/run_42_stream}
\caption{Stream plot at $z/c$=9.00, $V_{free}$=17.24, station 5.}
\end{figure}


\begin{figure}[H]
\centering
\includegraphics[width=4.25in]{figs/run_42/run_42_R_contour}
\caption{Contour plot of $\overline{r}$ at $z/c$=9.00, $V_{free}$=17.24, station 5.}
\end{figure}


\begin{figure}[H]
\centering
\includegraphics[width=4.25in]{figs/run_42/run_42_T_contour}
\caption{Contour plot of $\overline{t}$ at $z/c$=9.00, $V_{free}$=17.24, station 5.}
\end{figure}


\begin{figure}[H]
\centering
\includegraphics[width=4.25in]{figs/run_42/run_42_W_contour}
\caption{Contour plot of $\overline{w}$ at $z/c$=9.00, $V_{free}$=17.24, station 5.}
\end{figure}


\begin{figure}[H]
\centering
\includegraphics[width=4.25in]{figs/run_42/run_42_rt_contour}
\caption{Contour plot of $\overline{r^\prime t^\prime}$ at $z/c$=9.00, $V_{free}$=17.24, station 5.}
\end{figure}


\begin{figure}[H]
\centering
\includegraphics[width=4.25in]{figs/run_42/run_42_rw_contour}
\caption{Contour plot of $\overline{r^\prime w^\prime}$ at $z/c$=9.00, $V_{free}$=17.24, station 5.}
\end{figure}


\begin{figure}[H]
\centering
\includegraphics[width=4.25in]{figs/run_42/run_42_tw_contour}
\caption{Contour plot of $\overline{t^\prime w^\prime}$ at $z/c$=9.00, $V_{free}$=17.24, station 5.}
\end{figure}


\begin{figure}[H]
\centering
\includegraphics[width=4.25in]{figs/run_42/run_42_rr_contour}
\caption{Contour plot of $\overline{r^\prime r^\prime}$ at $z/c$=9.00, $V_{free}$=17.24, station 5.}
\end{figure}


\begin{figure}[H]
\centering
\includegraphics[width=4.25in]{figs/run_42/run_42_tt_contour}
\caption{Contour plot of $\overline{t^\prime t^\prime}$ at $z/c$=9.00, $V_{free}$=17.24, station 5.}
\end{figure}


\begin{figure}[H]
\centering
\includegraphics[width=4.25in]{figs/run_42/run_42_ww_contour}
\caption{Contour plot of $\overline{w^\prime w^\prime}$ at $z/c$=9.00, $V_{free}$=17.24, station 5.}
\end{figure}


\begin{figure}[H]
\centering
\includegraphics[width=4.25in]{figs/run_42/run_42_ctke_contour}
\caption{Contour plot of $k$ at $z/c$=9.00, $V_{free}$=17.24, station 5.}
\end{figure}


\begin{figure}[H]
\centering
\includegraphics[width=4.25in]{figs/run_42/run_42_num_contour}
\caption{Contour plot of $N$ at $z/c$=9.00, $V_{free}$=17.24, station 5.}
\end{figure}


\begin{figure}[H]
\centering
\includegraphics[width=6in]{figs/run_42/run_42_T_vs_r_meshscatter}
\caption{Scatter plot of azimuthal velocity vs radius at $z/c$=9.00, $V_{free}$=17.24, station5.}
\end{figure}


\begin{figure}[H]
\centering
\includegraphics[width=6in]{figs/run_42/run_42_T_vs_r_mesh_ctkescatter}
\caption{Scatter plot of azimuthal velocity vs radius, turbulent kinetic energy, at $z/c$=9.00, $V_{free}$=17.24, station 5.}
\end{figure}


\begin{figure}[H]
\centering
\includegraphics[width=6in]{figs/run_42/run_42_ctke_vs_r_meshscatter}
\caption{Scatter plot of turbulent kinetic energy vs radius at $z/c$=9.00, $V_{free}$=17.24, station 5.}
\end{figure}


\begin{figure}[H]
\centering
\includegraphics[width=6in]{figs/run_42/run_42_turb_visc_ettap_top_vs_r_meshscatter}
\caption{Scatter plot of $
u_T$ reynolds stress term vs radius at $z/c$=9.00, $V_{free}$=17.24, station 5.}
\end{figure}


\begin{figure}[H]
\centering
\includegraphics[width=6in]{figs/run_42/run_42_turb_visc_ettap_bot_vs_r_meshscatter}
\caption{Scatter plot of $\nu_T$ velocity gradient term vs radius at $z/c$=9.00, $V_{free}$=17.24, station 5.}
\end{figure}


\begin{figure}[H]
\centering
\includegraphics[width=6in]{figs/run_42/run_42_turb_visc_ettap_vs_r_meshscatter}
\caption{Scatter plot of $\nu_T$ pressure relaxation term vs radius at $z/c$=9.00, $V_{free}$=17.24, station 5.}
\end{figure}


\begin{figure}[H]
\centering
\includegraphics[width=6in]{figs/run_42/run_42_turb_visc_vs_r_meshscatter}
\caption{Scatter plot of $\nu_T$ as calculated from the turbulent viscosity hypothesis. $z/c$=9.00, $V_{free}$=17.24, station 5.}
\end{figure}


\begin{figure}[H]
\centering
\includegraphics[width=6in]{figs/run_42/run_42_ctke_01rdynamic}
\caption{Plot showing dynamic variations in turbulent kinetic energy within the core. $z/c$=9.00, $V_{free}$=17.24, station 5.}
\end{figure}


\begin{figure}[H]
\centering
\includegraphics[width=6in]{figs/run_42/run_42_ctke_12rdynamic}
\caption{Plot showing dynamic variations in turbulent kinetic energy between 1 and 2 core radii. $z/c$=9.00, $V_{free}$=17.24, station 5.}
\end{figure}


\begin{figure}[H]
\centering
\includegraphics[width=6in]{figs/run_42/run_42_ctke_005rdynamic}
\caption{Plot showing dynamic variations in turbulent kinetic energy between 0 and 0.5 core radii. $z/c$=9.00, $V_{free}$=17.24, station 5.}
\end{figure}



\begin{figure}[H]
\centering
\includegraphics[width=6in]{figs/run_43/run_43_comparison}
\caption{Theoretical vortex profile fits to experimental data at $z/c$=9.00, $V_{free}$=19.08, station 5.}
\end{figure}


\begin{figure}[H]
\centering
\includegraphics[width=4.25in]{figs/run_43/run_43_stream}
\caption{Stream plot at $z/c$=9.00, $V_{free}$=19.08, station 5.}
\end{figure}


\begin{figure}[H]
\centering
\includegraphics[width=4.25in]{figs/run_43/run_43_R_contour}
\caption{Contour plot of $\overline{r}$ at $z/c$=9.00, $V_{free}$=19.08, station 5.}
\end{figure}


\begin{figure}[H]
\centering
\includegraphics[width=4.25in]{figs/run_43/run_43_T_contour}
\caption{Contour plot of $\overline{t}$ at $z/c$=9.00, $V_{free}$=19.08, station 5.}
\end{figure}


\begin{figure}[H]
\centering
\includegraphics[width=4.25in]{figs/run_43/run_43_W_contour}
\caption{Contour plot of $\overline{w}$ at $z/c$=9.00, $V_{free}$=19.08, station 5.}
\end{figure}


\begin{figure}[H]
\centering
\includegraphics[width=4.25in]{figs/run_43/run_43_rt_contour}
\caption{Contour plot of $\overline{r^\prime t^\prime}$ at $z/c$=9.00, $V_{free}$=19.08, station 5.}
\end{figure}


\begin{figure}[H]
\centering
\includegraphics[width=4.25in]{figs/run_43/run_43_rw_contour}
\caption{Contour plot of $\overline{r^\prime w^\prime}$ at $z/c$=9.00, $V_{free}$=19.08, station 5.}
\end{figure}


\begin{figure}[H]
\centering
\includegraphics[width=4.25in]{figs/run_43/run_43_tw_contour}
\caption{Contour plot of $\overline{t^\prime w^\prime}$ at $z/c$=9.00, $V_{free}$=19.08, station 5.}
\end{figure}


\begin{figure}[H]
\centering
\includegraphics[width=4.25in]{figs/run_43/run_43_rr_contour}
\caption{Contour plot of $\overline{r^\prime r^\prime}$ at $z/c$=9.00, $V_{free}$=19.08, station 5.}
\end{figure}


\begin{figure}[H]
\centering
\includegraphics[width=4.25in]{figs/run_43/run_43_tt_contour}
\caption{Contour plot of $\overline{t^\prime t^\prime}$ at $z/c$=9.00, $V_{free}$=19.08, station 5.}
\end{figure}


\begin{figure}[H]
\centering
\includegraphics[width=4.25in]{figs/run_43/run_43_ww_contour}
\caption{Contour plot of $\overline{w^\prime w^\prime}$ at $z/c$=9.00, $V_{free}$=19.08, station 5.}
\end{figure}


\begin{figure}[H]
\centering
\includegraphics[width=4.25in]{figs/run_43/run_43_ctke_contour}
\caption{Contour plot of $k$ at $z/c$=9.00, $V_{free}$=19.08, station 5.}
\end{figure}


\begin{figure}[H]
\centering
\includegraphics[width=4.25in]{figs/run_43/run_43_num_contour}
\caption{Contour plot of $N$ at $z/c$=9.00, $V_{free}$=19.08, station 5.}
\end{figure}


\begin{figure}[H]
\centering
\includegraphics[width=6in]{figs/run_43/run_43_T_vs_r_meshscatter}
\caption{Scatter plot of azimuthal velocity vs radius at $z/c$=9.00, $V_{free}$=19.08, station5.}
\end{figure}


\begin{figure}[H]
\centering
\includegraphics[width=6in]{figs/run_43/run_43_T_vs_r_mesh_ctkescatter}
\caption{Scatter plot of azimuthal velocity vs radius, turbulent kinetic energy, at $z/c$=9.00, $V_{free}$=19.08, station 5.}
\end{figure}


\begin{figure}[H]
\centering
\includegraphics[width=6in]{figs/run_43/run_43_ctke_vs_r_meshscatter}
\caption{Scatter plot of turbulent kinetic energy vs radius at $z/c$=9.00, $V_{free}$=19.08, station 5.}
\end{figure}


\begin{figure}[H]
\centering
\includegraphics[width=6in]{figs/run_43/run_43_turb_visc_reynolds_vs_r_meshscatter}
\caption{Scatter plot of $
u_T$ reynolds stress term vs radius at $z/c$=9.00, $V_{free}$=19.08, station 5.}
\end{figure}


\begin{figure}[H]
\centering
\includegraphics[width=6in]{figs/run_43/run_43_turb_visc_vel_grad_vs_r_meshscatter}
\caption{Scatter plot of $\nu_T$ velocity gradient term vs radius at $z/c$=9.00, $V_{free}$=19.08, station 5.}
\end{figure}


\begin{figure}[H]
\centering
\includegraphics[width=6in]{figs/run_43/run_43_turb_visc_ettap_vs_r_meshscatter}
\caption{Scatter plot of $\nu_T$ pressure relaxation term vs radius at $z/c$=9.00, $V_{free}$=19.08, station 5.}
\end{figure}


\begin{figure}[H]
\centering
\includegraphics[width=6in]{figs/run_43/run_43_turb_visc_total_vs_r_meshscatter}
\caption{Scatter plot of non-equilibrium based $\nu_T$ vs radius at $z/c$=9.00, $V_{free}$=19.08, station 5.}
\end{figure}


\begin{figure}[H]
\centering
\includegraphics[width=6in]{figs/run_43/run_43_turb_visc_vs_r_meshscatter}
\caption{Scatter plot of $\nu_T$ vs radius as calculated from the turbulent viscosity hypothesis. $z/c$=9.00, $V_{free}$=19.08, station 5.}
\end{figure}


\begin{figure}[H]
\centering
\includegraphics[width=6in]{figs/run_43/run_43_ctke_01rdynamic}
\caption{Plot showing dynamic variations in turbulent kinetic energy within the core. $z/c$=9.00, $V_{free}$=19.08, station 5.}
\end{figure}


\begin{figure}[H]
\centering
\includegraphics[width=6in]{figs/run_43/run_43_ctke_12rdynamic}
\caption{Plot showing dynamic variations in turbulent kinetic energy between 1 and 2 core radii. $z/c$=9.00, $V_{free}$=19.08, station 5.}
\end{figure}


\begin{figure}[H]
\centering
\includegraphics[width=6in]{figs/run_43/run_43_ctke_005rdynamic}
\caption{Plot showing dynamic variations in turbulent kinetic energy between 0 and 0.5 core radii. $z/c$=9.00, $V_{free}$=19.08, station 5.}
\end{figure}



\begin{figure}[H]
\centering
\includegraphics[width=6in]{figs/run_44/run_44_comparison}
\caption{Theoretical vortex profile fits to experimental data at $z/c$=9.00, $V_{free}$=21.18, station 5.}
\end{figure}


\begin{figure}[H]
\centering
\includegraphics[width=4.25in]{figs/run_44/run_44_stream}
\caption{Stream plot at $z/c$=9.00, $V_{free}$=21.18, station 5.}
\end{figure}


\begin{figure}[H]
\centering
\includegraphics[width=4.25in]{figs/run_44/run_44_R_contour}
\caption{Contour plot of $\overline{r}$ at $z/c$=9.00, $V_{free}$=21.18, station 5.}
\end{figure}


\begin{figure}[H]
\centering
\includegraphics[width=4.25in]{figs/run_44/run_44_T_contour}
\caption{Contour plot of $\overline{t}$ at $z/c$=9.00, $V_{free}$=21.18, station 5.}
\end{figure}


\begin{figure}[H]
\centering
\includegraphics[width=4.25in]{figs/run_44/run_44_W_contour}
\caption{Contour plot of $\overline{w}$ at $z/c$=9.00, $V_{free}$=21.18, station 5.}
\end{figure}


\begin{figure}[H]
\centering
\includegraphics[width=4.25in]{figs/run_44/run_44_rt_contour}
\caption{Contour plot of $\overline{r^\prime t^\prime}$ at $z/c$=9.00, $V_{free}$=21.18, station 5.}
\end{figure}


\begin{figure}[H]
\centering
\includegraphics[width=4.25in]{figs/run_44/run_44_rw_contour}
\caption{Contour plot of $\overline{r^\prime w^\prime}$ at $z/c$=9.00, $V_{free}$=21.18, station 5.}
\end{figure}


\begin{figure}[H]
\centering
\includegraphics[width=4.25in]{figs/run_44/run_44_tw_contour}
\caption{Contour plot of $\overline{t^\prime w^\prime}$ at $z/c$=9.00, $V_{free}$=21.18, station 5.}
\end{figure}


\begin{figure}[H]
\centering
\includegraphics[width=4.25in]{figs/run_44/run_44_rr_contour}
\caption{Contour plot of $\overline{r^\prime r^\prime}$ at $z/c$=9.00, $V_{free}$=21.18, station 5.}
\end{figure}


\begin{figure}[H]
\centering
\includegraphics[width=4.25in]{figs/run_44/run_44_tt_contour}
\caption{Contour plot of $\overline{t^\prime t^\prime}$ at $z/c$=9.00, $V_{free}$=21.18, station 5.}
\end{figure}


\begin{figure}[H]
\centering
\includegraphics[width=4.25in]{figs/run_44/run_44_ww_contour}
\caption{Contour plot of $\overline{w^\prime w^\prime}$ at $z/c$=9.00, $V_{free}$=21.18, station 5.}
\end{figure}


\begin{figure}[H]
\centering
\includegraphics[width=4.25in]{figs/run_44/run_44_ctke_contour}
\caption{Contour plot of $k$ at $z/c$=9.00, $V_{free}$=21.18, station 5.}
\end{figure}


\begin{figure}[H]
\centering
\includegraphics[width=4.25in]{figs/run_44/run_44_num_contour}
\caption{Contour plot of $N$ at $z/c$=9.00, $V_{free}$=21.18, station 5.}
\end{figure}


\begin{figure}[H]
\centering
\includegraphics[width=6in]{figs/run_44/run_44_T_vs_r_meshscatter}
\caption{Scatter plot of azimuthal velocity vs radius at $z/c$=9.00, $V_{free}$=21.18, station5.}
\end{figure}


\begin{figure}[H]
\centering
\includegraphics[width=6in]{figs/run_44/run_44_T_vs_r_mesh_ctkescatter}
\caption{Scatter plot of azimuthal velocity vs radius, turbulent kinetic energy, at $z/c$=9.00, $V_{free}$=21.18, station 5.}
\end{figure}


\begin{figure}[H]
\centering
\includegraphics[width=6in]{figs/run_44/run_44_ctke_vs_r_meshscatter}
\caption{Scatter plot of turbulent kinetic energy vs radius at $z/c$=9.00, $V_{free}$=21.18, station 5.}
\end{figure}


\begin{figure}[H]
\centering
\includegraphics[width=6in]{figs/run_44/run_44_turb_visc_reynolds_vs_r_meshscatter}
\caption{Scatter plot of $
u_T$ reynolds stress term vs radius at $z/c$=9.00, $V_{free}$=21.18, station 5.}
\end{figure}


\begin{figure}[H]
\centering
\includegraphics[width=6in]{figs/run_44/run_44_turb_visc_vel_grad_vs_r_meshscatter}
\caption{Scatter plot of $\nu_T$ velocity gradient term vs radius at $z/c$=9.00, $V_{free}$=21.18, station 5.}
\end{figure}


\begin{figure}[H]
\centering
\includegraphics[width=6in]{figs/run_44/run_44_turb_visc_ettap_vs_r_meshscatter}
\caption{Scatter plot of $\nu_T$ pressure relaxation term vs radius at $z/c$=9.00, $V_{free}$=21.18, station 5.}
\end{figure}


\begin{figure}[H]
\centering
\includegraphics[width=6in]{figs/run_44/run_44_turb_visc_total_vs_r_meshscatter}
\caption{Scatter plot of non-equilibrium based $\nu_T$ vs radius at $z/c$=9.00, $V_{free}$=21.18, station 5.}
\end{figure}


\begin{figure}[H]
\centering
\includegraphics[width=6in]{figs/run_44/run_44_turb_visc_vs_r_meshscatter}
\caption{Scatter plot of $\nu_T$ vs radius as calculated from the turbulent viscosity hypothesis. $z/c$=9.00, $V_{free}$=21.18, station 5.}
\end{figure}


\begin{figure}[H]
\centering
\includegraphics[width=6in]{figs/run_44/run_44_ctke_01rdynamic}
\caption{Plot showing dynamic variations in turbulent kinetic energy within the core. $z/c$=9.00, $V_{free}$=21.18, station 5.}
\end{figure}


\begin{figure}[H]
\centering
\includegraphics[width=6in]{figs/run_44/run_44_ctke_12rdynamic}
\caption{Plot showing dynamic variations in turbulent kinetic energy between 1 and 2 core radii. $z/c$=9.00, $V_{free}$=21.18, station 5.}
\end{figure}


\begin{figure}[H]
\centering
\includegraphics[width=6in]{figs/run_44/run_44_ctke_005rdynamic}
\caption{Plot showing dynamic variations in turbulent kinetic energy between 0 and 0.5 core radii. $z/c$=9.00, $V_{free}$=21.18, station 5.}
\end{figure}



\input{figs/tex/run_45}
\input{figs/tex/run_46}
\begin{figure}[H]
\centering
\includegraphics[width=6in]{figs/run_47/run_47_comparison}
\caption{Theoretical vortex profile fits to experimental data at $z/c$=9.00, $V_{free}$=27.08, station 5.}
\end{figure}


\begin{figure}[H]
\centering
\includegraphics[width=4.25in]{figs/run_47/run_47_stream}
\caption{Stream plot at $z/c$=9.00, $V_{free}$=27.08, station 5.}
\end{figure}


\begin{figure}[H]
\centering
\includegraphics[width=4.25in]{figs/run_47/run_47_R_contour}
\caption{Contour plot of $\overline{r}$ at $z/c$=9.00, $V_{free}$=27.08, station 5.}
\end{figure}


\begin{figure}[H]
\centering
\includegraphics[width=4.25in]{figs/run_47/run_47_T_contour}
\caption{Contour plot of $\overline{t}$ at $z/c$=9.00, $V_{free}$=27.08, station 5.}
\end{figure}


\begin{figure}[H]
\centering
\includegraphics[width=4.25in]{figs/run_47/run_47_W_contour}
\caption{Contour plot of $\overline{w}$ at $z/c$=9.00, $V_{free}$=27.08, station 5.}
\end{figure}


\begin{figure}[H]
\centering
\includegraphics[width=4.25in]{figs/run_47/run_47_rt_contour}
\caption{Contour plot of $\overline{r^\prime t^\prime}$ at $z/c$=9.00, $V_{free}$=27.08, station 5.}
\end{figure}


\begin{figure}[H]
\centering
\includegraphics[width=4.25in]{figs/run_47/run_47_rw_contour}
\caption{Contour plot of $\overline{r^\prime w^\prime}$ at $z/c$=9.00, $V_{free}$=27.08, station 5.}
\end{figure}


\begin{figure}[H]
\centering
\includegraphics[width=4.25in]{figs/run_47/run_47_tw_contour}
\caption{Contour plot of $\overline{t^\prime w^\prime}$ at $z/c$=9.00, $V_{free}$=27.08, station 5.}
\end{figure}


\begin{figure}[H]
\centering
\includegraphics[width=4.25in]{figs/run_47/run_47_rr_contour}
\caption{Contour plot of $\overline{r^\prime r^\prime}$ at $z/c$=9.00, $V_{free}$=27.08, station 5.}
\end{figure}


\begin{figure}[H]
\centering
\includegraphics[width=4.25in]{figs/run_47/run_47_tt_contour}
\caption{Contour plot of $\overline{t^\prime t^\prime}$ at $z/c$=9.00, $V_{free}$=27.08, station 5.}
\end{figure}


\begin{figure}[H]
\centering
\includegraphics[width=4.25in]{figs/run_47/run_47_ww_contour}
\caption{Contour plot of $\overline{w^\prime w^\prime}$ at $z/c$=9.00, $V_{free}$=27.08, station 5.}
\end{figure}


\begin{figure}[H]
\centering
\includegraphics[width=4.25in]{figs/run_47/run_47_ctke_contour}
\caption{Contour plot of $k$ at $z/c$=9.00, $V_{free}$=27.08, station 5.}
\end{figure}


\begin{figure}[H]
\centering
\includegraphics[width=4.25in]{figs/run_47/run_47_num_contour}
\caption{Contour plot of $N$ at $z/c$=9.00, $V_{free}$=27.08, station 5.}
\end{figure}


\begin{figure}[H]
\centering
\includegraphics[width=6in]{figs/run_47/run_47_T_vs_r_meshscatter}
\caption{Scatter plot of azimuthal velocity vs radius at $z/c$=9.00, $V_{free}$=27.08, station5.}
\end{figure}


\begin{figure}[H]
\centering
\includegraphics[width=6in]{figs/run_47/run_47_T_vs_r_mesh_ctkescatter}
\caption{Scatter plot of azimuthal velocity vs radius, turbulent kinetic energy, at $z/c$=9.00, $V_{free}$=27.08, station 5.}
\end{figure}


\begin{figure}[H]
\centering
\includegraphics[width=6in]{figs/run_47/run_47_ctke_vs_r_meshscatter}
\caption{Scatter plot of turbulent kinetic energy vs radius at $z/c$=9.00, $V_{free}$=27.08, station 5.}
\end{figure}


\begin{figure}[H]
\centering
\includegraphics[width=6in]{figs/run_47/run_47_turb_visc_reynolds_vs_r_meshscatter}
\caption{Scatter plot of $
u_T$ reynolds stress term vs radius at $z/c$=9.00, $V_{free}$=27.08, station 5.}
\end{figure}


\begin{figure}[H]
\centering
\includegraphics[width=6in]{figs/run_47/run_47_turb_visc_vel_grad_vs_r_meshscatter}
\caption{Scatter plot of $\nu_T$ velocity gradient term vs radius at $z/c$=9.00, $V_{free}$=27.08, station 5.}
\end{figure}


\begin{figure}[H]
\centering
\includegraphics[width=6in]{figs/run_47/run_47_turb_visc_ettap_vs_r_meshscatter}
\caption{Scatter plot of $\nu_T$ pressure relaxation term vs radius at $z/c$=9.00, $V_{free}$=27.08, station 5.}
\end{figure}


\begin{figure}[H]
\centering
\includegraphics[width=6in]{figs/run_47/run_47_turb_visc_total_vs_r_meshscatter}
\caption{Scatter plot of non-equilibrium based $\nu_T$ vs radius at $z/c$=9.00, $V_{free}$=27.08, station 5.}
\end{figure}


\begin{figure}[H]
\centering
\includegraphics[width=6in]{figs/run_47/run_47_turb_visc_vs_r_meshscatter}
\caption{Scatter plot of $\nu_T$ vs radius as calculated from the turbulent viscosity hypothesis. $z/c$=9.00, $V_{free}$=27.08, station 5.}
\end{figure}


\begin{figure}[H]
\centering
\includegraphics[width=6in]{figs/run_47/run_47_ctke_01rdynamic}
\caption{Plot showing dynamic variations in turbulent kinetic energy within the core. $z/c$=9.00, $V_{free}$=27.08, station 5.}
\end{figure}


\begin{figure}[H]
\centering
\includegraphics[width=6in]{figs/run_47/run_47_ctke_12rdynamic}
\caption{Plot showing dynamic variations in turbulent kinetic energy between 1 and 2 core radii. $z/c$=9.00, $V_{free}$=27.08, station 5.}
\end{figure}


\begin{figure}[H]
\centering
\includegraphics[width=6in]{figs/run_47/run_47_ctke_005rdynamic}
\caption{Plot showing dynamic variations in turbulent kinetic energy between 0 and 0.5 core radii. $z/c$=9.00, $V_{free}$=27.08, station 5.}
\end{figure}



\input{figs/tex/run_48}
\begin{figure}[H]
\centering
\includegraphics[width=6in]{figs/run_49/run_49_comparison}
\caption{Theoretical vortex profile fits to experimental data at $z/c$=9.00, $V_{free}$=31.28, station 5.}
\end{figure}


\begin{figure}[H]
\centering
\includegraphics[width=4.25in]{figs/run_49/run_49_stream}
\caption{Stream plot at $z/c$=9.00, $V_{free}$=31.28, station 5.}
\end{figure}


\begin{figure}[H]
\centering
\includegraphics[width=4.25in]{figs/run_49/run_49_R_contour}
\caption{Contour plot of $\overline{v_{r}}$ at $z/c$=9.00, $V_{free}$=31.28, station 5.}
\end{figure}


\begin{figure}[H]
\centering
\includegraphics[width=4.25in]{figs/run_49/run_49_T_contour}
\caption{Contour plot of $\overline{v_{\theta}}$ at $z/c$=9.00, $V_{free}$=31.28, station 5.}
\end{figure}


\begin{figure}[H]
\centering
\includegraphics[width=4.25in]{figs/run_49/run_49_W_contour}
\caption{Contour plot of $\overline{v_{z}}$ at $z/c$=9.00, $V_{free}$=31.28, station 5.}
\end{figure}


\begin{figure}[H]
\centering
\includegraphics[width=4.25in]{figs/run_49/run_49_rt_contour}
\caption{Contour plot of $\overline{v_{r}^{\prime} v_{\theta}^{\prime}}$ at $z/c$=9.00, $V_{free}$=31.28, station 5.}
\end{figure}


\begin{figure}[H]
\centering
\includegraphics[width=4.25in]{figs/run_49/run_49_rw_contour}
\caption{Contour plot of $\overline{v_{r}^{\prime} v_{z}^{\prime}}$ at $z/c$=9.00, $V_{free}$=31.28, station 5.}
\end{figure}


\begin{figure}[H]
\centering
\includegraphics[width=4.25in]{figs/run_49/run_49_tw_contour}
\caption{Contour plot of $\overline{v_{\theta}^{\prime} v_{z}^{\prime}}$ at $z/c$=9.00, $V_{free}$=31.28, station 5.}
\end{figure}


\begin{figure}[H]
\centering
\includegraphics[width=4.25in]{figs/run_49/run_49_rr_contour}
\caption{Contour plot of $\overline{v_{r}^{\prime} v_{r}^{\prime}}$ at $z/c$=9.00, $V_{free}$=31.28, station 5.}
\end{figure}


\begin{figure}[H]
\centering
\includegraphics[width=4.25in]{figs/run_49/run_49_tt_contour}
\caption{Contour plot of $\overline{v_{\theta}^{\prime} v_{\theta}^{\prime}}$ at $z/c$=9.00, $V_{free}$=31.28, station 5.}
\end{figure}


\begin{figure}[H]
\centering
\includegraphics[width=4.25in]{figs/run_49/run_49_ww_contour}
\caption{Contour plot of $\overline{v_{z}^{\prime} v_{z}^{\prime}}$ at $z/c$=9.00, $V_{free}$=31.28, station 5.}
\end{figure}


\begin{figure}[H]
\centering
\includegraphics[width=4.25in]{figs/run_49/run_49_ctke_contour}
\caption{Contour plot of $k$ at $z/c$=9.00, $V_{free}$=31.28, station 5.}
\end{figure}


\begin{figure}[H]
\centering
\includegraphics[width=4.25in]{figs/run_49/run_49_num_contour}
\caption{Contour plot of $N$ at $z/c$=9.00, $V_{free}$=31.28, station 5.}
\end{figure}


\begin{figure}[H]
\centering
\includegraphics[width=6in]{figs/run_49/run_49_T_vs_r_meshscatter}
\caption{Scatter plot of azimuthal velocity vs radius at $z/c$=9.00, $V_{free}$=31.28, station 5.}
\end{figure}


\begin{figure}[H]
\centering
\includegraphics[width=6in]{figs/run_49/run_49_T_vs_r_mesh_ctkescatter}
\caption{Scatter plot of azimuthal velocity vs radius, turbulent kinetic energy, at $z/c$=9.00, $V_{free}$=31.28, station 5.}
\end{figure}


\begin{figure}[H]
\centering
\includegraphics[width=6in]{figs/run_49/run_49_ctke_vs_r_meshscatter}
\caption{Scatter plot of turbulent kinetic energy vs radius at $z/c$=9.00, $V_{free}$=31.28, station 5.}
\end{figure}


\begin{figure}[H]
\centering
\includegraphics[width=6in]{figs/run_49/run_49_turb_visc_reynolds_vs_r_meshscatter}
\caption{Scatter plot of $
u_T$ reynolds stress term vs radius at $z/c$=9.00, $V_{free}$=31.28, station 5.}
\end{figure}


\begin{figure}[H]
\centering
\includegraphics[width=6in]{figs/run_49/run_49_turb_visc_vel_grad_vs_r_meshscatter}
\caption{Scatter plot of $\nu_T$ velocity gradient term vs radius at $z/c$=9.00, $V_{free}$=31.28, station 5.}
\end{figure}


\begin{figure}[H]
\centering
\includegraphics[width=6in]{figs/run_49/run_49_turb_visc_by_vtheta_vs_r_meshscatter}
\caption{Scatter plot of total viscosity $\nu$ calculated by mean $\bar{v}_{\theta}$ equation with non-equilibrium pressure at $z/c$=9.00, $V_{free}$=31.28, station 5.}
\end{figure}


\begin{figure}[H]
\centering
\includegraphics[width=6in]{figs/run_49/run_49_momentum_vel_grad_vs_r_meshscatter}
\caption{Scatter plot of the momentum based velocity gradient term vs radius at $z/c$=9.00, $V_{free}$=31.28, station 5.}
\end{figure}


\begin{figure}[H]
\centering
\includegraphics[width=6in]{figs/run_49/run_49_turb_visc_ettap_vs_r_meshscatter}
\caption{Scatter plot of $\nu_T$ pressure relaxation term vs radius at $z/c$=9.00, $V_{free}$=31.28, station 5.}
\end{figure}


\begin{figure}[H]
\centering
\includegraphics[width=6in]{figs/run_49/run_49_turb_visc_total_vs_r_meshscatter}
\caption{Scatter plot of non-equilibrium based $\nu_T$ vs radius at $z/c$=9.00, $V_{free}$=31.28, station 5.}
\end{figure}


\begin{figure}[H]
\centering
\includegraphics[width=6in]{figs/run_49/run_49_turb_visc_vs_r_meshscatter}
\caption{Scatter plot of $\nu_T$ vs radius as calculated from the turbulent viscosity hypothesis. $z/c$=9.00, $V_{free}$=31.28, station 5.}
\end{figure}


\begin{figure}[H]
\centering
\includegraphics[width=6in]{figs/run_49/run_49_ctke_01rdynamic}
\caption{Plot showing dynamic variations in turbulent kinetic energy within the core. $z/c$=9.00, $V_{free}$=31.28, station 5.}
\end{figure}


\begin{figure}[H]
\centering
\includegraphics[width=6in]{figs/run_49/run_49_ctke_12rdynamic}
\caption{Plot showing dynamic variations in turbulent kinetic energy between 1 and 2 core radii. $z/c$=9.00, $V_{free}$=31.28, station 5.}
\end{figure}


\begin{figure}[H]
\centering
\includegraphics[width=6in]{figs/run_49/run_49_ctke_005rdynamic}
\caption{Plot showing dynamic variations in turbulent kinetic energy between 0 and 0.5 core radii. $z/c$=9.00, $V_{free}$=31.28, station 5.}
\end{figure}



\input{figs/tex/run_50}

\section{Station 6 PIV Data: 965mm Downstream}
\begin{table}[H]
\begin{center}
\begin{tabular}{|ccccccccccc|}
	\hline
	Run & $I_Z$ & $V_{nom}$ & $dt$ & $V_{fs}$ & $\sigma_{V_{fs}}$ & $Q$ & $P_{atm}$ & $T_{tunnel}$ & $\phi$ & $\eta_P$\\
	$ID$ & $(in)$ & $(m/s)$ & $(\mu s)$ & $(m/s)$ & $(m/s)$ & $(Pa)$ & $(Pa)$ & $(\degree K)$ & $(\%)$ & $(\mu s)$\\
	\hline
	51 & 965 & 15 & 40 & 15.31 & 0.02 & 140 & 102039 & 294.85 & 61.2 & 0.381\\
	52 & 965 & 17 & 40 & 17.36 & 0.03 & 182 & 102034 & 294.95 & 61.2 & 0.381\\
	53 & 965 & 19 & 40 & 19.08 & 0.03 & 219 & 102035 & 295.15 & 59.5 & 0.392\\
	54 & 965 & 21 & 40 & 21.23 & 0.03 & 270 & 102037 & 295.35 & 59.5 & 0.392\\
	55 & 965 & 23 & 40 & 23.33 & 0.03 & 327 & 102018 & 295.65 & 59.5 & 0.392\\
	56 & 965 & 25 & 25 & 24.97 & 0.03 & 375 & 102021 & 295.95 & 59.5 & 0.392\\
	57 & 965 & 27 & 25 & 27.05 & 0.06 & 437 & 102008 & 297.85 & 52.6 & 0.422\\
	58 & 965 & 29 & 25 & 29.17 & 0.04 & 505 & 102005 & 298.15 & 47.4 & 0.454\\
	59 & 965 & 31 & 25 & 30.9 & 0.06 & 571 & 102015 & 297.85 & 53.7 & 0.421\\
	60 & 965 & 33 & 25 & 33.04 & 0.06 & 653 & 102009 & 297.35 & 53.7 & 0.421\\
	\hline
\end{tabular}
\caption{Experimental measurements for station 6}
\label{table:station_6_measurements}
\end{center}
\end{table}

\input{figs/tex/run_51}
\input{figs/tex/run_52}
\begin{figure}[H]
\centering
\includegraphics[width=6in]{figs/run_53/run_53_comparison}
\caption{Theoretical vortex profile fits to experimental data at $z/c$=9.50, $V_{free}$=19.08, station 6.}
\label{fig:run_53_comparison}
\end{figure}


\begin{figure}[H]
\centering
\includegraphics[width=4.25in]{figs/run_53/run_53_stream}
\caption{Stream plot at $z/c$=9.50, $V_{free}$=19.08, station 6.}
\label{fig:run_53_stream}
\end{figure}


\begin{figure}[H]
\centering
\includegraphics[width=4.25in]{figs/run_53/run_53_R_contour}
\caption{Contour plot of $\overline{r}$ at $z/c$=9.50, $V_{free}$=19.08, station 6.}
\label{fig:run_53_R_contour}
\end{figure}


\begin{figure}[H]
\centering
\includegraphics[width=4.25in]{figs/run_53/run_53_T_contour}
\caption{Contour plot of $\overline{t}$ at $z/c$=9.50, $V_{free}$=19.08, station 6.}
\label{fig:run_53_T_contour}
\end{figure}


\begin{figure}[H]
\centering
\includegraphics[width=4.25in]{figs/run_53/run_53_W_contour}
\caption{Contour plot of $\overline{w}$ at $z/c$=9.50, $V_{free}$=19.08, station 6.}
\label{fig:run_53_W_contour}
\end{figure}


\begin{figure}[H]
\centering
\includegraphics[width=4.25in]{figs/run_53/run_53_rt_contour}
\caption{Contour plot of $\overline{r^\prime t^\prime}$ at $z/c$=9.50, $V_{free}$=19.08, station 6.}
\label{fig:run_53_rt_contour}
\end{figure}


\begin{figure}[H]
\centering
\includegraphics[width=4.25in]{figs/run_53/run_53_rw_contour}
\caption{Contour plot of $\overline{r^\prime w^\prime}$ at $z/c$=9.50, $V_{free}$=19.08, station 6.}
\label{fig:run_53_rw_contour}
\end{figure}


\begin{figure}[H]
\centering
\includegraphics[width=4.25in]{figs/run_53/run_53_tw_contour}
\caption{Contour plot of $\overline{t^\prime w^\prime}$ at $z/c$=9.50, $V_{free}$=19.08, station 6.}
\label{fig:run_53_tw_contour}
\end{figure}


\begin{figure}[H]
\centering
\includegraphics[width=4.25in]{figs/run_53/run_53_rr_contour}
\caption{Contour plot of $\overline{r^\prime r^\prime}$ at $z/c$=9.50, $V_{free}$=19.08, station 6.}
\label{fig:run_53_rr_contour}
\end{figure}


\begin{figure}[H]
\centering
\includegraphics[width=4.25in]{figs/run_53/run_53_tt_contour}
\caption{Contour plot of $\overline{t^\prime t^\prime}$ at $z/c$=9.50, $V_{free}$=19.08, station 6.}
\label{fig:run_53_tt_contour}
\end{figure}


\begin{figure}[H]
\centering
\includegraphics[width=4.25in]{figs/run_53/run_53_ww_contour}
\caption{Contour plot of $\overline{w^\prime w^\prime}$ at $z/c$=9.50, $V_{free}$=19.08, station 6.}
\label{fig:run_53_ww_contour}
\end{figure}


\begin{figure}[H]
\centering
\includegraphics[width=4.25in]{figs/run_53/run_53_ctke_contour}
\caption{Contour plot of $k$ at $z/c$=9.50, $V_{free}$=19.08, station 6.}
\label{fig:run_53_ctke_contour}
\end{figure}


\begin{figure}[H]
\centering
\includegraphics[width=4.25in]{figs/run_53/run_53_num_contour}
\caption{Contour plot of $N$ at $z/c$=9.50, $V_{free}$=19.08, station 6.}
\label{fig:run_53_num_contour}
\end{figure}


\begin{figure}[H]
\centering
\includegraphics[width=6in]{figs/run_53/run_53_T_vs_r_meshscatter}
\caption{Scatter plot of azimuthal velocity vs radius at $z/c$=9.50, $V_{free}$=19.08, station6.}
\label{fig:run_53_T_vs_r_meshscatter}
\end{figure}


\begin{figure}[H]
\centering
\includegraphics[width=6in]{figs/run_53/run_53_T_vs_r_mesh_ctkescatter}
\caption{Scatter plot of azimuthal velocity vs radius, turbulent kinetic energy, at $z/c$=9.50, $V_{free}$=19.08, station 6.}
\label{fig:run_53_T_vs_r_mesh_ctkescatter}
\end{figure}


\begin{figure}[H]
\centering
\includegraphics[width=6in]{figs/run_53/run_53_ctke_vs_r_meshscatter}
\caption{Scatter plot of turbulent kinetic energy vs radius at $z/c$=9.50, $V_{free}$=19.08, station 6.}
\label{fig:run_53_ctke_vs_r_meshscatter}
\end{figure}


\begin{figure}[H]
\centering
\includegraphics[width=6in]{figs/run_53/run_53_turb_visc_ettap_top_vs_r_meshscatter}
\caption{Scatter plot of $
u_T$ reynolds stress term vs radius at $z/c$=9.50, $V_{free}$=19.08, station 6.}
\label{fig:run_53_turb_visc_ettap_top_vs_r_meshscatter}
\end{figure}


\begin{figure}[H]
\centering
\includegraphics[width=6in]{figs/run_53/run_53_turb_visc_ettap_bot_vs_r_meshscatter}
\caption{Scatter plot of $
u_T$ velocity gradient term vs radius at $z/c$=9.50, $V_{free}$=19.08, station 6.}
\label{fig:run_53_turb_visc_ettap_bot_vs_r_meshscatter}
\end{figure}


\begin{figure}[H]
\centering
\includegraphics[width=6in]{figs/run_53/run_53_turb_visc_ettap_vs_r_meshscatter}
\caption{Scatter plot of $
u_T$ pressure relaxation term vs radius at $z/c$=9.50, $V_{free}$=19.08, station 6.}
\label{fig:run_53_turb_visc_ettap_vs_r_meshscatter}
\end{figure}


\begin{figure}[H]
\centering
\includegraphics[width=6in]{figs/run_53/run_53_ctke_01rdynamic}
\caption{Plot showing dynamic variations in turbulent kinetic energy within the core. $z/c$=9.50, $V_{free}$=19.08, station 6.}
\label{fig:run_53_ctke_01rdynamic}
\end{figure}


\begin{figure}[H]
\centering
\includegraphics[width=6in]{figs/run_53/run_53_ctke_12rdynamic}
\caption{Plot showing dynamic variations in turbulent kinetic energy between 1 and 2 core radii. $z/c$=9.50, $V_{free}$=19.08, station 6.}
\label{fig:run_53_ctke_12rdynamic}
\end{figure}


\begin{figure}[H]
\centering
\includegraphics[width=6in]{figs/run_53/run_53_ctke_005rdynamic}
\caption{Plot showing dynamic variations in turbulent kinetic energy between 0 and 0.5 core radii. $z/c$=9.50, $V_{free}$=19.08, station 6.}
\label{fig:run_53_ctke_005rdynamic}
\end{figure}



\begin{figure}[H]
\centering
\includegraphics[width=6in]{figs/run_54/run_54_comparison}
\caption{Theoretical vortex profile fits to experimental data at $z/c$=9.50, $V_{free}$=21.23, station 6.}
\end{figure}


\begin{figure}[H]
\centering
\includegraphics[width=4.25in]{figs/run_54/run_54_stream}
\caption{Stream plot at $z/c$=9.50, $V_{free}$=21.23, station 6.}
\end{figure}


\begin{figure}[H]
\centering
\includegraphics[width=4.25in]{figs/run_54/run_54_R_contour}
\caption{Contour plot of $\overline{v_{r}}$ at $z/c$=9.50, $V_{free}$=21.23, station 6.}
\end{figure}


\begin{figure}[H]
\centering
\includegraphics[width=4.25in]{figs/run_54/run_54_T_contour}
\caption{Contour plot of $\overline{v_{\theta}}$ at $z/c$=9.50, $V_{free}$=21.23, station 6.}
\end{figure}


\begin{figure}[H]
\centering
\includegraphics[width=4.25in]{figs/run_54/run_54_W_contour}
\caption{Contour plot of $\overline{v_{z}}$ at $z/c$=9.50, $V_{free}$=21.23, station 6.}
\end{figure}


\begin{figure}[H]
\centering
\includegraphics[width=4.25in]{figs/run_54/run_54_rt_contour}
\caption{Contour plot of $\overline{v_{r}^{\prime} v_{\theta}^{\prime}}$ at $z/c$=9.50, $V_{free}$=21.23, station 6.}
\end{figure}


\begin{figure}[H]
\centering
\includegraphics[width=4.25in]{figs/run_54/run_54_rw_contour}
\caption{Contour plot of $\overline{v_{r}^{\prime} v_{z}^{\prime}}$ at $z/c$=9.50, $V_{free}$=21.23, station 6.}
\end{figure}


\begin{figure}[H]
\centering
\includegraphics[width=4.25in]{figs/run_54/run_54_tw_contour}
\caption{Contour plot of $\overline{v_{\theta}^{\prime} v_{z}^{\prime}}$ at $z/c$=9.50, $V_{free}$=21.23, station 6.}
\end{figure}


\begin{figure}[H]
\centering
\includegraphics[width=4.25in]{figs/run_54/run_54_rr_contour}
\caption{Contour plot of $\overline{v_{r}^{\prime} v_{r}^{\prime}}$ at $z/c$=9.50, $V_{free}$=21.23, station 6.}
\end{figure}


\begin{figure}[H]
\centering
\includegraphics[width=4.25in]{figs/run_54/run_54_tt_contour}
\caption{Contour plot of $\overline{v_{\theta}^{\prime} v_{\theta}^{\prime}}$ at $z/c$=9.50, $V_{free}$=21.23, station 6.}
\end{figure}


\begin{figure}[H]
\centering
\includegraphics[width=4.25in]{figs/run_54/run_54_ww_contour}
\caption{Contour plot of $\overline{v_{z}^{\prime} v_{z}^{\prime}}$ at $z/c$=9.50, $V_{free}$=21.23, station 6.}
\end{figure}


\begin{figure}[H]
\centering
\includegraphics[width=4.25in]{figs/run_54/run_54_ctke_contour}
\caption{Contour plot of $k$ at $z/c$=9.50, $V_{free}$=21.23, station 6.}
\end{figure}


\begin{figure}[H]
\centering
\includegraphics[width=4.25in]{figs/run_54/run_54_num_contour}
\caption{Contour plot of $N$ at $z/c$=9.50, $V_{free}$=21.23, station 6.}
\end{figure}


\begin{figure}[H]
\centering
\includegraphics[width=6in]{figs/run_54/run_54_T_vs_r_meshscatter}
\caption{Scatter plot of azimuthal velocity vs radius at $z/c$=9.50, $V_{free}$=21.23, station 6.}
\end{figure}


\begin{figure}[H]
\centering
\includegraphics[width=6in]{figs/run_54/run_54_T_vs_r_mesh_ctkescatter}
\caption{Scatter plot of azimuthal velocity vs radius, turbulent kinetic energy, at $z/c$=9.50, $V_{free}$=21.23, station 6.}
\end{figure}


\begin{figure}[H]
\centering
\includegraphics[width=6in]{figs/run_54/run_54_ctke_vs_r_meshscatter}
\caption{Scatter plot of turbulent kinetic energy vs radius at $z/c$=9.50, $V_{free}$=21.23, station 6.}
\end{figure}


\begin{figure}[H]
\centering
\includegraphics[width=6in]{figs/run_54/run_54_turb_visc_reynolds_vs_r_meshscatter}
\caption{Scatter plot of $
u_T$ reynolds stress term vs radius at $z/c$=9.50, $V_{free}$=21.23, station 6.}
\end{figure}


\begin{figure}[H]
\centering
\includegraphics[width=6in]{figs/run_54/run_54_turb_visc_vel_grad_vs_r_meshscatter}
\caption{Scatter plot of $\nu_T$ velocity gradient term vs radius at $z/c$=9.50, $V_{free}$=21.23, station 6.}
\end{figure}


\begin{figure}[H]
\centering
\includegraphics[width=6in]{figs/run_54/run_54_momentum_vel_grad_vs_r_meshscatter}
\caption{Scatter plot of the momentum based velocity gradient term vs radius at $z/c$=9.50, $V_{free}$=21.23, station 6.}
\end{figure}


\begin{figure}[H]
\centering
\includegraphics[width=6in]{figs/run_54/run_54_turb_visc_ettap_vs_r_meshscatter}
\caption{Scatter plot of $\nu_T$ pressure relaxation term vs radius at $z/c$=9.50, $V_{free}$=21.23, station 6.}
\end{figure}


\begin{figure}[H]
\centering
\includegraphics[width=6in]{figs/run_54/run_54_turb_visc_total_vs_r_meshscatter}
\caption{Scatter plot of non-equilibrium based $\nu_T$ vs radius at $z/c$=9.50, $V_{free}$=21.23, station 6.}
\end{figure}


\begin{figure}[H]
\centering
\includegraphics[width=6in]{figs/run_54/run_54_turb_visc_vs_r_meshscatter}
\caption{Scatter plot of $\nu_T$ vs radius as calculated from the turbulent viscosity hypothesis. $z/c$=9.50, $V_{free}$=21.23, station 6.}
\end{figure}


\begin{figure}[H]
\centering
\includegraphics[width=4.25in]{figs/run_54/run_54_dPdr_contour}
\caption{Diverging contour plot of $\frac{d\bar{P}}{dr}$ from non-equilibrium theory. $z/c$=9.50, $V_{free}$=21.23, station 6.}
\end{figure}



\begin{figure}[H]
\centering
\includegraphics[width=6in]{figs/run_55/run_55_comparison}
\caption{Theoretical vortex profile fits to experimental data at $z/c$=9.50, $V_{free}$=23.33, station 6.}
\end{figure}


\begin{figure}[H]
\centering
\includegraphics[width=4.25in]{figs/run_55/run_55_stream}
\caption{Stream plot at $z/c$=9.50, $V_{free}$=23.33, station 6.}
\end{figure}


\begin{figure}[H]
\centering
\includegraphics[width=4.25in]{figs/run_55/run_55_R_contour}
\caption{Contour plot of $\overline{r}$ at $z/c$=9.50, $V_{free}$=23.33, station 6.}
\end{figure}


\begin{figure}[H]
\centering
\includegraphics[width=4.25in]{figs/run_55/run_55_T_contour}
\caption{Contour plot of $\overline{t}$ at $z/c$=9.50, $V_{free}$=23.33, station 6.}
\end{figure}


\begin{figure}[H]
\centering
\includegraphics[width=4.25in]{figs/run_55/run_55_W_contour}
\caption{Contour plot of $\overline{w}$ at $z/c$=9.50, $V_{free}$=23.33, station 6.}
\end{figure}


\begin{figure}[H]
\centering
\includegraphics[width=4.25in]{figs/run_55/run_55_rt_contour}
\caption{Contour plot of $\overline{r^\prime t^\prime}$ at $z/c$=9.50, $V_{free}$=23.33, station 6.}
\end{figure}


\begin{figure}[H]
\centering
\includegraphics[width=4.25in]{figs/run_55/run_55_rw_contour}
\caption{Contour plot of $\overline{r^\prime w^\prime}$ at $z/c$=9.50, $V_{free}$=23.33, station 6.}
\end{figure}


\begin{figure}[H]
\centering
\includegraphics[width=4.25in]{figs/run_55/run_55_tw_contour}
\caption{Contour plot of $\overline{t^\prime w^\prime}$ at $z/c$=9.50, $V_{free}$=23.33, station 6.}
\end{figure}


\begin{figure}[H]
\centering
\includegraphics[width=4.25in]{figs/run_55/run_55_rr_contour}
\caption{Contour plot of $\overline{r^\prime r^\prime}$ at $z/c$=9.50, $V_{free}$=23.33, station 6.}
\end{figure}


\begin{figure}[H]
\centering
\includegraphics[width=4.25in]{figs/run_55/run_55_tt_contour}
\caption{Contour plot of $\overline{t^\prime t^\prime}$ at $z/c$=9.50, $V_{free}$=23.33, station 6.}
\end{figure}


\begin{figure}[H]
\centering
\includegraphics[width=4.25in]{figs/run_55/run_55_ww_contour}
\caption{Contour plot of $\overline{w^\prime w^\prime}$ at $z/c$=9.50, $V_{free}$=23.33, station 6.}
\end{figure}


\begin{figure}[H]
\centering
\includegraphics[width=4.25in]{figs/run_55/run_55_ctke_contour}
\caption{Contour plot of $k$ at $z/c$=9.50, $V_{free}$=23.33, station 6.}
\end{figure}


\begin{figure}[H]
\centering
\includegraphics[width=4.25in]{figs/run_55/run_55_num_contour}
\caption{Contour plot of $N$ at $z/c$=9.50, $V_{free}$=23.33, station 6.}
\end{figure}


\begin{figure}[H]
\centering
\includegraphics[width=4.25in]{figs/run_55/run_55_U_contour}
\caption{Contour plot of $\overline{u}$ at $z/c$=9.50, $V_{free}$=23.33, station 6.}
\end{figure}


\begin{figure}[H]
\centering
\includegraphics[width=4.25in]{figs/run_55/run_55_V_contour}
\caption{Contour plot of $\overline{v}$ at $z/c$=9.50, $V_{free}$=23.33, station 6.}
\end{figure}


\begin{figure}[H]
\centering
\includegraphics[width=4.25in]{figs/run_55/run_55_W_contour}
\caption{Contour plot of $\overline{w}$ at $z/c$=9.50, $V_{free}$=23.33, station 6.}
\end{figure}


\begin{figure}[H]
\centering
\includegraphics[width=6in]{figs/run_55/run_55_T_vs_r_meshscatter}
\caption{Scatter plot of azimuthal velocity vs radius at $z/c$=9.50, $V_{free}$=23.33, station6.}
\end{figure}


\begin{figure}[H]
\centering
\includegraphics[width=6in]{figs/run_55/run_55_T_vs_r_mesh_ctkescatter}
\caption{Scatter plot of azimuthal velocity vs radius, turbulent kinetic energy, at $z/c$=9.50, $V_{free}$=23.33, station 6.}
\end{figure}


\begin{figure}[H]
\centering
\includegraphics[width=6in]{figs/run_55/run_55_ctke_vs_r_meshscatter}
\caption{Scatter plot of turbulent kinetic energy vs radius at $z/c$=9.50, $V_{free}$=23.33, station 6.}
\end{figure}


\begin{figure}[H]
\centering
\includegraphics[width=6in]{figs/run_55/run_55_turb_visc_reynolds_vs_r_meshscatter}
\caption{Scatter plot of $
u_T$ reynolds stress term vs radius at $z/c$=9.50, $V_{free}$=23.33, station 6.}
\end{figure}


\begin{figure}[H]
\centering
\includegraphics[width=6in]{figs/run_55/run_55_turb_visc_vel_grad_vs_r_meshscatter}
\caption{Scatter plot of $\nu_T$ velocity gradient term vs radius at $z/c$=9.50, $V_{free}$=23.33, station 6.}
\end{figure}


\begin{figure}[H]
\centering
\includegraphics[width=6in]{figs/run_55/run_55_turb_visc_ettap_vs_r_meshscatter}
\caption{Scatter plot of $\nu_T$ pressure relaxation term vs radius at $z/c$=9.50, $V_{free}$=23.33, station 6.}
\end{figure}


\begin{figure}[H]
\centering
\includegraphics[width=6in]{figs/run_55/run_55_turb_visc_total_vs_r_meshscatter}
\caption{Scatter plot of non-equilibrium based $\nu_T$ radius at $z/c$=9.50, $V_{free}$=23.33, station 6.}
\end{figure}


\begin{figure}[H]
\centering
\includegraphics[width=6in]{figs/run_55/run_55_turb_visc_vs_r_meshscatter}
\caption{Scatter plot of $\nu_T$ as calculated from the turbulent viscosity hypothesis. $z/c$=9.50, $V_{free}$=23.33, station 6.}
\end{figure}



\input{figs/tex/run_56}
\input{figs/tex/run_57}
\input{figs/tex/run_58}
\input{figs/tex/run_59}
\input{figs/tex/run_60}

\section{Station 7 PIV Data: 1016mm Downstream}
\renewcommand\baselinestretch{1.3}\selectfont
\begin{table}[H]
\begin{center}
\begin{tabular}{|ccccccccccc|}
	\hline
	Run & $I_Z$ & $V_{nom}$ & $dt$ & $V_{fs}$ & $\sigma_{V_{fs}}$ & $Q$ & $P_{atm}$ & $T_{tunnel}$ & $\phi$ & $\eta_P$\\
	$ID$ & $(in)$ & $(m/s)$ & $(\mu s)$ & $(m/s)$ & $(m/s)$ & $(Pa)$ & $(Pa)$ & $(\degree K)$ & $(\%)$ & $(\mu s)$\\
	\hline
	61 & 1016 & 15 & 40 & 15.31 & 0.03 & 140 & 101977 & 296.15 & 52 & 0.434\\
	62 & 1016 & 17 & 40 & 16.98 & 0.04 & 171 & 101969 & 296.25 & 52 & 0.434\\
	63 & 1016 & 19 & 40 & 19.1 & 0.03 & 218 & 101961 & 296.3 & 52 & 0.434\\
	64 & 1016 & 21 & 40 & 21.15 & 0.05 & 269 & 101950 & 296.45 & 49.4 & 0.45\\
	65 & 1016 & 23 & 25 & 23.23 & 0.05 & 321 & 101953 & 296.77 & 52.6 & 0.422\\
	66 & 1016 & 25 & 25 & 24.96 & 0.05 & 371 & 101951 & 297.07 & 52.6 & 0.422\\
	67 & 1016 & 27 & 25 & 27.05 & 0.5 & 436 & 101936 & 297.25 & 52.6 & 0.422\\
	68 & 1016 & 29 & 25 & 29.17 & 0.03 & 505 & 101928 & 297.76 & 52.6 & 0.422\\
	69 & 1016 & 31 & 25 & 31.31 & 0.05 & 581 & 101921 & 297.85 & 52.6 & 0.422\\
	70 & 1016 & 33 & 25 & 33.01 & 0.05 & 647 & 101922 & 298.3 & 52.6 & 0.422\\
	\hline
\end{tabular}
\caption{Experimental measurements for station 7}
\label{table:station_7_measurements}
\end{center}
\end{table}
\renewcommand\baselinestretch{2}\selectfont

\input{figs/tex/run_61}
\input{figs/tex/run_62}
\input{figs/tex/run_63}
\input{figs/tex/run_64}
\input{figs/tex/run_65}
\begin{figure}[H]
\centering
\includegraphics[width=6in]{figs/run_66/run_66_comparison}
\caption{Theoretical vortex profile fits to experimental data at $z/c$=10.00, $V_{free}$=24.96, station 7.}
\end{figure}


\begin{figure}[H]
\centering
\includegraphics[width=4.25in]{figs/run_66/run_66_stream}
\caption{Stream plot at $z/c$=10.00, $V_{free}$=24.96, station 7.}
\end{figure}


\begin{figure}[H]
\centering
\includegraphics[width=4.25in]{figs/run_66/run_66_R_contour}
\caption{Contour plot of $\overline{v_{r}}$ at $z/c$=10.00, $V_{free}$=24.96, station 7.}
\end{figure}


\begin{figure}[H]
\centering
\includegraphics[width=4.25in]{figs/run_66/run_66_T_contour}
\caption{Contour plot of $\overline{v_{\theta}}$ at $z/c$=10.00, $V_{free}$=24.96, station 7.}
\end{figure}


\begin{figure}[H]
\centering
\includegraphics[width=4.25in]{figs/run_66/run_66_W_contour}
\caption{Contour plot of $\overline{v_{z}}$ at $z/c$=10.00, $V_{free}$=24.96, station 7.}
\end{figure}


\begin{figure}[H]
\centering
\includegraphics[width=4.25in]{figs/run_66/run_66_rt_contour}
\caption{Contour plot of $\overline{v_{r}^{\prime} v_{\theta}^{\prime}}$ at $z/c$=10.00, $V_{free}$=24.96, station 7.}
\end{figure}


\begin{figure}[H]
\centering
\includegraphics[width=4.25in]{figs/run_66/run_66_rw_contour}
\caption{Contour plot of $\overline{v_{r}^{\prime} v_{z}^{\prime}}$ at $z/c$=10.00, $V_{free}$=24.96, station 7.}
\end{figure}


\begin{figure}[H]
\centering
\includegraphics[width=4.25in]{figs/run_66/run_66_tw_contour}
\caption{Contour plot of $\overline{v_{\theta}^{\prime} v_{z}^{\prime}}$ at $z/c$=10.00, $V_{free}$=24.96, station 7.}
\end{figure}


\begin{figure}[H]
\centering
\includegraphics[width=4.25in]{figs/run_66/run_66_rr_contour}
\caption{Contour plot of $\overline{v_{r}^{\prime} v_{r}^{\prime}}$ at $z/c$=10.00, $V_{free}$=24.96, station 7.}
\end{figure}


\begin{figure}[H]
\centering
\includegraphics[width=4.25in]{figs/run_66/run_66_tt_contour}
\caption{Contour plot of $\overline{v_{\theta}^{\prime} v_{\theta}^{\prime}}$ at $z/c$=10.00, $V_{free}$=24.96, station 7.}
\end{figure}


\begin{figure}[H]
\centering
\includegraphics[width=4.25in]{figs/run_66/run_66_ww_contour}
\caption{Contour plot of $\overline{v_{z}^{\prime} v_{z}^{\prime}}$ at $z/c$=10.00, $V_{free}$=24.96, station 7.}
\end{figure}


\begin{figure}[H]
\centering
\includegraphics[width=4.25in]{figs/run_66/run_66_ctke_contour}
\caption{Contour plot of $k$ at $z/c$=10.00, $V_{free}$=24.96, station 7.}
\end{figure}


\begin{figure}[H]
\centering
\includegraphics[width=4.25in]{figs/run_66/run_66_num_contour}
\caption{Contour plot of $N$ at $z/c$=10.00, $V_{free}$=24.96, station 7.}
\end{figure}


\begin{figure}[H]
\centering
\includegraphics[width=6in]{figs/run_66/run_66_T_vs_r_meshscatter}
\caption{Scatter plot of azimuthal velocity vs radius at $z/c$=10.00, $V_{free}$=24.96, station7.}
\end{figure}


\begin{figure}[H]
\centering
\includegraphics[width=6in]{figs/run_66/run_66_T_vs_r_mesh_ctkescatter}
\caption{Scatter plot of azimuthal velocity vs radius, turbulent kinetic energy, at $z/c$=10.00, $V_{free}$=24.96, station 7.}
\end{figure}


\begin{figure}[H]
\centering
\includegraphics[width=6in]{figs/run_66/run_66_ctke_vs_r_meshscatter}
\caption{Scatter plot of turbulent kinetic energy vs radius at $z/c$=10.00, $V_{free}$=24.96, station 7.}
\end{figure}


\begin{figure}[H]
\centering
\includegraphics[width=6in]{figs/run_66/run_66_turb_visc_reynolds_vs_r_meshscatter}
\caption{Scatter plot of $
u_T$ reynolds stress term vs radius at $z/c$=10.00, $V_{free}$=24.96, station 7.}
\end{figure}


\begin{figure}[H]
\centering
\includegraphics[width=6in]{figs/run_66/run_66_turb_visc_vel_grad_vs_r_meshscatter}
\caption{Scatter plot of $\nu_T$ velocity gradient term vs radius at $z/c$=10.00, $V_{free}$=24.96, station 7.}
\end{figure}


\begin{figure}[H]
\centering
\includegraphics[width=6in]{figs/run_66/run_66_turb_visc_ettap_vs_r_meshscatter}
\caption{Scatter plot of $\nu_T$ pressure relaxation term vs radius at $z/c$=10.00, $V_{free}$=24.96, station 7.}
\end{figure}


\begin{figure}[H]
\centering
\includegraphics[width=6in]{figs/run_66/run_66_turb_visc_total_vs_r_meshscatter}
\caption{Scatter plot of non-equilibrium based $\nu_T$ vs radius at $z/c$=10.00, $V_{free}$=24.96, station 7.}
\end{figure}


\begin{figure}[H]
\centering
\includegraphics[width=6in]{figs/run_66/run_66_turb_visc_vs_r_meshscatter}
\caption{Scatter plot of $\nu_T$ vs radius as calculated from the turbulent viscosity hypothesis. $z/c$=10.00, $V_{free}$=24.96, station 7.}
\end{figure}


\begin{figure}[H]
\centering
\includegraphics[width=6in]{figs/run_66/run_66_ctke_01rdynamic}
\caption{Plot showing dynamic variations in turbulent kinetic energy within the core. $z/c$=10.00, $V_{free}$=24.96, station 7.}
\end{figure}


\begin{figure}[H]
\centering
\includegraphics[width=6in]{figs/run_66/run_66_ctke_12rdynamic}
\caption{Plot showing dynamic variations in turbulent kinetic energy between 1 and 2 core radii. $z/c$=10.00, $V_{free}$=24.96, station 7.}
\end{figure}


\begin{figure}[H]
\centering
\includegraphics[width=6in]{figs/run_66/run_66_ctke_005rdynamic}
\caption{Plot showing dynamic variations in turbulent kinetic energy between 0 and 0.5 core radii. $z/c$=10.00, $V_{free}$=24.96, station 7.}
\end{figure}



\input{figs/tex/run_67}
\input{figs/tex/run_68}
\input{figs/tex/run_69}
\input{figs/tex/run_70}	
%
\chapter{Measurement Uncertainty}
\begin{table}[H]
\begin{center}
\begin{tabular}{|ccccccccccc|}
	\hline
	Station & $dt$ & $u_{sim}$ & $v_{sim}$ & $w_{sim}$ & $\bar{u}$ & $\beta_u$ & $P_{u^{\prime}}$ & $P_{\bar{u}}$ & $U_{u^{\prime}}$ & $U_{\bar{u}}$\\
	\hline
	1 & 25 & 4.900 & 4.900 & 29.000 & 4.797 & -0.103 & 2.074 & 0.147 & 2.077 & 0.179\\
	1 & 40 & 3.300 & 3.300 & 19.000 & 3.081 & -0.219 & 1.072 & 0.076 & 1.094 & 0.231\\
	2 & 25 & 4.900 & 4.900 & 29.000 & 4.813 & -0.087 & 1.036 & 0.073 & 1.040 & 0.114\\
	2 & 40 & 3.300 & 3.300 & 19.000 & 3.374 & 0.074 & 0.954 & 0.067 & 0.957 & 0.100\\
	3 & 25 & 4.900 & 4.900 & 29.000 & 5.554 & 0.654 & 2.167 & 0.153 & 2.263 & 0.672\\
	3 & 40 & 3.300 & 3.300 & 19.000 & 3.808 & 0.508 & 1.706 & 0.121 & 1.780 & 0.522\\
	4 & 25 & 4.900 & 4.900 & 29.000 & 4.714 & -0.186 & 1.988 & 0.141 & 1.997 & 0.233\\
	4 & 40 & 3.300 & 3.300 & 19.000 & 3.075 & -0.225 & 1.007 & 0.071 & 1.032 & 0.236\\
	5 & 25 & 4.900 & 4.900 & 29.000 & 4.820 & -0.080 & 1.358 & 0.096 & 1.360 & 0.125\\
	5 & 40 & 3.300 & 3.300 & 19.000 & 3.154 & -0.146 & 0.591 & 0.042 & 0.609 & 0.151\\
	6 & 25 & 4.900 & 4.900 & 29.000 & 4.819 & -0.081 & 0.896 & 0.063 & 0.899 & 0.103\\
	6 & 40 & 3.300 & 3.300 & 19.000 & 3.255 & -0.045 & 0.659 & 0.047 & 0.660 & 0.065\\
	7 & 25 & 4.900 & 4.900 & 29.000 & 5.053 & 0.153 & 1.431 & 0.101 & 1.439 & 0.184\\
	7 & 40 & 3.300 & 3.300 & 19.000 & 3.614 & 0.314 & 1.528 & 0.108 & 1.560 & 0.332\\
	\hline
\end{tabular}
\caption{Uncertainty in $X$ direction velocity measurements. Unlabelled units are $m/s$.}
\label{table:uncertainties_u}
\end{center}
\end{table}

\renewcommand\baselinestretch{1.3}\selectfont
\begin{table}[H]
\begin{center}
\begin{tabular}{|ccccccccccc|}
	\hline
	Station & $dt$ & $u_{sim}$ & $v_{sim}$ & $w_{sim}$ & $\bar{v}$ & $\beta_v$ & $P_{v^{\prime}}$ & $P_{\bar{v}}$ & $U_{v^{\prime}}$ & $U_{\bar{v}}$\\
	\hline
	1 & 25 & 6.15 & 6.15 & 29.0 & 5.026 & -1.124 & 0.776 & 0.055 & 1.366 & 1.126\\
	1 & 40 & 5.6 & 5.6 & 19.0 & 4.168 & -1.432 & 0.671 & 0.047 & 1.581 & 1.432\\
	2 & 25 & 6.15 & 6.15 & 29.0 & 5.682 & -0.468 & 0.775 & 0.055 & 0.905 & 0.471\\
	2 & 40 & 5.6 & 5.6 & 19.0 & 4.832 & -0.768 & 0.705 & 0.050 & 1.043 & 0.770\\
	3 & 25 & 6.15 & 6.15 & 29.0 & 5.536 & -0.614 & 0.641 & 0.045 & 0.888 & 0.616\\
	3 & 40 & 5.6 & 5.6 & 19.0 & 4.866 & -0.734 & 0.511 & 0.036 & 0.894 & 0.735\\
	4 & 25 & 6.15 & 6.15 & 29.0 & 5.549 & -0.601 & 0.711 & 0.050 & 0.931 & 0.604\\
	4 & 40 & 5.6 & 5.6 & 19.0 & 4.973 & -0.627 & 0.720 & 0.051 & 0.955 & 0.629\\
	5 & 25 & 6.15 & 6.15 & 29.0 & 5.721 & -0.429 & 0.662 & 0.047 & 0.789 & 0.432\\
	5 & 40 & 5.6 & 5.6 & 19.0 & 5.153 & -0.447 & 0.619 & 0.044 & 0.764 & 0.449\\
	6 & 25 & 6.15 & 6.15 & 29.0 & 5.828 & -0.322 & 0.663 & 0.047 & 0.738 & 0.326\\
	6 & 40 & 5.6 & 5.6 & 19.0 & 5.266 & -0.334 & 0.523 & 0.037 & 0.620 & 0.336\\
	7 & 25 & 6.15 & 6.15 & 29.0 & 6.131 & -0.019 & 0.709 & 0.050 & 0.709 & 0.054\\
	7 & 40 & 5.6 & 5.6 & 19.0 & 5.618 & 0.018 & 0.555 & 0.039 & 0.555 & 0.043\\
	\hline
\end{tabular}
\caption{Uncertainty in $Y$ direction velocity measurements. Unlabelled units are $m/s$.}
\label{table:uncertainties_v}
\end{center}
\end{table}
\renewcommand\baselinestretch{2}\selectfont

\begin{table}[H]
\begin{center}
\begin{tabular}{|ccccccccccc|}
	\hline
	Station & $dt$ & $u_{sim}$ & $v_{sim}$ & $w_{sim}$ & $\bar{w}$ & $\beta_w$ & $P_{w^{\prime}}$ & $P_{\bar{w}}$ & $U_{w^{\prime}}$ & $U_{\bar{w}}$\\
	\hline
	1 & 25 & 4.900 & 4.900 & 29.000 & 18.598 & -10.402 & 0.770 & 0.054 & 10.431 & 10.402\\
	1 & 40 & 3.300 & 3.300 & 19.000 & 11.834 & -7.166 & 0.372 & 0.026 & 7.176 & 7.166\\
	2 & 25 & 4.900 & 4.900 & 29.000 & 19.882 & -9.118 & 0.663 & 0.047 & 9.142 & 9.118\\
	2 & 40 & 3.300 & 3.300 & 19.000 & 12.993 & -6.007 & 0.684 & 0.048 & 6.045 & 6.007\\
	3 & 25 & 4.900 & 4.900 & 29.000 & 22.271 & -6.729 & 1.752 & 0.124 & 6.954 & 6.730\\
	3 & 40 & 3.300 & 3.300 & 19.000 & 15.310 & -3.690 & 0.928 & 0.066 & 3.805 & 3.691\\
	4 & 25 & 4.900 & 4.900 & 29.000 & 25.811 & -3.189 & 0.948 & 0.067 & 3.327 & 3.189\\
	4 & 40 & 3.300 & 3.300 & 19.000 & 16.657 & -2.343 & 0.358 & 0.025 & 2.371 & 2.344\\
	5 & 25 & 4.900 & 4.900 & 29.000 & 26.704 & -2.296 & 0.562 & 0.040 & 2.364 & 2.297\\
	5 & 40 & 3.300 & 3.300 & 19.000 & 17.006 & -1.994 & 0.297 & 0.021 & 2.016 & 1.994\\
	6 & 25 & 4.900 & 4.900 & 29.000 & 27.204 & -1.796 & 0.503 & 0.036 & 1.865 & 1.796\\
	6 & 40 & 3.300 & 3.300 & 19.000 & 17.398 & -1.602 & 0.502 & 0.036 & 1.679 & 1.602\\
	7 & 25 & 4.900 & 4.900 & 29.000 & 28.094 & -0.906 & 0.953 & 0.067 & 1.314 & 0.908\\
	7 & 40 & 3.300 & 3.300 & 19.000 & 18.626 & -0.374 & 0.931 & 0.066 & 1.003 & 0.380\\
	\hline
\end{tabular}
\caption{Uncertainty in $Z$ direction velocity measurements. Unlabelled units are $m/s$.}
\label{table:uncertainties_w}
\end{center}
\end{table}


%\section{Station 1 Uncertainty: 546mm Downstream}

%\section{Station 2 PIV Data: 708mm Downstream}

%\section{Station 3 PIV Data: 787mm Downstream}

%\section{Station 4 PIV Data: 863mm Downstream}

%\section{Station 5 PIV Data: 914mm Downstream}

%\section{Station 6 PIV Data: 965mm Downstream}

%\section{Station 7 Uncertainty: 1016mm Downstream}


% starts blank page
\newpage

% initiates printing of the vita page
\vitapage

\end{document}
